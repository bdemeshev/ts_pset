\documentclass[11pt]{book}
\usepackage{libertine}


% this magick is to solve problem that appeared after update of texlive 2018 to texlive 2020
% https://tex.stackexchange.com/questions/511341/the-error-occurred-after-the-last-update
\makeatletter
\def\nobreak{\penalty\@M}
\makeatother
 

%%%%% russian xetex
\usepackage{fontspec}
\usepackage{polyglossia}
\usepackage{todonotes}

\setmainlanguage{russian}
\setotherlanguages{english}

% download "Linux Libertine" fonts:
% http://www.linuxlibertine.org/index.php?id=91&L=1
% \setmainfont{Linux Libertine O} % or Helvetica, Arial, Cambria
% why do we need \newfontfamily:
% http://tex.stackexchange.com/questions/91507/
% \newfontfamily{\cyrillicfonttt}{Linux Libertine O}
%%%%% end russian xetex


\usepackage{tikz}
\usepackage{minted} % TODO: перейти на listings
\usepackage{physics}


\usepackage{amsmath, amssymb, amsthm, mathrsfs, amsfonts, dsfont, fancyhdr}
\usepackage{amscd}
\usepackage{makeidx}
\usepackage[colorlinks = true]{hyperref}

\usepackage{cancel}

\usepackage[left = 2cm, right = 2cm, top = 2cm, bottom = 2cm]{geometry}

\usepackage{enumitem}
\usepackage{answers} % дележка ответов и вопросов

\newcommand{\e}{\varepsilon}
\newcommand{\cF}{\mathcal{F}}
\newcommand{\cH}{\mathcal{H}}
\newcommand{\hb}{\hat{\beta}}
\newcommand{\hs}{\hat{\sigma}}
\newcommand{\he}{\hat{\e}}
\newcommand{\hy}{\hat{y}}
\newcommand{\cN}{\mathcal{N}}


\DeclareMathOperator{\E}{\mathbb{E}}
\DeclareMathOperator{\Cov}{\mathbb{C}\mathrm{ov}}
\DeclareMathOperator{\Corr}{\mathbb{C}\mathrm{orr}}
\DeclareMathOperator{\pCorr}{\mathrm{p}\mathbb{C}\mathrm{orr}}
\DeclareMathOperator{\Var}{\mathbb{V}\mathrm{ar}}
\DeclareMathOperator{\plim}{plim}

\let\P\relax
\DeclareMathOperator{\P}{\mathbb{P}}



\usepackage[bibencoding = auto,
style = alphabetic,
backend = biber,
citestyle = alphabetic,
sorting = none]{biblatex}

\addbibresource{ts_pset_main.bib}


% не включаем в текст, тк есть репозиторий, однако для облегчения работы с репой:
% makefile: zip all tex, здесь сделать ссылку на zip - для воспроизводимости неплохо бы :)

\usepackage[matrix, arrow, curve]{xy}
\usepackage{soul}
\usepackage{color}


\title{Анализ временных рядов в задачах и упражнениях}
\author{Борзых\,Д.\,А., Демешев\,Б.\,Б.}
\date{\today}

\newtheoremstyle{mythm}{\topsep}{\topsep}{\it}{\parindent}{\bfseries}{.}{.5em}{}
\newtheoremstyle{mydef}{\topsep}{\topsep}{}{\parindent}{\bfseries}{.}{.5em}{}


\theoremstyle{mythm}
\newtheorem{Proposition}{Утверждение}[chapter]
\newtheorem{Lemma}{Лемма}[chapter]
\newtheorem{Theorem}{Теорема}[chapter]
\newtheorem{Corollary}{Следствие}[chapter]


\theoremstyle{mydef}
\newtheorem{Definition}{Определение}[chapter]
\newtheorem{Example}{Пример}[chapter]
\newtheorem{Task}{Задача}[chapter]
\newtheorem{Exercise}{Упражнение}[chapter]
\newtheorem{Remark}{Замечание}[chapter]
\newtheorem{Question}{Вопрос}[chapter]

%\newtheorem{solution}{Решение}
\newtheorem{question}{Задача}


\renewenvironment{proof}
    {\ifhmode\par\fi\addvspace{\topsep}\leavevmode{\it Доказательство.}}
    {\hfill$\scriptstyle\blacksquare$\par\addvspace{\topsep}}

% \newenvironment{solution}
%     {\ifhmode\par\fi\addvspace{\topsep}\leavevmode{\it Решение.}}
%     {\hfill$\scriptstyle\blacksquare$\par\addvspace{\topsep}}

\pagestyle{fancy}

\renewcommand{\chaptermark}[1]{\markboth{\chaptername\ \thechapter.\ #1}{}}
\renewcommand{\sectionmark}[1]{\markright{\thesection.\ #1}}

\fancyhead[LE,RO]{\small \thepage}
\fancyhead[RE]{\small \leftmark}
\fancyhead[LO]{\small \rightmark} %{\bfseries \leftmark}

\makeindex


\theoremstyle{definition}
%\newtheorem{problem}{Задача}
%\numberwithin{problem}{chapter}

\Newassociation{sol}{solution}{solution_file}
% sol --- имя окружения внутри задач
% solution --- имя окружения внутри solution_file
% solution_file --- имя файла в который будет идти запись решений
% можно изменить далее по ходу

% very useful during de-bugging!
% \usepackage[left]{showlabels}
% \showlabels{hypertarget}
% \showlabels{hyperlink}

\newlist{myenum}{enumerate}{3}
\newcounter{problem}[chapter]
\newenvironment{problem}%
  {%
  \refstepcounter{problem}%
  %  hyperlink to solution
       \hypertarget{problem:{\thechapter.\theproblem}}{}%
       \Writetofile{solution_file}{\protect\hypertarget{soln:\thechapter.\theproblem}{}}%
       \begin{myenum}[label=\bfseries\protect\hyperlink{soln:\thechapter.\theproblem}{\thechapter.\theproblem},ref=\thechapter.\theproblem]
       \item%
      }%
      {%
      \end{myenum}}



\AddEnumerateCounter{\asbuk}{\russian@alph}{щ} % для списков с русскими буквами
\setlist[enumerate, 1]{label=\asbuk*),ref=\asbuk*} % цифра рядом с enumerate = уровень нумерации
      


\begin{document}
\maketitle

\tableofcontents


\Opensolutionfile{solution_file}[solutions_all]

\Opensolutionfile{solution_file}[solutions/sols_005]
% в квадратных скобках фактическое имя файла

\chapter{Автокорреляция ошибок в линейной модели}

\begin{problem}
Билл Гейтс оценил модель $y_t=\beta_1 + \beta_2 t + \beta_3 y_{t-1} + \e_t$ с помощью МНК. Значение статистики Дарбина-Уотсона оказалось равно $DW=0.55$. Какой из этого следует вывод об автокорреляции ошибок первого порядка?


\begin{sol}
В данном случае статистика $DW$ не применима, так как есть лаг $y_{t-1}$ среди регрессоров.
\end{sol}
\end{problem}


\begin{problem}
Рассмотрим модель $y_t=\beta x_t +\e_t$, где $\e_1=u_1$ и $\e_t=u_t+u_{t-1}$ при $t\geq 2$. Случайные величины $u_i$ независимы с $\E(u_i)=0$ и $\Var(u_i)=\sigma^2$.
\begin{enumerate}
\item Найдите $\Var(\e_t)$
\item Являются ли ошибки $\e_t$ гетероскедастичными?
\item Найдите $\Cov(\e_i,\e_j)$
\item Являются ли ошибки $\e_t$ автокоррелированными?
\item Как выглядит матрица $\Var(\e)$?
\item Рассмотрим оценку
\[
\hb=\frac{\sum x_i y_i}{\sum x_i^2}
\]
Является ли она несмещенной для $\beta$? Является ли она эффективной в классе линейных по $y$ несмещенных оценок?
\item Если приведенная $\hb$ не является эффективной, то приведите формулу для эффективной оценки.
\end{enumerate}



\begin{sol}
\begin{enumerate}
\item $\E(\e_t)=0$, $\Var(\e_1)=\sigma^2$, $\Var(\e_t)=2\sigma^2$ при $t\geq 2$.  Гетероскедастичная.
\item $\Cov(e_t,e_{t+1})=\sigma^2$. Автокоррелированная.
\item $\hb$ --- несмещенная, неэффективная
\item Более эффективной будет $\hb_{gls}=(X'V^{-1}X)^{-1}X'V^{-1}y$, где
\[
X=\begin{pmatrix}
x_1 \\
x_2 \\
\vdots \\
x_n
\end{pmatrix}
\]

Матрица $V$ известна с точностью до константы $\sigma^2$, но в формуле для $\hb_{gls}$ неизвестная $\sigma^2$ сократится.

Другой способ построить эффективную оценку --- применить МНК к преобразованным наблюдениям, т.е. $\hb_{gls}=\frac{\sum x'_i y'_i}{\sum x_i^{\prime 2}}$, где $y'_1=y_1$, $x'_1=x_1$, $y'_t=y_t-y_{t-1}$, $x'_t=x_t-x_{t-1}$ при $t\geq 2$.
\end{enumerate}
\end{sol}
\end{problem}




\begin{problem}
Имеются данные $y=(1,\, 2,\, 0,\,  0,\, 2,\, 1)$. Предполагая модель с автокоррелированной ошибкой, $y_t=\mu+\e_t$, где $\e_t=\rho \e_{t-1}+u_t$ с помощью трёх тестов проверьте гипотезы
$H_0$: $\rho=0$,
$H_0$: $\mu=0$,
$H_0$: $\begin{cases}
\rho=0 \\
\mu = 0 \\
\sigma^2=1
\end{cases}$



\begin{sol}

Для простоты закроем глаза на малое количество наблюдений и как индейцы пираха будем считать, что пять --- это много.

\end{sol}
\end{problem}


\begin{problem}
Рассматривается модель $y_t = \mu + \varepsilon_t$, $t = 1,\ldots,T$, где $\varepsilon_t = \rho \varepsilon_{t-1} + u_t$, случайные величины $\varepsilon_0, u_1,\dots,u_T$ независимы, причем $\varepsilon_0 \sim N(0,\sigma^2/(1 - \rho^2))$, $u_t \sim N(0,\sigma^2)$. Имеются наблюдения $y' = (1, 2, 0, 0, 1)$.
\begin{enumerate}
  \item Выпишите функцию правдоподобия
  \[
  \mathrm{L}(\mu, \rho, \sigma^2) = f_{Y_1}(y_1)\prod_{t=2}^{T}f_{Y_t|Y_{t-1}}(y_t|y_{t-1}).
  \]
  \item Найдите оценки неизвестных параметров модели максимизируя условную функцию правдоподобия
  \[
  \mathrm{L}(\mu, \rho, \sigma^2|Y_1 = y_1) = \prod_{t=2}^{T}f_{Y_t|Y_{t-1}}(y_t|y_{t-1})
  \]
\end{enumerate}


\begin{sol}
1. Поскольку имеют место соотношения $\varepsilon_1 = \rho \varepsilon_0 + u_1$ и $Y_1 =\mu + \varepsilon_1$, то из условия задачи получаем, что $\varepsilon_1 \sim N(0,\sigma^2 / (1 - \rho^2))$
и $Y_1 \sim N(\mu,\sigma^2 / (1 - \rho^2))$. Поэтому
\[
f_{Y_1}(y_1) = \frac{1}{\sqrt{2\pi\sigma^2/(1-\rho^2)}}\exp{\left(-\frac{(y_1 - \mu)^2}{2\sigma^2/(1 - \rho^2)}\right)}.
\]

Далее, найдем $f_{Y_2|Y_1}(y_2|y_1)$. Учитывая, что $Y_2 = \rho Y_1 + (1- \rho) \mu + u_2$, получаем $Y_2|\{Y_1 = y_1\} \sim N(\rho y_1 + (1- \rho) \mu, \sigma^2)$. Значит,
\[
f_{Y_2|Y_1}(y_2|y_1) = \frac{1}{\sqrt{2\pi\sigma^2}}\exp{\left(-\frac{(y_2 - \rho y_1 - (1- \rho) \mu)^2}{2\sigma^2}\right)}.
\]

Действуя аналогично, получаем, что для всех $t \geq 2$ справедлива формула
\[
f_{Y_{t}|Y_{t-1}}(y_{t}|y_{t-1}) = \frac{1}{\sqrt{2\pi\sigma^2}}\exp{\left(-\frac{(y_{t} - \rho y_{t-1} - (1- \rho) \mu)^2}{2\sigma^2}\right)}.
\]

Таким образом, находим функцию правдоподобия
\[
\mathrm{L}(\mu, \rho, \sigma^2) = f_{Y_T,\ldots,Y_1}(y_T,\dots,y_1) = f_{Y_1}(y_1)\prod_{t=2}^{T}f_{Y_t|Y_{t-1}}(y_t|y_{t-1}) \text{,}
\]
где $f_{Y_1}(y_1)$ и $f_{Y_t|Y_{t-1}}(y_t|y_{t-1})$ получены выше.

2. Для нахождения неизвестных параметров модели запишем логарифмическую условную функцию правдоподобия:
\[
l(\mu, \rho, \sigma^2|Y_1 = y_1) = \sum_{t=2}^{T}\log{f_{Y_t|Y_{t-1}}(y_t|y_{t-1})} =
\]
\[
=-\frac{T-1}{2} \log(2 \pi) - \frac{T-1}{2} \log{\sigma^2} - \frac{1}{2\sigma^2} \sum_{t=2}^{T}(y_t - \rho y_{t-1} - (1 - \rho) \mu)^2 \text{.}
\]

Найдем производные функции $l(\mu, \rho, \sigma^2|Y_1 = y_1)$ по неизвестным параметрам:
\[
\frac{\partial l}{\partial \mu} = -\frac{1}{2\sigma^2} \sum_{t=2}^{T} 2(y_t - \rho y_{t-1} - (1 - \rho) \mu) \cdot (\rho - 1) \text{,}
\]
\[
\frac{\partial l}{\partial \rho} = -\frac{1}{2\sigma^2} \sum_{t=2}^{T} 2(y_t - \rho y_{t-1} - (1 - \rho) \mu) \cdot (\mu - y_{t-1}) \text{,}
\]
\[
\frac{\partial l}{\partial {\sigma^2}} =  - \frac{T-1}{2\sigma^2} + \frac{1}{2\sigma^4} \sum_{t=2}^{T}(y_t - \rho y_{t-1} - (1 - \rho) \mu)^2 \text{.}
\]

Оценки неизвестных параметров модели могут быть получены как решение следующей системы уравнений:
\[
\left\{
  \begin{aligned}
    \frac{\partial l}{\partial \mu} = 0 \text{,} \\
    \frac{\partial l}{\partial \rho} = 0 \text{,} \\
    \frac{\partial l}{\partial {\sigma^2}} = 0 \text{.}
  \end{aligned}
\right.
\]

Из первого уравнения системы получаем, что
\[
\sum_{t=2}^{T}y_{t} - \hat{\rho} \sum_{t=2}^{T}y_{t-1} = (T - 1) (1- \hat{\rho}) \hat{\mu} \text{,}
\]
откуда
\[
\hat{\mu} = \frac{\sum_{t=2}^{T}y_{t} - \hat{\rho} \sum_{t=2}^{T}y_{t-1}}{(T - 1) (1- \hat{\rho})} = \frac{3 - \hat{\rho} \cdot 3}{4\cdot(1-\hat{\rho})} = \frac{3}{4} \text{.}
\]

Далее, если второе уравнение системы переписать в виде
\[
\sum_{t=2}^{T}(y_t - \hat{\mu} - \hat{\rho} (y_{t-1} - \hat{\mu}))(y_{t-1} - \hat{\mu}) = 0 \text{,}
\]
то легко видеть, что
\[
\hat{\rho} = \frac{\sum_{t=2}^{T}(y_t - \hat{\mu})(y_{t-1} - \hat{\mu})}{\sum_{t=2}^{T}(y_{t-1} - \hat{\mu})^2} \text{.}
\]
Следовательно, $\hat{\rho} =-1/11= -0.0909$.

Наконец, из третьего уравнения системы
\[
\hs^2 =\frac{1}{T-1} \sum_{t=2}^{T}(y_t - \hat{\rho} y_{t-1} - (1 - \hat{\rho}) \hat{\mu})^2 \text{.}
\]
Значит, $\hs^2 = 165/242= 0.6818$. Ответы: $\hat{\mu} = 3/4= 0.75$, $\hat{\rho} = -1/11=-0.0909$, $\hs^2 =165/242=0.6818$.
\end{sol}
\end{problem}




\begin{problem}
Остаются ли в условиях автокорреляции МНК-
оценки в линейной модели несмещёнными? Состоятельными?
\begin{sol}
Несмещёнными остаются. Состоятельными не всегда остаются, например, состоятельность исчезает, если все случайные ошибки тождественно равны между собой.

\end{sol}
\end{problem}



\begin{problem}
Продавец мороженного оценил динамическую модель объёмов продаж:
\[
\ln \hat{Q}_t=26.7 + 0.2\ln \hat{Q}_{t-1}-0.6\ln P_t
\]
Здесь $Q_t$ --- число проданных в день $t$ вафельных стаканчиков, а $P_t$ --- цена одного стаканчика в рублях. Продавец также рассчитал остатки $\hat{e}_t$.
\begin{enumerate}
\item Чему, согласно полученным оценкам, равна долгосрочная эластичность объёма продаж по цене?
\item Предположим, что продавец решил проверить наличие автокорреляции первого порядка с помощью теста Бройша-Годфри. Выпишите уравнение регрессии, которое он должен оценить.
\end{enumerate}


\begin{sol}
\end{sol}
\end{problem}


\begin{problem}
Пусть $u_t$ --- независимые нормальные случайные величины с
математическим ожиданием $0$ и дисперсией $\sigma^2$. Известно, что $\e_1=u_1$, $\e_t=u_1+u_2+\ldots+u_t$. Рассмотрим модель $y_t=\beta_1+\beta_2 x_t + \e_t$.

\begin{enumerate}
\item Найдите $\Var(\e_t)$, $\Cov(\e_t,\e_s)$, $\Var(\e)$
\item Являются ли ошибки $\e_t$ гетероскедастичными?
\item Являются ли ошибки $\e_t$ автокоррелированными?
\item Предложите более эффективную оценку вектора коэффициентов регрессии по сравнению МНК-оценкой.
\item Результаты предыдущего пункта подтвердите симуляциями Монте-Карло на компьютере.
\end{enumerate}


\begin{sol}
\end{sol}
\end{problem}


\begin{problem}
Ошибки в модели $y_t=\beta_1+\beta_2 x_{t}+\e_t$ являются автокоррелированными первого порядка, $\e_t=\rho \e_{t-1}+u_t$. Шаман-эконометрист Ойуун выполняет два камлания-преобразования. Поясните смысл камланий:
\begin{enumerate}
\item Камлание А, при $t\geq 2$, Ойуун преобразует уравнение к виду $y_t-\rho y_{t-1}=\beta_1(1-\rho)+ \beta_2(x_t-\rho x_{t-1})+\e_t-\rho \e_{t-1}$
\item Камлание Б, при $t=1$, Ойуун преобразует уравнение к виду $\sqrt{1-\rho^2}y_1=\sqrt{1-\rho^2}\beta_1+\sqrt{1-\rho^2}\beta_2 x_1+\sqrt{1-\rho^2}\e_1$.
\end{enumerate}


\begin{sol}
\end{sol}
\end{problem}


\begin{problem}
Рассмотрим модель $y_t=\beta_1+\beta_2 x_{t1}+\ldots+\beta_k x_{tk}+\e_t$, где $\e_t$ подчиняются автокорреляционной схеме первого порядка, т.е.
\begin{enumerate}
\item $\e_t=\rho \e_{t-1}+u_t$, $-1<\rho<1$
\item $\Var(\e_t)=const$, $\E(\e_t)=const$
\item $\Var(u_t)=\sigma^2$, $\E(u_t)=0$
\item Величины $u_t$ независимы между собой
\item Величины $u_t$ и $\e_s$ независимы, если $t\geq s$
\end{enumerate}
Найдите:
\begin{enumerate}
\item $\E(\e_t)$, $\Var(\e_t)$
\item $\Cov(\e_t,\e_{t+h})$
\item $\Corr(\e_t,\e_{t+h})$
\end{enumerate}


\begin{sol}
\begin{enumerate}
\item $\E(\e_t)=0$, $\Var(\e_t)=\sigma^2/(1-\rho^2)$
\item $\Cov(\e_t,\e_{t+h})=\rho^h\cdot \sigma^2/(1-\rho^2)$
\item $\Corr(\e_t,\e_{t+h})=\rho^h$
\end{enumerate}
\end{sol}
\end{problem}



\begin{problem}
Рассматривается модель $y_t=\beta_1+\beta_2 x_{t1}+\ldots+\beta_k x_{tk}+\e_t$. Ошибки $\e_t$ гомоскедастичны, но в них возможно присутствует автокорреляция первого порядка, $\e_t=\rho \e_{t-1}+u_t$. При известном числе наблюдений $T$ на уровне значимости 5\% сделайте статистический вывод о наличии автокорреляции.
\begin{enumerate}
\item $T=25$, $k=2$, $DW=0.8$
\item $T=30$, $k=3$, $DW=1.6$
\item $T=50$, $k=4$, $DW=1.8$
\item $T=100$, $k=5$, $DW=1.1$
\end{enumerate}


\begin{sol}
\end{sol}
\end{problem}



\begin{problem}
По 100 наблюдениям была оценена модель линейной регрессии
$y_t=\beta_1+\beta_2 x_t+\e_t$. Оказалось, что $RSS=120$, $\he_1=-1$, $\he_{100}=2$, $\sum_{t=2}^{100} \he_t\he_{t-1}=-50$. Найдите $DW$ и $\rho$.


\begin{sol}
\end{sol}
\end{problem}



\begin{problem}
Применяется ли статистика Дарбина-Уотсона для выявления автокорреляции в следующих моделях
\begin{enumerate}
\item $y_t=\beta_1 x_t + \e_t$
\item $y_t=\beta_1 + \beta_2 x_t + \e_t$
\item $y_t=\beta_1 + \beta_2 y_{t-1} + \e_t$
\item $y_t=\beta_1 + \beta_2 t +\beta_3 y_{t-1} + \e_t$
\item $y_t=\beta_1 t + \beta_2 x_t + \e_t$
\item $y_t=\beta_1 + \beta_2 t +\beta_3 x_t +\beta_4 x_{t-1} + \e_t$
\end{enumerate}


\begin{sol}
\end{sol}
\end{problem}



\begin{problem}
По 21 наблюдению была оценена модель линейной регрессии
$\underset{(se)}{\hat{y}}=\underset{(0.3)}{1.2}+\underset{(0.18)}{0.9}\cdot y_{t-1}+\underset{(0.01)}{0.1}\cdot t$, $R^2=0.6$, $DW=1.21$. Протестируйте гипотезу об отсутствии автокорреляции ошибок на уровне значимости 5\%.


\begin{sol}
\end{sol}
\end{problem}




\begin{problem}
По 24 наблюдениям была оценена модель линейной регрессии
$\underset{(se)}{\hat{y}}=\underset{(0.01)}{0.5}+\underset{(0.02)}{2}\cdot t$, $R^2=0.9$, $DW=1.3$. Протестируйте гипотезу об отсутствии автокорреляции ошибок на уровне значимости 5\%.


\begin{sol}
\end{sol}
\end{problem}



\begin{problem}
По 32 наблюдениям была оценена модель линейной регрессии
$\underset{(se)}{\hat{y}}=\underset{(2.5)}{10}+\underset{(0.5)}{2.5}\cdot t- \underset{(0.01)}{0.1}\cdot t^2$, $R^2=0.75$, $DW=1.75$. Протестируйте гипотезу об отсутствии автокорреляции ошибок на уровне значимости 5\%.


\begin{sol}
\end{sol}
\end{problem}




\Closesolutionfile{solution_file}



\Opensolutionfile{solution_file}[solutions/sols_010]
% в квадратных скобках фактическое имя файла

\chapter{Стационарные процессы}

Сюда относятся задачи на стационарность до явного упоминания ARMA/ARIMA :)

\begin{problem}
Запишите процесс $y_t = 4 + 0.4y_{t-1} + 0.3\e_{t-1} + \e_t$ с помощью оператора лага.
\begin{sol}
\[
(1 - 0.4L)y_t = 4 + (1 + 0.3L)\e_t
\]
\end{sol}
\end{problem}


\begin{problem}
Пусть $x_{t}$, $t=0,1,2, \ldots$ — случайный процесс и $y_{t}=(1+L)^{t}x_{t}$.
Выразите $x_{t}$ с помощью $y_{t}$ и оператора лага $L$.

\begin{sol}
$x_{t}=(1-L)^{t}y_{t}$
\end{sol}
\end{problem}

\begin{problem}
Пусть $ F_{n} $ — последовательность чисел Фибоначчи. Рассмотрим величину
\[
\frac{F_{101}+C^{1}_{5}F_{102}+C^{2}_{5}F_{103}+C^{3}_{5}F_{104}+C^{4}_{5}F_{105}+C^{5}_{5}F_{106}}
{F_{111}}
\]
\begin{enumerate}
\item Запишите величину с помощью оператора лага
\item Упростите величину
\end{enumerate}

\begin{sol}
$ F_{n}=L(1+L)F_{n} $, значит $ F_{n}=L^{k}(1+L)^{k}F_{n} $ или $ F_{n+k}=(1+L)^{k}F_{n} $

Ответ: $1$
\end{sol}
\end{problem}




\begin{problem}
Пусть $x_{t}$, $t=\ldots -2,-1,0,1,2,\ldots $ — случайный процесс. И $y_{t}=x_{-t}$. Какое рассуждение верно?

\begin{enumerate}
\item $Ly_{t}=Lx_{-t}=x_{-t-1}$;
\item $Ly_{t}=y_{t-1}=x_{-t+1}$;
\item $x_t L y_t = x_t y_{t-1}$;
\item $x_t L y_t = x_{t-1} y_t$;
\end{enumerate}
\begin{sol}
а — неверно, б — верно, в — верно, г — нет.
\end{sol}
\end{problem}

%%%%%%%%%%%%%%%%% стационарность

\begin{problem}
Пусть $y_{t}$ — стационарный процесс. Верно ли, что стационарны:
\begin{enumerate}
\item $z_{t}=2y_{t}$
\item $z_{t}=y_{t}+1$
\item $z_{t}=\Delta y_{t}$
\item $z_{t}=2y_{t}+3y_{t-1}$
\end{enumerate}
\begin{sol}
а, б, в, г — стационарны
\end{sol}
\end{problem}





\begin{problem}
Известно, что временной ряд $y_{t}$ порожден стационарным процессом, задаваемым соотношением $y_{t}=1+0.5y_{t-1}+\varepsilon_{t}$. Имеется 1000 наблюдений.


Вася построил регрессию $y_{t}$ на константу и $y_{t-1}$. Петя построил регрессию на константу и $y_{t+1}$.


Как примерно будут соотносится между собой их оценки коэффициентов?
\begin{sol}
Они будут примерно одинаковы. Оценка наклона определяется автоковариационной функцией.
\end{sol}
\end{problem}


\begin{problem}
Правильный кубик подбрасывают три раза, обозначим результаты подбрасываний $X_1$, $X_2$ и $X_3$. Также ввёдем обозначения для сумм $L=X_1+X_2$, $R=X_2+X_3$ и $S=X_1+X_2+X_3$.
\begin{enumerate}
\item Интуитивно, без вычислений, определите знак обычных и частных корреляций $\Corr(L, R)$, $\Corr(L, S)$, $\pCorr(L, R; S)$, 
  $\pCorr(L, S; R)$, $\Corr(X_1, R)$, $\pCorr(X_1, R; S)$, $\pCorr(X_1, R; L)$, $\pCorr(L, R; X_2)$, $\pCorr(L, R; X_1)$;  
\item Какие из корреляций по модулю равны единице?
\item Найдите все упомянутые обычные и частные корреляции.
\end{enumerate}
\begin{sol}

\end{sol}
\end{problem}


\begin{problem}
Известно, что $\e_t$ — белый шум. У каких разностных уравнений есть слабо стационарные решения?
\begin{enumerate}
\item $y_t=1+\e_t+0.5\e_{t-1}+0.25\e_{t-2}$
\item $y_t=-2y_{t-1}-3y_{t-2}+\e_t+\e_{t-1}$
\item $y_t=-0.5y_{t-1} + \e_t$
\item $y_t=1-1.5 y_{t-1} - 0.5 y_{t-2} + \e_t - 1.5\e_{t-1} - 0.5\e_{t-2}$
\item $y_t=1+0.64y_{t-2}+\e_t+0.64\e_{t-1}$
\item $y_t=1+t+\e_t$
\item $y_t=1+y_{t-1}+\e_t$
\end{enumerate}
\begin{sol}

\begin{enumerate}
\item $y_t=1+\e_t+0.5\e_{t-1}+0.25\e_{t-2}$ — стационарный
\item $y_t=-2y_{t-1}-3y_{t-2}+\e_t+\e_{t-1}$
\item $y_t=-0.5y_{t-1} + \e_t$ — стационарный
\item $y_t=1-1.5 y_{t-1} - 0.5 y_{t-2} + \e_t - 1.5\e_{t-1} - 0.5\e_{t-2}$
\item $y_t=1+0.64y_{t-2}+\e_t+0.64\e_{t-1}$ — стационарный
\item $y_t=1+t+\e_t$ — нестационарный
\item $y_t=1+y_{t-1}+\e_t$ — нестационарный
\end{enumerate}
\end{sol}
\end{problem}







\begin{problem}
Пусть $\e_t$ — белый шум. Рассмотрим процесс $y_t=2+0.5y_{t-1}+\e_t$ с различными начальными условиями, указанными ниже.

\begin{enumerate}
\item Найдите $\E(y_t)$, $\Var(y_t)$ и определите, является ли процесс  стационарным, если:
\begin{enumerate}
\item $y_1=0$
\item $y_1=4$
\item $y_1=4+\e_1$
\item $y_1=4+\frac{2}{\sqrt{3}}\e_1$
\end{enumerate}
\item Как точно следует понимать фразу «процесс $y_t=2+0.5y_{t-1}+\e_t$ является стационарным»?
\end{enumerate}




\begin{sol}
Процесс стационарен только при $y_1=4+\frac{2}{\sqrt{3}}\e_1$. Фразу нужно понимать как «у стохастического разностного уравнения $y_t=2+0.5y_{t-1}+\e_t$ есть стационарное решение».
\end{sol}
\end{problem}




\begin{problem}
Верно ли, что при удалении из стационарного ряда каждого второго наблюдения получается стационарный ряд?


\begin{sol}
да, стационарный
\end{sol}
\end{problem}



\begin{problem}
У эконометрессы Ефросиньи был стационарный ряд. Ей было скучно и она подбрасывала неправильную монетку, выпадающую орлом с вероятностью $0.7$. Если выпадал орёл, она оставляла очередной $y_t$, если решка — то зачёркивала. Получается ли у Ефросиньи стационарный ряд?


\begin{sol}
да, получается
\end{sol}
\end{problem}


\begin{problem}
Имеется временной ряд, $\e_1$, $\e_2$, \ldots, $\e_{101}$. Величины $\e_t$ нормально распределены, $N(0,\sigma^2)$, и независимы. Построим график этого процесса.
\begin{enumerate}
\item Является ли этот процесс белым шумом?
\item Сколько в среднем раз график пересекает ось абсцисс?
\item Оцените вероятность того, что график пересечет ось абсцисс более 60 раз.
\end{enumerate}



\begin{sol}
да, это белый шум. Величина $N$ распределена биномиально, $Bin(n=100,p=1/2)$, $\E(N)=50$.
\end{sol}
\end{problem}


\begin{problem}
Величины $x_t$ независимы и равновероятно принимают значения $0$ и $1$. Величины $y_t$ независимы и нормальны $\cN(0;24)$.
Процессы $(x_t)$ и $(y_t)$ независимы. Для каждого из пунктов ответьте на три вопроса. Верно ли, что величины $z_t$ одинаково распределены? Верно ли, что они независимы? Верно ли, что процесс $(z_t)$ — белый шум? 
\begin{enumerate}
  \item $z_t = x_t (1-x_{t-1})y_t$;  
\item $z_t = y_{t-1}y_t$;
\end{enumerate}
\begin{sol}

\begin{enumerate}
  \item $z_t = x_t (1-x_{t-1})y_t$;  
    Процесс $z_t$ — белый шум, $\E(z_t)=0$, $\Var(z_t)=6$. Величины $z_t$ зависимы. Например, если $z_t \neq 0$, то $z_{t+1}=z_{t-1}=0$. Величины $z_t$ одинаково распределены.
\item $z_t = y_{t-1}y_t$; 
Процесс $z_t$ — белый шум. Величины $z_t$ зависимы. Величины $z_t$ одинаково распределены.
\end{enumerate}


\end{sol}
\end{problem}




\begin{problem}
Величина $Z$ равновероятно принимает значения $0$ и $1$. Условное распределение вектора $X=(X_1, X_2)$ при известном $Z$ известно:
\[
  \begin{pmatrix}
    X_1 \\
    X_2 \\
  \end{pmatrix}|Z=0 \sim \cN\left( 
    \begin{pmatrix}
      0 \\
      0 \\
    \end{pmatrix};
    \begin{pmatrix}
      1 & 0 \\
      0 & 1 \\
    \end{pmatrix}
    \right)
\]

\[
  \begin{pmatrix}
    X_1 \\
    X_2 \\
  \end{pmatrix}|Z=1 \sim \cN\left( 
    \begin{pmatrix}
      1 \\
      1 \\
    \end{pmatrix};
    \begin{pmatrix}
      4 & -1 \\
      -1 & 9 \\
    \end{pmatrix}
    \right)
\]

Найдите 
\begin{enumerate}
  \item Частную корреляцию $\pCorr(X_1, X_2; Z)$;
  \item Условную корреляцию $\Corr(X_1, X_2 | Z)$;
\end{enumerate}

  \begin{sol}
Проекции: $\tilde X_1 = X_1 + Z$; $\tilde X_2 = X_2 + Z$; $\E(X_i|Z)=1-Z$; $\Cov(X_i, Z)=-1/4$;

Величина $Z$ имеет распределение Бернулли, поэтому $\E(Z)=1/2$ и $\Var(Z)=1/4$;
    
\[
  \pCorr(X_1, X_2; Z) = \frac{-1/2}{12.5} = -\frac{1}{\sqrt{50}}
\]
\[
  \Corr(X_1, X_2|Z)=-Z/6
\]
  \end{sol}
\end{problem}









\chapter{ARMA}

\begin{problem}
Рассмотрим модель $y_t=\mu + \e_t$, где $\e_t$ — стационарный AR(1) процесс $\e_t=\rho \e_{t-1} + u_t$ с $u_t \sim N(0,\sigma^2)$. Найдите условную логарифмическую функцию правдоподобия $l(\mu, \rho, \sigma^2 | y_1)$.
\begin{sol}

\end{sol}
\end{problem}

\begin{problem}
Известно, что $\e_t$ — белый шум. Классифицируйте в рамках классификации ARIMA процесс $y_t=1+\e_t + 0.5\e_{t-1} + 0.4\e_{t-2} + 0.3\e_{t-3} + 0.2y_{t-1} + 0.1y_{t-2}$.
\begin{sol}

ARMA(2,3), ARIMA(2,0,3)
\end{sol}
\end{problem}


\begin{problem}
На графике представлены данные по уровню озера Гур\'{о}н в футах в 1875-1972 годах:

\begin{minted}[mathescape,
               linenos,
               numbersep=5pt,
               frame=lines,
               framesep=2mm]{r}
level <- LakeHuron
df <- data.frame(level, obs = 1875:1972)
n <- nrow(df) # used later for answers
v.acf <- acf(level, plot = FALSE)$acf
v.pacf <- pacf(level, plot = FALSE)$acf
acfs.df <- data.frame(lag = c(1:15, 1:15),
    acf = c(v.acf[2:16], v.pacf[1:15]),
    acf.type = rep(c("ACF", "PACF"), each = 15))
model <- arima(level, order = c(1, 0, 1))
resids <- model$residuals
resid.acf <- acf(resids, plot = FALSE)$acf
\end{minted}



\begin{minted}[mathescape,
               linenos,
               numbersep=5pt,
               frame=lines,
               framesep=2mm]{r}
tikz("../R_plots/huron_ts.tikz", standAlone = FALSE, bareBones = TRUE)
ggplot(df, aes(x = obs, y = level)) + geom_line() +
    labs(x = "Год", y = "Уровень озера (футы)")
dev.off()
\end{minted}


%\begin{minipage}{\textwidth}
%\begin{tikzpicture}[scale = 0.025]
%\input{R_plots/huron_ts.tikz}
%\end{tikzpicture}
%\end{minipage}




График автокорреляционной и частной автокорреляционной функций:

\begin{minted}[mathescape,
               linenos,
               numbersep=5pt,
               frame=lines,
               framesep=2mm]{r}
ggplot(acfs.df, aes(x = lag, y = acf, fill = acf.type))+
    geom_histogram(position = "dodge", stat = "identity")+
  xlab("Лаг") + ylab("Корреляция") +
  guides(fill = guide_legend(title = NULL))+
  geom_hline(yintercept = 1.96 / sqrt(nrow(df)))+
  geom_hline(yintercept = -1.96 / sqrt(nrow(df)))
\end{minted}


\begin{enumerate}
\item Судя по графикам, какие модели класса ARMA или ARIMA имеет смысл оценить?
\item По результатам оценки некоей модели ARMA c двумя параметрами, исследователь посчитал оценки автокорреляционной функции для остатков модели. Известно, что для остатков модели первые три выборочные автокорреляции равны соответственно $0.00467$, $-0.0129$ и $-0.063$. С помощью подходящей статистики проверьте гипотезу о том, что первые три корреляции ошибок модели равны нулю.
\end{enumerate}


\begin{sol}
\begin{enumerate}
\item Процесс $AR(2)$, т.к. две первые частные корреляции значимо отличаются от нуля, а гипотезы о том, что каждая последующая равна нулю не отвергаются.
\item Можно использовать одну из двух статистик
\[
\text{Ljung-Box}=n(n+2)\sum_{k=1}^3\frac{\hat{\rho}_k^2}{n-k}=
0.42886
\]
\[
\text{Box-Pierce}=n\sum_{k=1}^3\hat{\rho}_k^2=
0.4076
\]
Критическое значение хи-квадрат распределения с 3-мя степенями свободы для $\alpha=0.05$ равно $\chi^2_{3,crit}=7.81$.
Вывод: гипотеза $H_0$ об отсутствии корреляции ошибок модели не отвергается.
\end{enumerate}
\end{sol}
\end{problem}




\begin{problem}
Процесс $x_t$ — это процесс $y_t$, наблюдаемый с ошибкой, т.е. $x_t=y_t+\nu_t$. Ошибки $\nu_t$ являются белым шумом и не коррелированы с $y_t$.
\begin{enumerate}
\item Является ли процесс $x_t$ MA(1) процессом, если $y_t$ —  MA(1) процесс? Если да, то как связаны их автокорреляционные функциии?
\item Является ли процесс $x_t$ стационарным AR(1) процессом, если $y_t$ —  стационарный AR(1) процесс? Если да, то как связаны их автокорреляционные функциии?
\end{enumerate}


\begin{sol}

\end{sol}
\end{problem}


\begin{problem}
Рассмотрим стационарный AR(1) процесс $y_t=\rho y_{t-1} + \e_t$, где $\e_t \sim N(0,1)$. Имеется ряд $y_1$, $y_2$, \ldots, $y_{101}$. Построен график этого процесса. Как от $\rho$ зависит математическое ожидание количества пересечений графика с осью абсцисс?


\begin{sol}
Среднее количество пересечений равно 50 помножить на вероятность того, что два соседних $y_t$ разного знака. Найдём вдвое меньшую вероятность, $\P(y_1>0, y_2 <0)$.
\end{sol}
\end{problem}



\begin{problem}
Рассмотрим процессы:

\begin{enumerate}
\item[A] Процесс скользящего среднего:
\[
y_t=\e_t+2\e_{t-1}+3
\]

\item[B]
\[
a_t=\e_t+\e_1 + 3
\]

\item[C]
\[
b_t=t\e_t + 3
\]

\item[D]
\[
c_t=\cos\left(\frac{\pi t}{2}\right)\e_1 + \sin\left(\frac{\pi t}{2}\right)\e_2 + 2
\]

\item[E] Процесс случайного блуждания со смещением:
\[
\begin{cases}
z_t=\e_t+z_{t-1}+3 \\
z_0=0
\end{cases}
\]

\item[F] Процесс с трендом:
\[
w_t=2+3t+\e_t
\]

\item[G] Еще один процесс:
\[
r_t=\begin{cases}
1, \; \text{при четных t} \\
-1, \; \text{при нечетных t}
\end{cases}
\]

\item[H] Приращение случайного блуждания
\[
s_t=\Delta z_t
\]

\item[I] Приращение процесса с трендом
\[
d_t=\Delta w_t
\]
\end{enumerate}

Для каждого процесса:

\begin{enumerate}
\item Найдите $\E(y_t)$, $\Var(y_t)$
\item Найдите $\gamma_k = \Cov(y_t, y_{t-k})$
\item Найдите $\rho_k = \Corr(y_t,y_{t-k})$. Если ни одна корреляция $\rho_k$ не зависит от времени $t$, то постройте график зависимости $\rho_k$ от $k$.
\item Является ли процесс стационарным?
\item Сгенерируйте одну реализацию процесса. Постройте её график и график оценки автокорреляционной функции.
\end{enumerate}


\begin{sol}

\[
\E(b_t) = 3
\]

\[
\Var(b_t) = t^2 \sigma^2_{\e}
\]

\[
\Cov(b_t, b_{t-k}) = 0, k \geq 1
\]

\[
\Corr(b_t, b_{t-k}) = 0, k \geq 1
\]

$b_t$ — нестационарный из-за дисперсии


\[
\E(c_t) = 2
\]

\[
\Var(c_t) = \sigma^2_{\e}
\]

\[
\Cov(c_t, c_{t-k}) = \cos( \pi k /2)\sigma^2_{\e}, k \geq 1
\]

\[
\Corr(c_t, c_{t-k}) = \cos( \pi k /2), k \geq 1
\]

$c_t$ — стационарный
\end{sol}
\end{problem}





\begin{problem}
Эконометресса Антуанетта построила график автоковариационной функции временного ряда и распечатала его:

здесь график

Потом она с ужасом обнаружила, что до презентации исследования остается совсем мало времени, а распечатать надо было график автокорреляционной функции. Что надо исправить Антуанетте на графике, чтобы успеть еще сделать причёску и макияж (это очень важно для презентации)?



\begin{sol}
зачеркнуть одну цифру
\end{sol}
\end{problem}


\begin{problem}
Рассмотрите стационарные процессы
\begin{enumerate}
\item[A.] AR(1): $y_t = 5 + 0.3y_{t-1} + \e_t$
\item[B.] AR(2): $y_t = 5 + 0.3y_{t-1} + 0.1 y_{t-2} + \e_t$
\item[C.] MA(1): $y_t = 5 + 0.3\e_{t-1} + \e_t$
\item[D.] MA(2): $y_t = 5 + 0.3\e_{t-1} + 0.9\e_{t-2} + \e_t$
\item[E.] ARMA(1, 1): $y_t = 5 + 0.3y_{t-1} + 0.4\e_{t-1} + \e_t$
\end{enumerate}

Если возможно, то представьте каждый процесс в виде:
\begin{enumerate}
\item $MA(\infty)$.
\item $AR(\infty)$.
\item $y_t = c + \gamma_1 y_{t-1} + u_t$, где $u_t$ некоррелирован с $y_{t-1}$. Будет ли $u_t$ белым шумом?
\item $y_t = c + \gamma_1 y_{t+1} + u_t$, где $u_t$ некоррелирован с $y_{t+1}$. Будет ли $u_t$ белым шумом?
\item $y_t = c + \gamma_1 y_{t-1} + \gamma_2 y_{t-2} + u_t$, где $u_t$ некоррелирован с $y_{t-1}$ и $y_{t-2}$. Будет ли $u_t$ белым шумом?
\item $y_t = c + \gamma_1 y_{t+1} + \gamma_2 y_{t+2} + u_t$, где $u_t$ некоррелирован с $y_{t+1}$ и $y_{t+2}$. Будет ли $u_t$ белым шумом?
\end{enumerate}
\begin{sol}
\end{sol}
\end{problem}


\begin{problem}
Рассмотрите стационарные процессы
\begin{enumerate}
\item[A.] AR(1): $y_t = 5 + 0.3y_{t-1} + \e_t$
\item[B.] AR(2): $y_t = 5 + 0.3y_{t-1} + 0.1 y_{t-2} + \e_t$
\item[C.] MA(1): $y_t = 5 + 0.3\e_{t-1} + \e_t$
\item[D.] MA(2): $y_t = 5 + 0.3\e_{t-1} + 0.9\e_{t-2} + \e_t$
\item[E.] ARMA(1, 1): $y_t = 5 + 0.3y_{t-1} + 0.4\e_{t-1} + \e_t$
\end{enumerate}

Для каждого из процессов:
\begin{enumerate}
\item Найдите математическое ожидание $\E(y_t)$.
\item Найдите первые три значения автокорреляционной функции $\rho_1$, $\rho_2$, $\rho_3$.
\item Найдите первые три значения частной автокорреляционной функции $\phi_{11}$, $\phi_{22}$, $\phi_{33}$.
\end{enumerate}
\begin{sol}
\end{sol}
\end{problem}

\begin{problem}
  Известна автокорреляционная функция процесса $(y_t)$: $\rho_1 = 0.7$, $\rho_2 = 0.3$, и $\rho_k = 0$ при $k\geq 3$. Кроме того, $\E(y_t)=4$. Выпишите возможные уравнения процесса.
\begin{sol}
\end{sol}
\end{problem}

\begin{problem}
  Известна частная автокорреляционная функция процесса $(y_t)$: $\phi_{11} = 0.7$, $\phi_{22} = 0.3$, и $\phi_{kk} = 0$ при $k\geq 3$. Кроме того, $\E(y_t)=4$. Выпишите возможные уравнения процесса.
\begin{sol}
\end{sol}
\end{problem}

\begin{problem}
Если возможно, то найдите процесс с данной автокорреляционной или частной автокорреляционной функцией.

\begin{enumerate}
  \item $ACF = (0.9, -0.9, 0, 0, 0, \ldots)$;
  \item $PACF = (0.9, -0.9, 0, 0, 0, \ldots)$;
  \item $PACF = (0.9, 0, 0, 0, 0, \ldots)$;
  \item $PACF = (0, 0.9, 0, 0, 0, 0, \ldots)$;
  \item $ACF = (0.9, 0, 0, 0, 0, \ldots)$;
  \item $ACF = (0, 0.9, 0, 0, 0, 0, \ldots)$;
\end{enumerate}

\begin{sol}
  \begin{enumerate}
    \item $ACF = (0.9, -0.9, 0, 0, 0, \ldots)$ не бывает, так как определитель корреляционной матрицы 3 на 3 отрицательный;
    \item $PACF = (0.9, -0.9, 0, 0, 0, \ldots)$ — AR(2);
    \item $PACF = (0.9, 0, 0, 0, 0, \ldots)$ — $y_t = 0.9y_{t-1} + u_t$;
    \item $PACF = (0, 0.9, 0, 0, 0, 0, \ldots)$ — $y_t = 0.9y_{t-2} + u_t$;
    \item $ACF = (0.9, 0, 0, 0, 0, \ldots)$ — не бывает, подозрение падает на MA(1), но решения только с комплексными коэффициентами, геометрически: два угла с косинусом 0.9, то есть примерно по 30 градусов, и они даже в сумме не могут дать перпендикуляр;
    \item $ACF = (0, 0.9, 0, 0, 0, 0, \ldots)$ — не бывает, если проредить процесс через один, то должна получится невозможная ACF;
  \end{enumerate}
   В целом PACF может быть любая,
   \url{http://projecteuclid.org/euclid.aos/1176342881}.
\end{sol}
\end{problem}


\begin{problem}
Рассмотрим стационарный процесс $y_t = 4 + 0.7y_{t-1} - 0.12y_{t-2} + \e_t$, где $\e_t$ — белый шум, причём $\Cov(\e_t, y_{t-k})=0$ при $k \geq 1$.

\begin{enumerate}
  \item Найдите автокорреляционную функцию: $\rho_1$, $\rho_2$ и общую формулу для $\rho_k$.
  \item Найдите $\lim_{k \to \infty} \rho_k$.
  \item Найдите частную автокорреляционную функцию: $\phi_{11}$, $\phi_{22}$, \ldots.
\end{enumerate}
\begin{sol}
  $\phi_{kk}=0$ при $k \geq 3$.
\end{sol}
\end{problem}


\begin{problem}
Рассмотрим стационарный процесс с уравнением
\[
y_t = 10 + 0.69 y_{t-1} + \e_t - 0.71 \e_{t-1}.
\]

Выпишите гораздо более простой процесс со свойствами близкими к свойствам данного процесса.
\begin{sol}
Заметим, что $0.69\approx 0.71$, сокращаем множитель $1-0.7L$, получаем $y_t = 100/3 + \e_t$.
\end{sol}
\end{problem}


\begin{problem}
Процесс $\e_t$ — белый шум. Рассмотрим уравнение
\[
y_t = 0.5 y_{t-1} + \e_t.
\]

Какие из указанных процессов $(y_t)$ являются его решением? Стационарным решением?
\begin{enumerate}
  \item $y_t = 0.5^t$;
  \item $y_t = \sum_{i=0}^{\infty} 0.5^i \e_{t-i}$;
  \item $y_t = 0.5^t + \sum_{i=0}^{\infty} 0.5^i \e_{t-i}$;
  \item $y_t = 0.5^t\e_{100} + \sum_{i=0}^{\infty} 0.5^i \e_{t-i}$;
  \item $y_t = 0.5^t + \sum_{i=0}^{t} 0.5^i \e_{t-i}$;
  \item $y_t = \sum_{i=0}^{t} 0.5^i \e_{t-i}$;
\end{enumerate}


\begin{sol}
Стационарным решением является $y_t = \sum_{i=0}^{\infty} 0.5^i \e_{t-i}$. Решениями также являются: $y_t = 0.5^t + \sum_{i=0}^{\infty} 0.5^i \e_{t-i}$, $y_t = 0.5^t\e_{100} + \sum_{i=0}^{\infty} 0.5^i \e_{t-i}$, $y_t = 0.5^t + \sum_{i=0}^{t} 0.5^i \e_{t-i}$, $y_t = \sum_{i=0}^{t} 0.5^i \e_{t-i}$.
\end{sol}
\end{problem}



\begin{problem}

Рассмотрим стационарный процесс $y_t$, задаваемый уравнением
\[
y_t = 2 + 0.6 \cdot y_{t-1} - 0.08 y_{t-2} + \e_t,
\]
где $\e_t \sim \cN(0; 4)$.

\begin{enumerate}
\item  Найдите $\E_t(y_{t+1})$, $\Var_t(y_{t+1})$
\item Найдите $\E_t(y_{t+2})$, $\Var_t(y_{t+2})$
\item Постройте 95\%-ый предиктивный интервал для $y_{102}$, если $y_{99}=5$, $y_{100}=5.1$
\item Найдите $\E(y_t)$, $\Var(y_t)$
\item Найдите $\lim_{h\to\infty}\E_t(y_{t+h})$, $\lim_{h\to\infty}\Var_t(y_{t+h})$
\end{enumerate}


\begin{sol}

$\E_t(y_{t+1})=2+0.6y_{t-1}-0.08y_{t-2}$, $\Var_t(y_{t+1})=4$

$\E_t(y_{t+2})=3.2 + 0.28 y_t- 0.048y_{t-1}$, $\Var_t(y_{t+2})=1.36 \cdot 4$

$\E_{100}(y_{102})= 4.388$, $\Var_{100}(y_{102})=5.44$.

Предиктивный интервал $[4.388 - 1.96 \sqrt{5.44};4.388 + 1.96 \sqrt{5.44}]$

$\E(y_t)=\frac{2}{0.48}\approx 4.17$

\end{sol}
\end{problem}



\begin{problem}
Задан процесс $y_t = 7 + u_t + 0.2 u_{t-1}$, где $u_t$ независимы и нормальны $u_t \sim \cN(0;4)$. Известно, что $y_{100}=7.2$, $u_{100}=1.3$, $y_{100}+(-0.2)y_{99}+(-0.2)^2y_{98}+\ldots+(-0.2)^{99}y_1=5.6$.

Пусть $\cF_t=\sigma(y_t, y_{t-1}, \ldots, y_1, u_t, u_{t-1}, \ldots, u_1)$ и $\cH_t = \sigma(y_t, y_{t-1}, \ldots, y_1)$.
\begin{enumerate}
  \item Найдите $\E(y_{101}|\cF_{100})$, $\Var(y_{101}|\cF_{100})$.
  \item С помощью $AR(\infty)$ представления примерно найдите $\E(y_{101}|\cH_{100})$, $\Var(y_{101}|\cH_{100})$. Постройте 95\%-ый предиктивный интервал для $y_{101}$.
  \item Найдите $\E(y_{101}|y_{100})$, $\Var(y_{101}|y_{100})$.
  \item Найдите $\E(y_{101}|y_{100}, y_{99})$, $\Var(y_{101}|y_{100}, y_{99})$.
\end{enumerate}

\begin{sol}
Заметим, что $\Var(u_t|\cF_t)=0$. Более того, для обратимого процесса $\Var(u_t|y_t, y_{t-1}, \ldots, y_1) \approx \Var(u_t|y_t, y_{t-1}, \ldots) = 0$.
\[
\E(y_{101}|y_{100})=7 + 0 + 0.2\E(u_{100}|y_{100})
\]
\[
\E(u_{100}|y_{100}) = \beta_1 + \beta_2 y_{100}
\]
\[
\beta_2 = \frac{\Cov(y_{100}, u_{100})}{\Var(y_{100})}=4/4.16, \beta_1 = \E(u_{100}) - \beta_2 \E(y_{100})=-4\cdot 7/4.16
\]
\[
\frac{y_t}{1+0.2L} = \frac{7}{1+0.2L} + u_t
\]
Заметим, что $\frac{7}{1+0.2L}=7/1.2$, так как $L\cdot 7 = 7$ (вчера семь равнялось семи).

По условию $\frac{y_{100}}{1+0.2L} \approx 5.6$. Знак «примерно равно» возникает из-за замены бесконечной суммы на конечную.

\end{sol}
\end{problem}



\begin{problem}
  У исследовательницы Аграфены три наблюдения, $y_1 = 0.1$, $y_2 = -0.2$, $y_3 = 0.2$. Аграфена предполагает, что данные подчиняются стационарному AR(1) процессу $y_t = \beta y_{t-1} + u_t$, где $u_t \sim \cN(0;\sigma^2_u)$. 

  \begin{enumerate}
    \item Найдите $\E(y_1)$, $\E(y_2|y_1)$, $\E(y_3|y_2)$;
    \item Найдите $\Var(y_1)$, $\Var(y_2|y_1)$, $\Var(y_3|y_2)$;
    \item Найдите функции плотности $f(y_1)$, $f(y_2|y_1)$, $f(y_3|y_2)$;
    \item Выпишете полную логарифмическую функцию правдоподобия $\ell(y|\beta, \sigma^2_u)$. 
    \item Если возможно, явно решите задачу максимизации полного правдоподобия.
    \item Выпишите условную логарифмическую функцию правдоподобия $\ell(y_2, y_3|\beta, \sigma^2_u, y_1)$. 
    \item Если возможно, явно решите задачу максимизации условного правдоподобия при фиксированном $y_1$.
  \end{enumerate}
  \begin{sol}
    $\E(y_1)=0$, $\Var(y_1)=\sigma^2_u/(1-\beta^2)$, $\E(y_t|y_{t-1})=\beta y_{t-1}$, $\Var(y_t|y_{t-1})=\sigma^2_u$.

    При максимизации условного правдоподобия получаем:
    \[
         \hb = \frac{y_1 y_2 + y_2 y_3}{y_1^2 + y_2^2}
    \]
  \end{sol}
\end{problem}

\begin{problem}
  Белые шумы $u_t$ и $v_t$ независимы, $\Var(u_t) = 1$, $\Var(v_t)=1$. Рассмотрим процесс $y_t = 5u_{t-1} - 4 v_{t-1} + u_t + v_t$.

  \begin{enumerate}
    \item Выпишите классическое представление процесса $y_t$ как ARMA-процесса.
    \item Выразите белый шум из полученного классического представления $y_t$ через белые шумы $(u_t)$ и $(v_t)$.
  \end{enumerate}
  \todo[inline]{можно подобрать цифры, чтобы коэффициент был хороший :)}
  \begin{sol}
  \end{sol}
\end{problem}









\chapter{ETS}


\begin{problem}
  Рассмотрим ETS-ANN модель с $\alpha = 1/2$, $y_1=6$, $y_2=9$, $y_3 = 6$, $\sigma^2=9$. 


  \begin{enumerate}
    \item Найдите величину $l_0$, которая минимизирует $RSS$;
    \item Постройте точечный прогноз $\hat y_{4|2}$, $\hat y_{5|2}$;
     \item Постройте 95\%-ый предиктивный интервал для $y_{4}$ и $y_{5}$.
  \end{enumerate}
\begin{sol}
\end{sol}
\end{problem}

\begin{problem}
  Рассмотрим ETS-AAN модель с $\alpha = 1/2$, $\beta=3/4$, $l_{0}=7$, $b_0=2$, $y_1=6$, $y_2=9$, $y_3=3$, $\sigma^2=9$.  
  \begin{enumerate}
    \item Постройте точечный прогноз $\hat y_{4|3}$, $\hat y_{5|3}$;
    \item Постройте 95\%-ый предиктивный интервал для $y_{4}$ и $y_{5}$.
  \end{enumerate}
\begin{sol}
\end{sol}
\end{problem}

\begin{problem}
  Рассмотрим ETS-AAN модель с $\alpha = 1/2$, $\beta=3/4$, $l_{0}=7$, $y_1=6$, $y_2=9$, $\sigma^2=16$.    


  \begin{enumerate}
    \item Найдите величину $b_0$, которая минимизирует $RSS$;
    \item Постройте точечный прогноз $\hat y_{3|2}$, $\hat y_{4|2}$;
     \item Постройте 95\%-ый предиктивный интервал для $y_{3}$ и $y_{4}$.
  \end{enumerate}
\begin{sol}
\end{sol}
\end{problem}

\begin{problem}
  Рассмотрим ETS-AAN модель с $\alpha = 1/2$, $\beta=3/4$, $l_{0}=7$, $y_1=6$, $y_2=9$, $y_3=3$.  

  Выпишите сумму квадратов ошибок прогнозов на один шаг вперёд через $b_0$.
  % минимизация тут длинная
\begin{sol}
\end{sol}
\end{problem}



\begin{problem}
  Рассмотрим ETS-AAN модель с $\alpha = 1/2$, $\beta=3/4$, $l_{99}=8$, $b_{99}=1$, $y_{99}=10$, $y_{100}=8$, $\sigma^2=16$.
  \begin{enumerate}
    \item Найдите $l_{100}$, $b_{100}$, $l_{98}$, $b_{98}$;
    \item Постройте точечный прогноз $\hat y_{101|100}$, $\hat y_{102|100}$;
    \item Постройте 95\%-ый предиктивный интервал для $y_{101}$ и $y_{102}$.
  \end{enumerate}
\begin{sol}
\end{sol}
\end{problem}


\begin{problem}
Для каждой из ETS моделей найдите эквивалентную модель класса ARIMA:
	\begin{enumerate}
		\item Простое экспоненциальное сглаживание, ETS-ANN;
		\item Аддитивное сглаживание Хольта, ETS-AAN;
		\item Аддитивное сглаживание Хольта с угасающим трендом, ETS-AAdN;
		\item Аддитивное сглаживание Хольта-Винтерса для месячных данных, ETS-AAA;
		\item Аддитивное сглаживание Хольта-Винтерса с угасающим трендом для месячных данных, ETS-AAdA;
		\item ETS-ANA;
	\end{enumerate}
\begin{sol}
	\begin{enumerate}
		\item Простое экспоненциальное сглаживание, ETS-ANN; ARIMA(0,1,1)
		\item Аддитивное сглаживание Хольта, ETS-AAN; ARIMA(0,2,2)
		\item Аддитивное сглаживание Хольта с угасающим трендом, ETS-AAdN; ARIMA(1,1,2)
		\item Аддитивное сглаживание Хольта-Винтерса для месячных данных, ETS-AAA; ARIMA(0,1,13)-SARIMA(0,1,0)
		\item Аддитивное сглаживание Хольта-Винтерса с угасающим трендом для месячных данных, ETS-AAdA; ARIMA(0,1,13)-SARIMA(0,1,0)
		\item ETS-ANA; ARIMA(0,1,12)-SARIMA(0,1,0)
	\end{enumerate}
\end{sol}
\end{problem}


\begin{problem}
  Рассмотрим ETS-AAN модель. По каким параметрам модели оптимальные точки можно получить в явном виде?


\begin{sol}
По $l_0$, $b_0$;
\end{sol}
\end{problem}





\chapter{TBATS}


\begin{problem}
  Найдите предел 
  \[
    \lim_{w \to 0} \frac{y^w - 1}{w} 
  \]
\begin{sol}
  $\ln y$
\end{sol}
\end{problem}








\chapter{Вступайте в ряды Фурье!}


Суть преобразования Фурье. Вместо исходного временного ряда $x_0$, $x_1$, \ldots, $x_{N-1}$ мы получаем ряд комплексных чисел $X_0$, $X_1$, \ldots, $X_{N-1}$. 
Эти комплексные числа $X_k$ показывают, насколько сильно проявляется каждая частота в исходном ряду.

Чтобы получить одно комплексное число $X_k$:

\begin{enumerate}
  \item Разрежем круг на $N$ равных частей. Каждая часть образует угол $2\pi/N$.
  \item Разместим исходные числа $x_0$, $x_1$, \ldots, $x_{N-1}$ на разрезах по часовой стрелке с шагом $k$. 
    При этом число $x_0$ приходится на угол $0$; число $x_1$ — на угол $2\pi/N \cdot k$; 
число $x_2$ — на угол $2\pi/N \cdot 2k$, и так далее. 
  \item Трактуем $x_i$ как силу ветра в направлении разреза.
  \item $X_k$ — усреднённая сила ветра.
\end{enumerate}


Прямое преобразование Фурье задаётся формулой\footnote{Иногда множитель $1/N$ относят к обратному преобразованию Фурье, иногда поровну разносят как $1/\sqrt{N}$.}:
\[
  X_k = \frac{1}{N} \sum_{n=0}^{N-1} x_n w^{kn}, 
\]
где комплексное число $w$ кодирует поворот на $1/N$ часть круга по часовой стрелке, $w = \exp\left(\frac{-2i\pi}{N} \right)$.



Обратное преобразование Фурье
\[
  x_n = \sum_{k=0}^{N-1} X_k (w^{*})^{nk}, 
\]
где комплексное число $w^{*}$ является сопряжённым к числу $w$.


\begin{problem}
  Немножко теории:
  \begin{enumerate}
    \item Посмотрите видео от 3blue1brown, \url{https://www.youtube.com/watch?v=cV7L95IkVdE}.
    \item Прочтите про дискретное преобразование Фурье на brilliant, \url{https://brilliant.org/wiki/discrete-fourier-transform/}.
  \end{enumerate}
\begin{sol}
\end{sol}
\end{problem}



\begin{problem}
  Про Фурье :)
  \begin{enumerate}
    \item Зачем Фурье собирал огарки свечей в бенедиктинской артиллерийской школе?
    \item Первый раз Фурье был арестован за недостаточную поддержку якобинцев. За что Фурье был арестован во второй раз?
    \item Что было в руке у Фурье во время переговоров о перемирии после потери французами Каира?
  \end{enumerate}
  \begin{sol}
    \begin{enumerate}
      \item Чтобы заниматься математикой по ночам.
      \item За поддержку якобинцев.
      \item Кофейник. Был разбит пулей.
    \end{enumerate}
\end{sol}
\end{problem}


\begin{problem}
  Вспомним комплексные числа :)
  \begin{enumerate}
    \item Найдите сумму $7 + 7 \exp(2i\pi/3) + 7 \exp(4i\pi/3)$;
    \item Найдите сумму $6 + 4\exp(i\pi)$;
  \end{enumerate}

  \begin{sol}
  \end{sol}
\end{problem}


\begin{problem}
Найдите прямое преобразование Фурье последовательностей
\begin{enumerate}
  \item $1$, $4$, $1$, $4$, $1$, $4$;
  \item $1$, $9$;
  \item $8$;
  \item $1$, $0$, $0$, $0$;
\end{enumerate}
  \begin{sol}
  \end{sol}
\end{problem}

\begin{problem}
  Прямое преобразование Фурье можно записать в матричном виде $X = \frac{1}{N}Fx$. 
  \begin{enumerate}
    \item Как устроена матрица $F$?
    \item Найдите $F\cdot F^{*}$, где $F^{*}$ — транспонированная и сопряжённая матрица к $F$;
    \item Как устроена матрица $F^{-1}$?
    \item Как записывается обратное преобразование Фурье в матричном виде?
  \end{enumerate}

  \begin{sol}
  \end{sol}
\end{problem}


\begin{problem}

Обратное преобразование Фурье задаётся формулой
\[
  x_n = \sum_{k=0}^{N-1} X_k (w^{*})^{nk}, 
\]
где комплексное число $w^{*}$ является сопряжённым к числу $w = \exp\left(\frac{-2i\pi}{N} \right)$.


  Докажите, что обратное преобразование Фурье, действительно, от комплексных чисел $(X_k)$ переходит к исходныму ряду $(x_n)$.
  \begin{sol}
  \end{sol}
\end{problem}


\begin{problem}
  В типичной задаче исходный ряд $x_0$, $x_1$, \ldots, $x_{N-1}$ является действительными числами. 
  Докажите, что при дискретном преобразовании Фурье числа $X_k$ и $X_{N-k}$ являются комплексно-сопряжёнными.

  \begin{sol}
  \end{sol}
\end{problem}


\begin{problem}
  Рассмотрим ряд месячной периодичности. Число наблюдений делится на 12. Исследователь Василий рассматривает в качестве регрессоров следующие переменные: столбец из единиц,
  $\sin\left(\frac{2\pi}{12} t\right)$, 
 $\cos\left(\frac{2\pi}{12} t\right)$, 
 $\sin\left(\frac{2\pi}{12} 2t\right)$, 
 $\cos\left(\frac{2\pi}{12} 2t\right)$, 
 $\sin\left(\frac{2\pi}{12} 3t\right)$, 
 $\cos\left(\frac{2\pi}{12} 3t\right)$, 
 $\sin\left(\frac{2\pi}{12} 4t\right)$, 
 $\cos\left(\frac{2\pi}{12} 4t\right)$, 
 $\sin\left(\frac{2\pi}{12} 5t\right)$, 
 $\cos\left(\frac{2\pi}{12} 5t\right)$, 
 $\cos\left(\frac{2\pi}{12} 6t\right)$.

 \begin{enumerate}
   \item Являются ли эти регрессоры ортогональными?
   \item Василий рассматривает два варианта действий. 
     Вариант А: построить 12 регрессий исходного ряда на каждый регрессор в отдельности. Вариант Б: построить одну регрессию.
     Будут ли отличаться коэффициенты при регрессорах?
    \item Можно ли добавить в качестве регрессора $\sin\left(\frac{2\pi}{12} 6t\right)$ или  $\cos\left(\frac{2\pi}{12} 7t\right)$?
  \end{enumerate}
 \begin{sol}
   Да, ряды являются ортогональными. Можно строить регрессии на эти регрессоры в любых комбинациях, оценки бет выходят одни и те же.
   Другие ряды добавить нельзя — будет строгая мультиколлинеарность.
 \end{sol}
 \end{problem}

 \begin{problem}
   Исследовательница Агриппина взяла ряд длиной 6 наблюдений и построила его регрессию на тригонометрические ряды Фурье:
   \[
     \hat x_t = 3.5 - 1.73 \sin(2\pi t/6) + 1.00 \cos(2\pi t/6) - 0.58\sin(4\pi t/6) + 1.00 \cos(4\pi t/6) +0.30 \cos(6\pi t/6)
   \]

   Найдите прямое преобразование Фурье исходного ряда.
   \begin{sol}
     На всякий случай, это был ряд $1$, $2$, $3$, $4$, $5$, $6$.
   \end{sol}
 \end{problem}


 \begin{problem}
   Исследовательница Агриппина взяла ряд длиной 6 наблюдений и нашла его преобразование Фурье: 
   \[
     1.5, \; -\frac{1}{6}+\frac{1}{\sqrt{12}}i, \; 0, \; -\frac{1}{6}, \; 0, -\frac{1}{6} - \frac{1}{\sqrt{12}}i.
   \]
   \begin{enumerate}
     \item Найдите регрессию этого ряда на тригонометрические ряды Фурье;
     \item Восстановите исходный ряд;
   \end{enumerate}

   \begin{sol}
   $1$, $1$, $1$, $2$, $2$, $2$
   \end{sol}
 \end{problem}


\Closesolutionfile{solution_file}


% !TEX root = ../ts_pset_main.tex

\chapter{ARMA}

Многие источники неверно рассказывают критерий стационарности ARIMA процесса. 
Проверено, мин нет: \cite{van2010time}, \cite{tsay2005analysis}.


\begin{problem}
Рассмотрим три разностных уравнения:
\begin{align*}
  (A) y_t = 1 + 0.5 y_{t-1} \\
  (B) y_t = 1 + y_{t-1} \\
  (C) y_t = 1 + 2 y_{t-1} \\
\end{align*}
\begin{enumerate}
  \item Найдите все постоянные решения каждого уравнения.
  \item Найдите все решения каждого уравнения. 
  \item Сколько постоянных решений имеет уравнение $y_t = 1 + \beta y_{t-1}$ в зависимости от $\beta$?
\end{enumerate}
\begin{sol}
  \begin{enumerate}
    \item $a_t = 2$, уравнение $(B)$ не имеет постоянных решений, $c_t = -1$
    \item $a_t = 2 + d 0.5^t$, $b_t = d + t$, $c_t = -1 + d 2^t$. 
    \item уравнение $y_t = 1 + \beta y_{t-1}$ имеет единственное постоянное решение при $\beta \neq 1$ 
  \end{enumerate}  
\end{sol}
\end{problem}
  

\begin{problem}
Рассмотрим модель $y_t=\mu + \e_t$, где $\e_t$ — стационарный AR(1) процесс $\e_t=\rho \e_{t-1} + u_t$ с $u_t \sim \cN(0,\sigma^2)$. 

Найдите условную логарифмическую функцию правдоподобия $\ln f(y_2, y_3, \ldots, y_n \mid \mu, \rho, \sigma^2, y_1)$.
\begin{sol}

\end{sol}
\end{problem}

\begin{problem}
Известно, что $\e_t$ — белый шум. 
Классифицируйте в рамках классификации ARIMA процесс $y_t=1+\e_t + 0.5\e_{t-1} + 0.4\e_{t-2} + 0.3\e_{t-3} + 0.2y_{t-1} + 0.1y_{t-2}$.
\begin{sol}

ARMA(2,3), ARIMA(2,0,3)
\end{sol}
\end{problem}


\begin{problem}
На графике представлены данные по уровню озера Гур\'{о}н в футах в 1875-1972 годах:

\begin{minted}[mathescape,
               linenos,
               numbersep=5pt,
               frame=lines,
               framesep=2mm]{r}
level <- LakeHuron
df <- data.frame(level, obs = 1875:1972)
n <- nrow(df) # used later for answers
v.acf <- acf(level, plot = FALSE)$acf
v.pacf <- pacf(level, plot = FALSE)$acf
acfs.df <- data.frame(lag = c(1:15, 1:15),
    acf = c(v.acf[2:16], v.pacf[1:15]),
    acf.type = rep(c("ACF", "PACF"), each = 15))
model <- arima(level, order = c(1, 0, 1))
resids <- model$residuals
resid.acf <- acf(resids, plot = FALSE)$acf
\end{minted}



\begin{minted}[mathescape,
               linenos,
               numbersep=5pt,
               frame=lines,
               framesep=2mm]{r}
tikz("../R_plots/huron_ts.tikz", standAlone = FALSE, bareBones = TRUE)
ggplot(df, aes(x = obs, y = level)) + geom_line() +
    labs(x = "Год", y = "Уровень озера (футы)")
dev.off()
\end{minted}


%\begin{minipage}{\textwidth}
%\begin{tikzpicture}[scale = 0.025]
%\input{R_plots/huron_ts.tikz}
%\end{tikzpicture}
%\end{minipage}




График автокорреляционной и частной автокорреляционной функций:

\begin{minted}[mathescape,
               linenos,
               numbersep=5pt,
               frame=lines,
               framesep=2mm]{r}
ggplot(acfs.df, aes(x = lag, y = acf, fill = acf.type))+
    geom_histogram(position = "dodge", stat = "identity")+
  xlab("Лаг") + ylab("Корреляция") +
  guides(fill = guide_legend(title = NULL))+
  geom_hline(yintercept = 1.96 / sqrt(nrow(df)))+
  geom_hline(yintercept = -1.96 / sqrt(nrow(df)))
\end{minted}


\begin{enumerate}
\item Судя по графикам, какие модели класса ARMA или ARIMA имеет смысл оценить?
\item По результатам оценки некоей модели ARMA c двумя параметрами, исследователь посчитал оценки автокорреляционной функции для остатков модели. 
Известно, что для остатков модели первые три выборочные автокорреляции равны соответственно $0.00467$, $-0.0129$ и $-0.063$. 
С помощью подходящей статистики проверьте гипотезу о том, что первые три корреляции ошибок модели равны нулю.
\end{enumerate}


\begin{sol}
\begin{enumerate}
\item Процесс $AR(2)$, т.к. две первые частные корреляции значимо отличаются от нуля, а гипотезы о том, что каждая последующая равна нулю не отвергаются.
\item Можно использовать одну из двух статистик
\[
\text{Ljung-Box}=n(n+2)\sum_{k=1}^3\frac{\hat{\rho}_k^2}{n-k}=
0.42886
\]
\[
\text{Box-Pierce}=n\sum_{k=1}^3\hat{\rho}_k^2=
0.4076
\]
Критическое значение хи-квадрат распределения с 3-мя степенями свободы для $\alpha=0.05$ равно $\chi^2_{3,crit}=7.81$.
Вывод: гипотеза $H_0$ об отсутствии корреляции ошибок модели не отвергается.
\end{enumerate}
\end{sol}
\end{problem}




\begin{problem}
Процесс $x_t$ — это процесс $y_t$, наблюдаемый с ошибкой, т.е. $x_t=y_t+\nu_t$. 
Ошибки $\nu_t$ являются белым шумом и не коррелированы с $y_t$.
\begin{enumerate}
\item Является ли процесс $x_t$ MA(1) процессом, если $y_t$ —  MA(1) процесс? 
Если да, то как связаны их автокорреляционные функциии?
\item Является ли процесс $x_t$ стационарным AR(1) процессом, если $y_t$ —  стационарный AR(1) процесс? 
Если да, то как связаны их автокорреляционные функциии?
\end{enumerate}


\begin{sol}

\end{sol}
\end{problem}


\begin{problem}
Рассмотрим стационарный AR(1) процесс $y_t=\rho y_{t-1} + \e_t$, где $\e_t \sim \cN(0,1)$. 
Имеется ряд $y_1$, $y_2$, \ldots, $y_{101}$. Построен график этого процесса. 
Как от $\rho$ зависит математическое ожидание количества пересечений графика с осью абсцисс?


\begin{sol}
Среднее количество пересечений равно 50 помножить на вероятность того, что два соседних $y_t$ разного знака. 
Найдём вдвое меньшую вероятность, $\P(y_1>0, y_2 <0)$.
\end{sol}
\end{problem}



\begin{problem}
Рассмотрим процессы:

\begin{enumerate}
\item[A] Процесс скользящего среднего:
\[
y_t=\e_t+2\e_{t-1}+3
\]

\item[B]
\[
a_t=\e_t+\e_1 + 3
\]

\item[C]
\[
b_t=t\e_t + 3
\]

\item[D]
\[
c_t=\cos\left(\frac{\pi t}{2}\right)\e_1 + \sin\left(\frac{\pi t}{2}\right)\e_2 + 2
\]

\item[E] Процесс случайного блуждания со смещением:
\[
\begin{cases}
z_t=\e_t+z_{t-1}+3 \\
z_0=0
\end{cases}
\]

\item[F] Процесс с трендом:
\[
w_t=2+3t+\e_t
\]

\item[G] Еще один процесс:
\[
r_t=\begin{cases}
1, \; \text{при четных t} \\
-1, \; \text{при нечетных t}
\end{cases}
\]

\item[H] Приращение случайного блуждания
\[
s_t=\Delta z_t
\]

\item[I] Приращение процесса с трендом
\[
d_t=\Delta w_t
\]
\end{enumerate}

Для каждого процесса:

\begin{enumerate}
\item Найдите $\E(y_t)$, $\Var(y_t)$
\item Найдите $\gamma_k = \Cov(y_t, y_{t-k})$
\item Найдите $\rho_k = \Corr(y_t,y_{t-k})$. 
Если ни одна корреляция $\rho_k$ не зависит от времени $t$, то постройте график зависимости $\rho_k$ от $k$.
\item Является ли процесс стационарным?
\item Сгенерируйте одну реализацию процесса. Постройте её график и график оценки автокорреляционной функции.
\end{enumerate}


\begin{sol}

\[
\E(b_t) = 3
\]

\[
\Var(b_t) = t^2 \sigma^2_{\e}
\]

\[
\Cov(b_t, b_{t-k}) = 0, k \geq 1
\]

\[
\Corr(b_t, b_{t-k}) = 0, k \geq 1
\]

$b_t$ — нестационарный из-за дисперсии


\[
\E(c_t) = 2
\]

\[
\Var(c_t) = \sigma^2_{\e}
\]

\[
\Cov(c_t, c_{t-k}) = \cos( \pi k /2)\sigma^2_{\e}, k \geq 1
\]

\[
\Corr(c_t, c_{t-k}) = \cos( \pi k /2), k \geq 1
\]

$c_t$ — стационарный
\end{sol}
\end{problem}





\begin{problem}
Эконометресса Антуанетта построила график автоковариационной функции временного ряда и распечатала его:

здесь график

Потом она с ужасом обнаружила, что до презентации исследования остается совсем мало времени, 
а распечатать надо было график автокорреляционной функции. 

Что надо исправить Антуанетте на графике, чтобы успеть ещё сделать причёску и макияж (это очень важно для презентации)?



\begin{sol}
зачеркнуть одну цифру
\end{sol}
\end{problem}


\begin{problem}
Рассмотрите стационарные процессы
\begin{enumerate}
\item[A.] AR(1): $y_t = 5 + 0.3y_{t-1} + \e_t$
\item[B.] AR(2): $y_t = 5 + 0.3y_{t-1} + 0.1 y_{t-2} + \e_t$
\item[C.] MA(1): $y_t = 5 + 0.3\e_{t-1} + \e_t$
\item[D.] MA(2): $y_t = 5 + 0.3\e_{t-1} + 0.9\e_{t-2} + \e_t$
\item[E.] ARMA(1, 1): $y_t = 5 + 0.3y_{t-1} + 0.4\e_{t-1} + \e_t$
\end{enumerate}

Если возможно, то представьте каждый процесс в виде:
\begin{enumerate}
\item $MA(\infty)$.
\item $AR(\infty)$.
\item $y_t = c + \gamma_1 y_{t-1} + u_t$, где $u_t$ некоррелирован с $y_{t-1}$. Будет ли $u_t$ белым шумом?
\item $y_t = c + \gamma_1 y_{t+1} + u_t$, где $u_t$ некоррелирован с $y_{t+1}$. Будет ли $u_t$ белым шумом?
\item $y_t = c + \gamma_1 y_{t-1} + \gamma_2 y_{t-2} + u_t$, где $u_t$ некоррелирован с $y_{t-1}$ и $y_{t-2}$. 
Будет ли $u_t$ белым шумом?
\item $y_t = c + \gamma_1 y_{t+1} + \gamma_2 y_{t+2} + u_t$, где $u_t$ некоррелирован с $y_{t+1}$ и $y_{t+2}$. 
Будет ли $u_t$ белым шумом?
\end{enumerate}
\begin{sol}
\end{sol}
\end{problem}


\begin{problem}
Рассмотрите стационарные процессы
\begin{enumerate}
\item[A.] AR(1): $y_t = 5 + 0.3y_{t-1} + \e_t$
\item[B.] AR(2): $y_t = 5 + 0.3y_{t-1} + 0.1 y_{t-2} + \e_t$
\item[C.] MA(1): $y_t = 5 + 0.3\e_{t-1} + \e_t$
\item[D.] MA(2): $y_t = 5 + 0.3\e_{t-1} + 0.9\e_{t-2} + \e_t$
\item[E.] ARMA(1, 1): $y_t = 5 + 0.3y_{t-1} + 0.4\e_{t-1} + \e_t$
\end{enumerate}

Для каждого из процессов:
\begin{enumerate}
\item Найдите математическое ожидание $\E(y_t)$.
\item Найдите первые три значения автокорреляционной функции $\rho_1$, $\rho_2$, $\rho_3$.
\item Найдите первые три значения частной автокорреляционной функции $\phi_{11}$, $\phi_{22}$, $\phi_{33}$.
\end{enumerate}
\begin{sol}
\end{sol}
\end{problem}

\begin{problem}
  Известна автокорреляционная функция стационарного процесса $(y_t)$: $\rho_1 = 0.7$, $\rho_2 = 0.3$, и $\rho_k = 0$ при $k\geq 3$. 
  Кроме того, $\E(y_t)=4$. 
  Выпишите возможные уравнения процесса.
\begin{sol}
По нулевым корреляциям догадываемся, что это процесс $MA(2)$.
\[
y_y = 4 + u_t + \alpha_1 u_{t-1} + \alpha_2 u_{t-2}
\]
\[
\begin{cases}
  \frac{\alpha_1 \alpha_2 + \alpha_1}{\alpha_2} = 7/3 \\
  \frac{\alpha_1^2 + \alpha_2^2 + 1}{\alpha_2} = 10/3 \\
\end{cases}
\]


\end{sol}
\end{problem}

\begin{problem}
  Известна частная автокорреляционная функция стационарного процесса $(y_t)$: $\phi_{11} = 0.7$, $\phi_{22} = 0.3$, и $\phi_{kk} = 0$ при $k\geq 3$. Кроме того, $\E(y_t)=4$. Выпишите возможные уравнения процесса.
\begin{sol}
\end{sol}
\end{problem}

\begin{problem}
Если возможно, то найдите процесс с данной автокорреляционной или частной автокорреляционной функцией.

\begin{enumerate}
  \item $ACF = (0.9, -0.9, 0, 0, 0, \ldots)$;
  \item $PACF = (0.9, -0.9, 0, 0, 0, \ldots)$;
  \item $PACF = (0.9, 0, 0, 0, 0, \ldots)$;
  \item $PACF = (0, 0.9, 0, 0, 0, 0, \ldots)$;
  \item $ACF = (0.9, 0, 0, 0, 0, \ldots)$;
  \item $ACF = (0, 0.9, 0, 0, 0, 0, \ldots)$;
\end{enumerate}

\begin{sol}
  \begin{enumerate}
    \item $ACF = (0.9, -0.9, 0, 0, 0, \ldots)$ не бывает, так как определитель корреляционной матрицы 3 на 3 отрицательный;
    \item $PACF = (0.9, -0.9, 0, 0, 0, \ldots)$ — AR(2);
    \item $PACF = (0.9, 0, 0, 0, 0, \ldots)$ — $y_t = 0.9y_{t-1} + u_t$;
    \item $PACF = (0, 0.9, 0, 0, 0, 0, \ldots)$ — $y_t = 0.9y_{t-2} + u_t$;
    \item $ACF = (0.9, 0, 0, 0, 0, \ldots)$ — не бывает, подозрение падает на MA(1), но решения только с комплексными коэффициентами, геометрически: два угла с косинусом 0.9, то есть примерно по 30 градусов, и они даже в сумме не могут дать перпендикуляр;
    \item $ACF = (0, 0.9, 0, 0, 0, 0, \ldots)$ — не бывает, если проредить процесс через один, то должна получится невозможная ACF;
  \end{enumerate}
   В целом PACF может быть любая,
   \url{http://projecteuclid.org/euclid.aos/1176342881}.
\end{sol}
\end{problem}


\begin{problem}
Рассмотрим стационарный процесс $y_t = 4 + 0.7y_{t-1} - 0.12y_{t-2} + \e_t$, где $\e_t$ — белый шум, причём $\Cov(\e_t, y_{t-k})=0$ при $k \geq 1$.

\begin{enumerate}
  \item Найдите автокорреляционную функцию: $\rho_1$, $\rho_2$ и общую формулу для $\rho_k$.
  \item Найдите $\lim_{k \to \infty} \rho_k$.
  \item Найдите частную автокорреляционную функцию: $\phi_{11}$, $\phi_{22}$, \ldots.
\end{enumerate}
\begin{sol}
  $\phi_{kk}=0$ при $k \geq 3$.
\end{sol}
\end{problem}


\begin{problem}
Рассмотрим стационарный процесс с уравнением
\[
y_t = 10 + 0.69 y_{t-1} + \e_t - 0.71 \e_{t-1}.
\]

Выпишите гораздо более простой процесс со свойствами близкими к свойствам данного процесса.
\begin{sol}
Заметим, что $0.69\approx 0.71$, сокращаем множитель $1-0.7L$, получаем $y_t = 100/3 + \e_t$.
\end{sol}
\end{problem}


\begin{problem}
Процесс $\e_t$ — белый шум. Рассмотрим уравнение
\[
y_t = 0.5 y_{t-1} + \e_t.
\]

Какие из указанных процессов $(y_t)$ являются его решением? Стационарным решением?
\begin{enumerate}
  \item $y_t = 0.5^t$;
  \item $y_t = \sum_{i=0}^{\infty} 0.5^i \e_{t-i}$;
  \item $y_t = 0.5^t + \sum_{i=0}^{\infty} 0.5^i \e_{t-i}$;
  \item $y_t = 0.5^t\e_{100} + \sum_{i=0}^{\infty} 0.5^i \e_{t-i}$;
  \item $y_t = 0.5^t + \sum_{i=0}^{t} 0.5^i \e_{t-i}$;
  \item $y_t = \sum_{i=0}^{t} 0.5^i \e_{t-i}$;
\end{enumerate}


\begin{sol}
Стационарным решением является $y_t = \sum_{i=0}^{\infty} 0.5^i \e_{t-i}$. 
Решениями также являются: $y_t = 0.5^t + \sum_{i=0}^{\infty} 0.5^i \e_{t-i}$, $y_t = 0.5^t\e_{100} + \sum_{i=0}^{\infty} 0.5^i \e_{t-i}$, $y_t = 0.5^t + \sum_{i=0}^{t} 0.5^i \e_{t-i}$, $y_t = \sum_{i=0}^{t} 0.5^i \e_{t-i}$.
\end{sol}
\end{problem}



\begin{problem}

Рассмотрим стационарный процесс $y_t$, задаваемый уравнением
\[
y_t = 2 + 0.6 \cdot y_{t-1} - 0.08 y_{t-2} + \e_t,
\]
где $\e_t \sim \cN(0; 4)$.

\begin{enumerate}
\item  Найдите $\E_t(y_{t+1})$, $\Var_t(y_{t+1})$
\item Найдите $\E_t(y_{t+2})$, $\Var_t(y_{t+2})$
\item Постройте 95\%-ый предиктивный интервал для $y_{102}$, если $y_{99}=5$, $y_{100}=5.1$
\item Найдите $\E(y_t)$, $\Var(y_t)$
\item Найдите $\lim_{h\to\infty}\E_t(y_{t+h})$, $\lim_{h\to\infty}\Var_t(y_{t+h})$
\end{enumerate}


\begin{sol}

$\E_t(y_{t+1})=2+0.6y_{t-1}-0.08y_{t-2}$, $\Var_t(y_{t+1})=4$

$\E_t(y_{t+2})=3.2 + 0.28 y_t- 0.048y_{t-1}$, $\Var_t(y_{t+2})=1.36 \cdot 4$

$\E_{100}(y_{102})= 4.388$, $\Var_{100}(y_{102})=5.44$.

Предиктивный интервал $[4.388 - 1.96 \sqrt{5.44};4.388 + 1.96 \sqrt{5.44}]$

$\E(y_t)=\frac{2}{0.48}\approx 4.17$

\end{sol}
\end{problem}



\begin{problem}
Задан процесс $y_t = 7 + u_t + 0.2 u_{t-1}$, где $u_t$ независимы и нормальны $u_t \sim \cN(0;4)$. 
Известно, что $y_{100}=7.2$, $u_{100}=1.3$, $y_{100}+(-0.2)y_{99}+(-0.2)^2y_{98}+\ldots+(-0.2)^{99}y_1=5.6$.

Пусть $\cF_t=\sigma(y_t, y_{t-1}, \ldots, y_1, u_t, u_{t-1}, \ldots, u_1)$ и $\cH_t = \sigma(y_t, y_{t-1}, \ldots, y_1)$.
\begin{enumerate}
  \item Найдите $\E(y_{101} \mid \cF_{100})$, $\Var(y_{101}|\cF_{100})$.
  \item С помощью $AR(\infty)$ представления примерно найдите $\E(y_{101}|\cH_{100})$, $\Var(y_{101}|\cH_{100})$. 
  Постройте 95\%-ый предиктивный интервал для $y_{101}$.
  \item Найдите $\E(y_{101} \mid y_{100})$, $\Var(y_{101}|y_{100})$.
  \item Найдите $\E(y_{101} \mid y_{100}, y_{99})$, $\Var(y_{101}|y_{100}, y_{99})$.
\end{enumerate}

\begin{sol}
Заметим, что $\Var(u_t|\cF_t)=0$. Более того, для обратимого процесса $\Var(u_t|y_t, y_{t-1}, \ldots, y_1) \approx \Var(u_t|y_t, y_{t-1}, \ldots) = 0$.
\[
\E(y_{101}|y_{100})=7 + 0 + 0.2\E(u_{100}|y_{100})
\]
\[
\E(u_{100}|y_{100}) = \beta_1 + \beta_2 y_{100}
\]
\[
\beta_2 = \frac{\Cov(y_{100}, u_{100})}{\Var(y_{100})}=4/4.16, \beta_1 = \E(u_{100}) - \beta_2 \E(y_{100})=-4\cdot 7/4.16
\]
\[
\frac{y_t}{1+0.2L} = \frac{7}{1+0.2L} + u_t
\]
Заметим, что $\frac{7}{1+0.2L}=7/1.2$, так как $L\cdot 7 = 7$ (вчера семь равнялось семи).

По условию $\frac{y_{100}}{1+0.2L} \approx 5.6$. Знак «примерно равно» возникает из-за замены бесконечной суммы на конечную.

\end{sol}
\end{problem}



\begin{problem}
  У исследовательницы Аграфены три наблюдения, $y_1 = 0.1$, $y_2 = -0.2$, $y_3 = 0.2$. 
  Аграфена предполагает, что данные подчиняются стационарному AR(1) процессу $y_t = \beta y_{t-1} + u_t$ с $\abs{\beta}<1$ и независимыми $u_t \sim \cN(0;\sigma^2_u)$.

  \begin{enumerate}
    \item Найдите $\E(y_1)$, $\E(y_2 \mid y_1)$, $\E(y_3 \mid y_2)$;
    \item Найдите $\Var(y_1)$, $\Var(y_2 \mid y_1)$, $\Var(y_3 \mid y_2)$;
    \item Найдите функции плотности $f(y_1)$, $f(y_2 \mid y_1)$, $f(y_3 \mid y_2)$;
    \item Выпишете полную логарифмическую функцию правдоподобия $\ln f(y_1, y_2, y_3 \mid \beta, \sigma^2_u)$.
    \item Если возможно, явно решите задачу максимизации полного правдоподобия.
    \item Выпишите условную логарифмическую функцию правдоподобия $\ln f(y_2, y_3 \mid \beta, \sigma^2_u, y_1)$.
    \item Если возможно, явно решите задачу максимизации условного правдоподобия при фиксированном $y_1$.
  \end{enumerate}
  \begin{sol}
    $\E(y_1)=0$, $\Var(y_1)=\sigma^2_u/(1-\beta^2)$, $\E(y_t|y_{t-1})=\beta y_{t-1}$, $\Var(y_t|y_{t-1})=\sigma^2_u$.

    При максимизации условного правдоподобия получаем:
    \[
         \hb = \frac{y_1 y_2 + y_2 y_3}{y_1^2 + y_2^2}
    \]
  \end{sol}
\end{problem}

\begin{problem}
  Белые шумы $u_t$ и $v_t$ независимы, $\Var(u_t) = 1$, $\Var(v_t)=1$. Рассмотрим процесс $y_t = 5u_{t-1} - 4 v_{t-1} + u_t + v_t$.

  \begin{enumerate}
    \item Выпишите классическое представление процесса $y_t$ как ARMA-процесса.
    \item Выразите белый шум из полученного классического представления $y_t$ через белые шумы $(u_t)$ и $(v_t)$.
  \end{enumerate}
  \todo[inline]{можно подобрать цифры, чтобы коэффициент был хороший :)}
  \begin{sol}
  \end{sol}
\end{problem}



\begin{problem}
  Рассмотрим модель случайного блуждания, 
  \[
  \begin{cases}
    y_0 = c, \\
    y_t = y_{t-1} + u_t, \\
    u_t \sim \cN(0, \sigma^2_u) \text{ и независимы}\\
  \end{cases}
  \]
  \begin{enumerate}
    \item Найдите $\E(y_{10})$, $\Var(y_{10})$, закон распределения $y_{10}$;
    \item Найдите $\E(y_{10}|y_7)$, $\Var(y_{10}|y_7)$, условный закон распределения $y_{10}$ при известном $y_7$;
    \item Найдите условный закон распределения $y_{101}$ при известном $y_{100}$, 
    условный закон распределения $y_{102}$ при известном $y_{100}$.
    \item Постройте 95\%-й предиктивный интервал для $y_{101}$, 95\%-й предиктивный интервал для $y_{102}$, 
    если известно, что $c=4$, $\sigma^2_u = 9$, $y_{100}=20$.
    \item Оцените параметры $c$ и $\sigma^2_u$ методом максимального правдоподобия, если $y_1 = 4$, $y_2 = 7$, $y_3 = 6$.
    \item Оцените параметры $c$ и $\sigma^2_u$ методом максимального правдоподобия в общем случае.
  \end{enumerate}

  \begin{sol}
  \end{sol}
\end{problem}


\begin{problem}

  Процессы $y_t$ и $u_t$ стационарны и заданы системой уравнений
  \[
  \begin{cases}
    y_t = \beta y_{t-1} + u_t \\
    u_t = \alpha u_{t-1} + \nu_t, \\
  \end{cases}  
  \]
  где $(\nu_t)$ — белый шум. Коэффициенты $\beta$ и $\alpha$ по модулю меньше единицы.

  Исследовательница Ада оценивает обычную регрессию $\hat y_t = \hat \beta_1 + \hat \beta_2 y_{t-1}$ с помощью МНК.

  Какие оценки она получит при большом размере выборки?
  
  \begin{sol}
    \[
    \plim \hat \beta_2 = \frac{\beta + \alpha}{1 + \beta \alpha}  
    \]
  \end{sol}

\end{problem}


\begin{problem}
Процесс $(u_t)$ — белый шум с дисперсией $\sigma^2_u$. 
Процесс $(y_t)$ задан уравнением  $y_t = 5 + u_t + 2u_{t-1}$.

\begin{enumerate}
  \item Найдите $\E(y_t)$, $\Var(y_t)$, $\Cov(y_t, y_s)$.
\end{enumerate}

Про процесс $(z_t)$ известно, что он представим в виде 
$z_t = c + w_t + \alpha w_{t-1}$, где $(w_t)$ — белый шум с дисперсий $\sigma^2_w$.

Ожидание, дисперсия и автоковариационная функция процесса $(z_t)$ в точности такая же, 
как и у процесса $(y_t)$. А именно, $\E(z_t) = \E(y_t)$, $\Var(z_t) = \Var(y_t)$,
$\Cov(z_t, z_s) = \Cov(y_t, y_s)$. Однако, $\alpha \neq 2$.

\begin{enumerate}[resume]
  \item Найдите константы $c$, $\alpha$ и отношение $\sigma^2_w/\sigma^2_u$.
\end{enumerate}
  
  \begin{sol}
Если обозначить отношение дисперсий буквой $R = \sigma^2_w/\sigma^2_u$,
то равенство дисперсии и ковариации даёт систему уравнений: 
\[
  \begin{cases}
    \alpha R = 2 \\
    (1+\alpha^2)R = 5 \\
  \end{cases}
\]
Решений у неё два, старый процесс $(\alpha=2, R=1)$, и новый $(\alpha=0.5, R=4)$.
Из равенства ожиданий следует, что $c=5$.
  \end{sol}
\end{problem}


\begin{problem}
Приведите три различных последовательности чисел $(a_t)_{t=-\infty}^{+\infty}$ таких, 
что $(1+0.5L)a_t = 0$.

  \begin{sol}
    Берем любое $a_0$, а дальше в обе стороны заполняем числа по принципу $a_t = -0.5 a_{t-1}$.
  \end{sol}
\end{problem}


\begin{problem}
  Процесс $(u_t)$ — белый шум.

  Рассмотрим процесс $w_t = (1+2L)(1-0.5L + 0.5^2 L^2 - 0.5^3 L^3 + \ldots )u_t$.

  \begin{enumerate}
    \item Верно ли, что $w_t$ белый шум?
    \item Придумайте ещё парочку белых шумов, линейно выражающихся через шум $u_t$.
  \end{enumerate}

  \begin{sol}
\begin{enumerate}
  \item 
Пусть $(u_t)$ — белый шум, рассмотрим следующий процесс:
\[
    w_t = (1 + 2L) (1 - 0.5L + 0.5^2 L^2 - 0.5^3 L^3+ \ldots)u_t
\]

Выпишем сначала определение белого шума $(u_t)$, а затем проверим все ли свойства выполняются для $(w_t)$.
\[
    \begin{cases}
       \E(u_t) = 0 \\
        \Var(u_t) = \sigma^2 \\
        \Cov(u_t, u_s) = 0 \quad \forall s \neq t
    \end{cases}
    \]

Преобразуем выражение для $w_t$:
\begin{align*}
    w_t =& (1 + 2L) (1 - 0.5L + 0.5^2 L^2 - 0.5^3 L^3+ \ldots)u_t \\ \\
    &\quad \Rightarrow \quad w_t = \frac{1 + 2L}{1 - 0.5L}\cdot u_t \\
    &\quad \Rightarrow \quad (1 - 0.5L)w_t = (1 + 2L) u_t \\
    &\quad \Rightarrow \quad w_t - 0.5w_{t-1} = u_t + 2u_{t+1} \\
    &\quad \Rightarrow \quad w_t = u_t + 2u_{t+1} + 0.5w_{t-1}
\end{align*}

Считаем, что процесс $(w_t)$ является стационарным, то есть для него выполняется:
\[
    \begin{cases}
        \E(w_t) = \mu \\
        \Var(w_t) = \sigma_w^2 \\
        \Cov(w_t, w_{t-k}) = \gamma_k \quad \forall k
    \end{cases}
\]

Теперь наконец найдём математическое ожидание $w_t$ используя выписанные выше свойства процессов $(u_t)$ и $(w_t)$.
\begin{gather*}
    \E(w_t) = \E(u_t + 2u_{t+1} + 0.5w_{t-1}) = \E(u_t) + 2\cdot \E(u_{t+1}) + 0.5\cdot \E(w_{t-1}) = 0.5\cdot \E{w_t} \quad \Rightarrow \quad \E{w_t} = 0
\end{gather*}

Из стационарности $(w_t)$ дисперсия $\Var{w_t}$ уже не зависит от $t$, следовательно, второе свойство из системы для белого шума тоже выполняется. 
Осталось найти коварицию $w_t$ и $w_{t-k}$ для произвольного $k$ и показать, что она равна 0, сделаем это с помощью индукции. Тогда базой является следующее равенство:
\[
    \Cov(w_t, w_{t-1}) = 0
\]

Раскроем коварицию и покажем, что это выполняется.
\begin{align*}
    \Cov(w_t, w_{t-1}) &= \Cov((1 + 2L) (1 - 0.5L + 0.5^2 L^2 - \ldots)u_t, (1 + 2L) (1 - 0.5L + 0.5^2 L^2 - \ldots)u_{t-1}) = \\
    &= \Cov(u_t + (2 - 0.5)u_{t-1} + (-1 +0.5^2)u_{t-2} + \ldots, u_{t-1} + (2 - 0.5)u_{t-2} + \ldots) = \\
    &= \left((2 - 0.5) + (-1 + 0.5^2)(2 - 0.5) + (0.5 - 0.5^3)(-1 + 0.5^2) + \ldots\right) \sigma^2 = \\
    &= \left((2 - 0.5) + \sum_{i = 0}^{\infty}(-1 + 0.5^2)\cdot (-0.5)^i \cdot (2 - 0.5) \cdot (-0.5)^i\right) \sigma^2 = \\
    &= \left((2 - 0.5) - (1 - 0.5^2)(2 - 0.5)\cdot \sum_{i = 0}^{\infty}(-0.5^2)^i\right) \sigma^2 = \\
    &= \left((2 - 0.5) - \cancel{(1 - 0.5^2)}(2 - 0.5)\cdot \frac{1}{\cancel{(1 - 0.5^2)}}\right) \sigma^2 = \\
    &= \big((2 - 0.5) - (2 - 0.5)\big) \sigma^2 = 0 \\
\end{align*}

Теперь докажем шаг индукции. Пусть для $k-1 > 0$ верно, что $\Cov(w_t, w_{t-(k-1)}) = 0$, выведем аналогичное утверждение для $k$.
\begin{align*}
    \Cov(w_t, w_{t-k+1}) &= \Cov(w_t, u_{t-k+1} + 2u_{t-k+2} + 0.5w_{t-k}) = \\
    &= \Cov(w_t, u_{t-k+1} + 2u_{t-k+2}) + 0.5 \cdot \Cov(w_t, w_{t-k}) \\ 
    \Cov(w_t, u_{t-k+1} + 2u_{t-k+2}) &= \Cov((1 + 2L) (1 - 0.5L + 0.5^2 L^2 - \ldots)u_t, u_{t-k+1} + 2u_{t-k+2}) = \\
    &= \Cov(u_t + (2 - 0.5)u_{t-1} + (-1 +0.5^2)u_{t-2} + \ldots, u_{t-k+1} + 2u_{t-k+2}) = \\
    &= \sum_{i=0}^{\infty} \Cov((2 - 0.5) \cdot (-0.5)^i u_{t-i-t}, u_{t-k+1} + 2u_{t-k+2}) = \\
    &= \left[\begin{aligned}
        t - i - 1 = t - k + 1 \quad \Rightarrow \quad i = k - 2 \\
        t - i - 1 = t - k + 2 \quad \Rightarrow \quad i = k - 3 \\
    \end{aligned}\right] = \\
    &= (2 - 0.5) \cdot (-0.5)^{k-2} \sigma^2 + (2 - 0.5) \cdot (-0.5)^{k-3}\cdot 2 \sigma^2 = \\
    &= (2 - 0.5) \cdot (-0.5)^{k-2} \sigma^2 \left(1 - 0.5 \cdot 2\right) = 0 \\ \\
    \Rightarrow \quad \Cov(w_t, w_{t-k}) &= 2\big(\Cov(w_t, w_{t-k+1}) - \Cov(w_t, u_{t-k+1} + 2u_{t-k+2})\big) = 0
\end{align*}

Значит, третье свойство из системы для белого шума тоже выполняется, и $(w_t)$ действительно является белым шумом. 

\item 
Как можно видеть из доказательства выше, умножение или деление на $(1 + \alpha L)$ для любого $|\alpha| \neq 1$ сохраняет белый шум. 
Аналогичное верно и для умножения или деления на $(1 + \alpha F)$ для любого $|\alpha| \neq 1$.

Тогда белым шумом являются и следующие стационарные процессы:
\begin{gather*}
    y_t = \frac{(1 + 0.2F)}{(1 + 0.3F)} u_t = (1 + 0.2F)(1 + 0.3F + 0.3^2F^2 + 0.3^3 F^3 + \ldots) u_t\\
    v_t = (1 - 3L)(1 + 0.2 F) u_t = (1 - 3L + 0.2F - 0.6LF)u_t = (0.4 - 3L + 0.2F) u_t
\end{gather*}
\end{enumerate}
  \end{sol}
\end{problem}





\begin{problem}
  Рассмотрим $MA(1)$ процесс $(y_t)$. 
  \begin{enumerate}
    \item В каких пределах может лежать корреляция $\Corr(y_t, y_{t+1})$?
    \item В каких пределах может лежать частная корреляция $\pCorr(y_t, y_{t+2} ; y_{t+1})$?
  \end{enumerate}
  
  \begin{sol}
    $\Corr(y_t, y_{t+1})  = a/(1+a^2) \in [-0.5; 0.5]$, $\pCorr(y_t, y_{t+2} ; y_{t+1}) = -a^2/(1 + a^2 + a^4) \in [-1/3; 0]$;

    Можно вспомнить, что $t + 1/t \geq 2$ при $t >0$ и обойтись без производных.
  \end{sol}
\end{problem}





\begin{problem}
Процессы $(a_t)$ и $(b_t)$ — обычное и сезонное случайные блуждания. 
Стартовые значения равны нулю, $a_0=0$, $b_{-11} = b_{-10} = \ldots = b_{-1} = 0$.
И далее $a_t = a_{t-1} + u_t$, $b_t = b_{t-12} + \nu_t$. 
Случайные процессы $(u_t)$ и $(\nu_t)$ — независимые белые шумы. 

\begin{enumerate}
  \item Получится ли взять несколько раз обычную разность $\Delta = 1 - L$ так, чтобы процесс $\Delta^d a_t$ был стационарным? 
  \item Получится ли взять несколько раз обычную разность $\Delta = 1 - L$ так, чтобы процесс $\Delta^d b_t$ был стационарным? 
  \item Как изменятся ответы на предыдущие вопросы, если брать сезонную разность $\Delta_{12} = 1 - L^{12}$?
\end{enumerate}

  \begin{sol}
    Процессы $\Delta b_t$, $\Delta_{12} a_t$, $\Delta_{12} b_t$ — стационарные. 
    Превратить сезонное случайное блуждание в стационарный процесс взятием обычной разности не получится. 
  \end{sol}
\end{problem}


\begin{problem}


  \begin{sol}
  \end{sol}
\end{problem}


\begin{problem}


  \begin{sol}
  \end{sol}
\end{problem}


% !TEX root = ../ts_pset_main.tex


\chapter{ETS}

Почитать про ETS модели в книжке \cite{hyndman2018forecasting}.

\begin{problem}
  Рассмотрим ETS-ANN модель с $\alpha = 1/2$, $y_1=6$, $y_2=9$, $y_3 = 6$, $\sigma^2=9$.


  \begin{enumerate}
    \item Найдите величину $l_0$, которая минимизирует $RSS$;
    \item Постройте точечный прогноз $\hat y_{4|2}$, $\hat y_{5|2}$;
     \item Постройте 95\%-ый предиктивный интервал для $y_{4}$ и $y_{5}$.
  \end{enumerate}
\begin{sol}
  \[
    \hat y_{4|3} = l_3 
  \]
  \[
    y_4 - \hat y_{4|3} = l_3 + \varepsilon_4 - l_3 = \varepsilon_4  
  \]
  \[
  \Var(y_4 - \hat y_{4|3} \mid \mathcal{F}_3) = \Var(\varepsilon_4 \mid \mathcal{F}_3) = \Var(\varepsilon_4)  
  \]
  \[
  \hat y_{5|3} = l_3 
  \]
  \begin{multline}
  y_5 - \hat y_{5|3} = l_4  + \varepsilon_5 - l_3  = (l_3 + \alpha \varepsilon_4)  + 
   \varepsilon_5 - l_3 = \\
  = \varepsilon_5 + \alpha  \varepsilon_4 
  \end{multline}
  \[
  \Var(y_5 - \hat y_{5|3} \mid \mathcal{F}_3) = \Var(\varepsilon_5 + \alpha  \varepsilon_4 )  
  \]
    
\end{sol}
\end{problem}

\begin{problem}
  Рассмотрим ETS-AAN модель с $\alpha = 1/2$, $\beta=3/4$, $l_{0}=7$, $b_0=2$, $y_1=6$, $y_2=9$, $y_3=3$, $\sigma^2=9$.
  \begin{enumerate}
    \item Постройте точечный прогноз $\hat y_{4|3}$, $\hat y_{5|3}$;
    \item Постройте 95\%-ый предиктивный интервал для $y_{4}$ и $y_{5}$.
  \end{enumerate}
\begin{sol}
\[
\hat y_{4|3} = l_3 + b_3
\]
\[
y_4 - \hat y_{4|3} = l_3 + b_3 + \varepsilon_4 - (l_3 + b_3) = \varepsilon_4  
\]
\[
\Var(y_4 - \hat y_{4|3} \mid \mathcal{F}_3) = \Var(\varepsilon_4 \mid \mathcal{F}_3) = \Var(\varepsilon_4)  
\]
\[
\hat y_{5|3} = l_3 + 2b_3
\]
\begin{multline}
y_5 - \hat y_{5|3} = l_4 + b_4 + \varepsilon_5 - (l_3 + 2b_3) = (l_3 + b_3 + \alpha \varepsilon_4)  + 
(b_3 + \beta \varepsilon_4) + \varepsilon_5 - (l_3 + 2b_3) = \\
= \varepsilon_5 + (\alpha + \beta) \varepsilon_4 
\end{multline}
\[
\Var(y_5 - \hat y_{5|3} \mid \mathcal{F}_3) = \Var(\varepsilon_5 + (\alpha + \beta) \varepsilon_4 )  
\]
\end{sol}
\end{problem}

\begin{problem}
  Рассмотрим ETS-AAN модель с $\alpha = 1/2$, $\beta=3/4$, $l_{0}=7$, $y_1=6$, $y_2=9$, $\sigma^2=16$.


  \begin{enumerate}
    \item Найдите величину $b_0$, которая минимизирует $RSS$;
    \item Постройте точечный прогноз $\hat y_{3|2}$, $\hat y_{4|2}$;
     \item Постройте 95\%-ый предиктивный интервал для $y_{3}$ и $y_{4}$.
  \end{enumerate}
\begin{sol}
\end{sol}
\end{problem}

\begin{problem}
  Рассмотрим ETS-AAN модель с $\alpha = 1/2$, $\beta=3/4$, $l_{0}=7$, $y_1=6$, $y_2=9$, $y_3=3$.

  Выпишите сумму квадратов ошибок прогнозов на один шаг вперёд через $b_0$.
  % минимизация тут длинная
\begin{sol}
\end{sol}
\end{problem}



\begin{problem}
  Рассмотрим ETS-AAN модель с $\alpha = 1/2$, $\beta=3/4$, $l_{99}=8$, $b_{99}=1$, $y_{99}=10$, $y_{100}=8$, $\sigma^2=16$.
  \begin{enumerate}
    \item Найдите $l_{100}$, $b_{100}$, $l_{98}$, $b_{98}$;
    \item Постройте точечный прогноз $\hat y_{101|100}$, $\hat y_{102|100}$;
    \item Постройте 95\%-ый предиктивный интервал для $y_{101}$ и $y_{102}$.
  \end{enumerate}
\begin{sol}
\end{sol}
\end{problem}


\begin{problem}
Для каждой из ETS моделей найдите эквивалентную модель класса ARIMA:
	\begin{enumerate}
		\item Простое экспоненциальное сглаживание, ETS-ANN;
		\item Аддитивное сглаживание Хольта, ETS-AAN;
		\item Аддитивное сглаживание Хольта с угасающим трендом, ETS-AAdN;
		\item Аддитивное сглаживание Хольта-Винтерса для месячных данных, ETS-AAA;
		\item Аддитивное сглаживание Хольта-Винтерса с угасающим трендом для месячных данных, ETS-AAdA;
		\item ETS-ANA;
	\end{enumerate}
\begin{sol}
	\begin{enumerate}
		\item Простое экспоненциальное сглаживание, ETS-ANN; ARIMA(0,1,1)
		\item Аддитивное сглаживание Хольта, ETS-AAN; ARIMA(0,2,2)
		\item Аддитивное сглаживание Хольта с угасающим трендом, ETS-AAdN; ARIMA(1,1,2)
		\item Аддитивное сглаживание Хольта-Винтерса для месячных данных, ETS-AAA; ARIMA(0,1,13)-SARIMA(0,1,0)
		\item Аддитивное сглаживание Хольта-Винтерса с угасающим трендом для месячных данных, ETS-AAdA; ARIMA(0,1,13)-SARIMA(0,1,0)
		\item ETS-ANA; ARIMA(0,1,12)-SARIMA(0,1,0)
	\end{enumerate}
\end{sol}
\end{problem}


\begin{problem}
Рассмотрим ETS-AAN модель. По каким параметрам модели оптимальные точки можно получить в явном виде?
\begin{sol}
По $l_0$, $b_0$;
\end{sol}
\end{problem}

\begin{problem}
Процесс $y_t$ описывается $ETS(MNM)$ моделью. 
Верно ли, что процесс $z_t = \ln y_t$ точно описывается $ETS(ANA)$ моделью? А примерно?
\begin{sol}
  Только примерно, $\ln (1 + x) \approx x$.
\end{sol}
\end{problem}
  

\begin{problem}
Рассмотрим $ETS(AA_dN)$ модель с $\phi = 0.9$, $\alpha=0.3$, $\beta=0.1$ и $\sigma^2=16$. 
Выразите 95\% предиктивный интервал для $y_{t+1}$ и $y_{t+2}$ через $\ell_t$, $b_t$, $y_t$ и $u_t$. 
\begin{sol}
\end{sol}
\end{problem}

\begin{problem}
Найдите $\E(y_t)$, $\Var(y_t)$, $\Cov(y_t, y_{t+1})$ для $ETS(AAN)$ модели с заданными $\ell_0$, $b_0$, $\alpha$, $\beta$ и $\sigma^2$.
\begin{sol}
\end{sol}
\end{problem}


\begin{problem}
Полугодовой $y_t$ моделируется с помощью $ETS(AAA)$ процесса:
    
\[
\begin{cases}
    u_t \sim \cN(0; 4) \\
    s_t = s_{t-2} + 0.1 u_t \\
    b_t = b_{t-1} + 0.2 u_t \\
    \ell_t = \ell_{t-1} + b_{t-1} + 0.3 u_t \\
    y_t = \ell_{t-1} + b_{t-1} + s_{t-2} + u_t \\
\end{cases}    
\]

\begin{enumerate}
    \item Известно, что $s_{100} = 2$, $s_{99} = -1.9$, $b_{100} = 0.5$, $\ell_{100} = 4$. Найдите 95\% предиктивный интервал для $y_{102}$. 
    \item В этой задаче все параметры известны. Сколько параметров оценивается в реальной задаче прогнозирования с помощью $ETS(AAA)$ модели?
\end{enumerate}
\begin{sol}
\end{sol}
\end{problem}

\begin{problem}
Вспомним $ETS(AAN)$ модель, кстати, вот и уравнения:

\[
\begin{cases}
y_t = \ell_{t-1} + b_{t-1} + u_t \\
\ell_t = \ell_{t-1} + b_{t-1} + \alpha u_t \\
b_t = b_{t-1} + \beta u_t \\
u_t \sim \cN(0;\sigma^2) \\
% s_t = s_{t-12} + \gamma \varepsilon_t \\
\end{cases}
\]

\begin{enumerate}
	\item 
	Докажите, что ни при каких $\ell_0$ и $b_0$ этот процесс не будет стационарным. 
	Или опровергните и приведите пример, при каких будет. 
	
	Константы $\alpha$, $\beta$ лежат в интервале $(0;1)$.
	
	\item При $l_{100} = 20$, $b_{100} = 2$, $\alpha=0.2$, $\beta=0.3$, $\sigma^2 = 16$ постройте
	интервальный прогноз на один и два шага вперёд. 
\end{enumerate}
\begin{sol}
\end{sol}
\end{problem}





\chapter{TBATS}

Оригинальная статья, \cite{de2011forecasting}.

Относим к ETS как модель с одной ошибкой в разных уравнениях. 

\begin{problem}
  Найдите предел
  \[
    \lim_{w \to 0} \frac{y^w - 1}{w}
  \]
\begin{sol}
  $\ln y$
\end{sol}
\end{problem}




% !TEX root = ../ts_pset_main.tex


\chapter{Вступайте в ряды Фурье!}


Суть преобразования Фурье. 
Вместо исходного временного ряда $x_0$, $x_1$, \ldots, $x_{N-1}$ мы получаем ряд комплексных чисел $X_0$, $X_1$, \ldots, $X_{N-1}$.
Эти комплексные числа $X_k$ показывают, насколько сильно проявляется каждая частота в исходном ряду.

Чтобы получить одно комплексное число $X_k$:

\begin{enumerate}
  \item Разрежем круг на $N$ равных частей. Каждая часть образует угол $2\pi/N$.
  \item Разместим исходные числа $x_0$, $x_1$, \ldots, $x_{N-1}$ на разрезах по часовой стрелке с шагом $k$.
    При этом число $x_0$ приходится на угол $0$; число $x_1$ — на угол $2\pi/N \cdot k$;
число $x_2$ — на угол $2\pi/N \cdot 2k$, и так далее.
  \item Трактуем $x_i$ как силу ветра в направлении разреза.
  \item $X_k$ — усреднённая сила ветра.
\end{enumerate}


Прямое преобразование Фурье задаётся формулой\footnote{Иногда множитель $1/N$ относят к обратному преобразованию Фурье, иногда поровну разносят как $1/\sqrt{N}$.}:
\[
  X_k = \frac{1}{N} \sum_{n=0}^{N-1} x_n w^{kn},
\]
где комплексное число $w$ кодирует поворот на $1/N$ часть круга по часовой стрелке, $w = \exp\left(\frac{-2i\pi}{N} \right)$.



Обратное преобразование Фурье
\[
  x_n = \sum_{k=0}^{N-1} X_k (w^{*})^{nk},
\]
где комплексное число $w^{*}$ является сопряжённым к числу $w$.


\begin{problem}
  Немножко теории:
  \begin{enumerate}
    \item Посмотрите видео от 3blue1brown, \url{https://www.youtube.com/watch?v=cV7L95IkVdE}.
    \item Прочтите про дискретное преобразование Фурье на brilliant, \url{https://brilliant.org/wiki/discrete-fourier-transform/}.
  \end{enumerate}
\begin{sol}
\end{sol}
\end{problem}



\begin{problem}
  Про Фурье :)
  \begin{enumerate}
    \item Зачем Фурье собирал огарки свечей в бенедиктинской артиллерийской школе?
    \item Первый раз Фурье был арестован за недостаточную поддержку якобинцев. За что Фурье был арестован во второй раз?
    \item После потерей французами Каира Фурье вёл переговоры о перимирии. Что было у него в руке в момент переговоров? 
    Что произошло с этим предметом?
  \end{enumerate}
  \begin{sol}
    \begin{enumerate}
      \item Чтобы заниматься математикой по ночам.
      \item За поддержку якобинцев.
      \item Кофейник. Был разбит пулей.
    \end{enumerate}
\end{sol}
\end{problem}


\begin{problem}
  Вспомним комплексные числа :)
  \begin{enumerate}
    \item Найдите сумму $7 + 7 \exp(2i\pi/3) + 7 \exp(4i\pi/3)$;
    \item Найдите сумму $6 + 4\exp(i\pi)$;
  \end{enumerate}

  \begin{sol}
  \end{sol}
\end{problem}


\begin{problem}
Найдите прямое преобразование Фурье последовательностей
\begin{enumerate}
  \item $1$, $4$, $1$, $4$, $1$, $4$;
  \item $1$, $9$;
  \item $8$;
  \item $1$, $0$, $0$, $0$;
\end{enumerate}
  \begin{sol}
  \end{sol}
\end{problem}

\begin{problem}
  Прямое преобразование Фурье можно записать в матричном виде $X = \frac{1}{N}Fx$.
  \begin{enumerate}
    \item Как устроена матрица $F$?
    \item Найдите $F\cdot F^{*}$, где $F^{*}$ — транспонированная и сопряжённая матрица к $F$;
    \item Как устроена матрица $F^{-1}$?
    \item Как записывается обратное преобразование Фурье в матричном виде?
  \end{enumerate}

  \begin{sol}
  \end{sol}
\end{problem}


\begin{problem}

Обратное преобразование Фурье задаётся формулой
\[
  x_n = \sum_{k=0}^{N-1} X_k (w^{*})^{nk},
\]
где комплексное число $w^{*}$ является сопряжённым к числу $w = \exp\left(\frac{-2i\pi}{N} \right)$.


  Докажите, что обратное преобразование Фурье, действительно, от комплексных чисел $(X_k)$ переходит к исходныму ряду $(x_n)$.
  \begin{sol}
  \end{sol}
\end{problem}


\begin{problem}
  В типичной задаче исходный ряд $x_0$, $x_1$, \ldots, $x_{N-1}$ является действительными числами.
  Докажите, что при дискретном преобразовании Фурье числа $X_k$ и $X_{N-k}$ являются комплексно-сопряжёнными.

  \begin{sol}
  \end{sol}
\end{problem}


\begin{problem}
  Рассмотрим ряд месячной периодичности. Число наблюдений делится на 12. Исследователь Василий рассматривает в качестве регрессоров следующие переменные: столбец из единиц,
  $\sin\left(\frac{2\pi}{12} t\right)$,
 $\cos\left(\frac{2\pi}{12} t\right)$,
 $\sin\left(\frac{2\pi}{12} 2t\right)$,
 $\cos\left(\frac{2\pi}{12} 2t\right)$,
 $\sin\left(\frac{2\pi}{12} 3t\right)$,
 $\cos\left(\frac{2\pi}{12} 3t\right)$,
 $\sin\left(\frac{2\pi}{12} 4t\right)$,
 $\cos\left(\frac{2\pi}{12} 4t\right)$,
 $\sin\left(\frac{2\pi}{12} 5t\right)$,
 $\cos\left(\frac{2\pi}{12} 5t\right)$,
 $\cos\left(\frac{2\pi}{12} 6t\right)$.

 \begin{enumerate}
   \item Являются ли эти регрессоры ортогональными?
   \item Василий рассматривает два варианта действий.
     Вариант А: построить 12 регрессий исходного ряда на каждый регрессор в отдельности. Вариант Б: построить одну регрессию.
     Будут ли отличаться коэффициенты при регрессорах?
    \item Можно ли добавить в качестве регрессора $\sin\left(\frac{2\pi}{12} 6t\right)$ или  $\cos\left(\frac{2\pi}{12} 7t\right)$?
  \end{enumerate}
 \begin{sol}
   Да, ряды являются ортогональными. Можно строить регрессии на эти регрессоры в любых комбинациях, оценки бет выходят одни и те же.
   Другие ряды добавить нельзя — будет строгая мультиколлинеарность.
 \end{sol}
 \end{problem}

 \begin{problem}
   Исследовательница Агриппина взяла ряд длиной 6 наблюдений и построила его регрессию на тригонометрические ряды Фурье:
   \[
     \hat x_t = 3.5 - 1.73 \sin(2\pi t/6) + 1.00 \cos(2\pi t/6) - 0.58\sin(4\pi t/6) + 1.00 \cos(4\pi t/6) +0.30 \cos(6\pi t/6)
   \]

   Найдите прямое преобразование Фурье исходного ряда.
   \begin{sol}
     На всякий случай, это был ряд $1$, $2$, $3$, $4$, $5$, $6$.
   \end{sol}
 \end{problem}


 \begin{problem}
   Исследовательница Агриппина взяла ряд длиной 6 наблюдений и нашла его преобразование Фурье:
   \[
     1.5, \; -\frac{1}{6}+\frac{1}{\sqrt{12}}i, \; 0, \; -\frac{1}{6}, \; 0, -\frac{1}{6} - \frac{1}{\sqrt{12}}i.
   \]
   \begin{enumerate}
     \item Найдите регрессию этого ряда на тригонометрические ряды Фурье;
     \item Восстановите исходный ряд;
   \end{enumerate}

   \begin{sol}
   $1$, $1$, $1$, $2$, $2$, $2$
   \end{sol}
 \end{problem}




% !TEX root = ../ts_pset_main.tex


\chapter{GARCH}

Книжечка: \cite{francq2019garch}.

% !TEX root = ../ts_pset_main.tex

\begin{center}
\textsc{Положение GARCH-модели среди классических моделей временных рядов}
\end{center}
\[
    Y_t = c + \sum\nolimits_{i=1}^p{{\phi_i}{Y_{t-i}}} + \varepsilon_t + \sum\nolimits_{j=1}^q{{\theta_j}{\varepsilon_{t-j}}} + \sum\nolimits_{j=1}^k{{\beta_j}{X_{tj}}} \text{,}
\]
\[
    \varepsilon_t = \sigma_t\cdot\xi_t \text{,}
\]
\[
    \sigma_t^2 = \omega + \sum\nolimits_{i=1}^s{\delta_i\sigma_{t-i}^2 + \sum\nolimits_{j=1}^r{\gamma_j\varepsilon_{t-j}^2}} \text{.}
\]

\begin{itemize}
  \item при $s=0$, $r=0$, $k=0$ ARMAX/GARCH --- это классическая ARMA($p,q$)-модель,
  \item при $s=0$, $r=0$ ARMAX/GARCH --- это ARMA($p,q$)-модель, в которой в качестве объясняющих переменных дополнительно включены экзогенные ряды $\{X_{t1}\}$,...,$\{X_{tk}\}$.
\end{itemize}

\begin{center}
\textsc{Пример использования GARCH-модели}
\end{center}

Пусть ${P_t}$ --- цена акции, фьючерса или значение некоторого индекса цен финансовых инструментов в момент времени $t$.
\begin{itemize}\index{доходность простая|textbf}\index{доходность логарифмическая|textbf}
  \item \textit{простой доходностью} называется $\frac{P_t-P_{t-1}}{P_{t-1}}$,
  \item \textit{логарифмической доходностью} называется $\ln\frac{P_t}{P_{t-1}}$.
\end{itemize}
\textsc{Связь между простой и логарифмической доходностью}
\begin{center}
    $\ln\frac{P_t}{P_{t-1}} = \ln\left(\frac{P_{t-1} + P_t - P_{t-1}}{P_{t-1}}\right) = \ln\left(1 + \frac{P_t - P_{t-1}}{P_{t-1}}\right)$.
\end{center}
Используя формулу Тейлора $\ln(1+x) = x + o(x)$ при $x\rightarrow0$, можем записать следующее приближенное равенство:
\begin{center}
    $\ln\frac{P_t}{P_{t-1}} \approx \frac{P_t - P_{t-1}}{P_{t-1}}$
\end{center}
при малых значениях простой доходности $\frac{P_t - P_{t-1}}{P_{t-1}}$.

В финансовой математике, как правило, используется логарифмическая доходность. Это связано с тем, что
\begin{center}
    $\ln\frac{P_T}{P_0} = \ln\frac{P_1}{P_0} + \ln\frac{P_2}{P_1} + ... + \ln\frac{P_{T}}{P_{T-1}}$,
\end{center}
т.\,е. логарифмическая доходность за период $[0;T]$ есть сумма логарифмических доходностей за периоды ${[0;1], [1;2], \ldots, [T-1;T]}$.
\begin{itemize}
  \item В качестве зависимой переменной $Y_t$ возьмём логарифмическую доходность $\ln\frac{P_t}{P_{t-1}}$ интересующего нас финансового инструмента.
  \item Простейшая модель для расчёта и прогнозирования волатильности --- ARMAX($p=0, q=0, k=0$)/GARCH($s=1, r=1$)-модель:
\end{itemize}
\[
    Y_t = c + \varepsilon_t \text{,}
\]
\[
    \varepsilon_t = \sigma_t\cdot\xi_t \text{,}
\]
\[
    \sigma_t^2 = \omega + \delta\cdot\sigma_{t-1}^2 + \gamma\cdot\varepsilon_{t-1}^2 \text{,}
\]
\begin{itemize}
  \item Дальнейшее изложение будем вести на примере данной модели.
\end{itemize}
\begin{Definition}\index{процесс GARCH|textbf}
Пусть $\omega > 0, \, \delta \geq 0, \, \gamma \geq 0, \, \delta + \gamma < 1$ --- некоторые параметры, а $\sigma_0, \, \xi_0, \, \xi_1, \, \xi_2, \ldots$ --- независимые случайные величины такие, что
\[
    \mathbb{E}\sigma_{0}^{2}=\frac{\omega }{1-\delta -\gamma} \text{,} \quad \mathbb{E}\xi_t = 0 \text{,} \quad \mathbb{E}\xi_t^2 = 1 \text{,} \quad t \geq 1 \text{.}
\]
В этом случае говорят, что последовательность случайных величин $\{\varepsilon_t\}_{t=0}^{\infty}$ образует \textit{GARCH(1,1)-процесс}, если выполнены следующие соотношения:
\[
    \varepsilon_0 = \sigma_0 \cdot \xi_0 \text{,}
\]
\[
    \varepsilon_t = \sigma_t \cdot \xi_t \text{,} \quad \sigma_t^2 = \omega + \delta \cdot \sigma_{t-1}^2 + \gamma \cdot \varepsilon_{t-1}^2 \text{,} \quad t \geq 1 \text{.}
\]
\end{Definition}

Напомним определения слабо стационарного процесса и белого шума.
\begin{Definition}
Случайный процесс $\{X_t\}_{t=0}^\infty$ называется \textit{слабо стационарным}, если
    \begin{enumerate}
      \item $\mathbb{E}{X_t^2} < \infty$ для всех $t \geq 0$;
      \item $\mathbb{E}{X_t} = \mathbb{E}{X_s}$ для всех $t,\, s \geq 0$;
      \item $\Variance{X_t} = \Variance{X_s}$ для всех $t,\, s \geq 0$;
      \item $\operatorname{cov}(X_{t+h},X_{s+h}) = \operatorname{cov}(X_{t},X_{s})$ для всех $t,\,s \geq 0$ и любого $h$ такого, что $t+h \geq 0$ и $s+h \geq 0$.
    \end{enumerate}
\end{Definition}

\begin{Definition}
Слабо стационарный процесс $\{X_t\}_{t=0}^\infty$ называется \textit{белым шумом}, если $\mathbb{E}{X_t} = 0$ и $\operatorname{cov}(X_t,X_s) = 0$ при $t, \,s \geq 0$, $t \neq s$.
\end{Definition}

Ниже мы покажем, что GARCH(1,1)-процесс $\{\varepsilon_t\}_{t=0}^{\infty}$ является белым шумом.

\begin{Lemma}\label{GARCH Lemma 1}
Пусть случайные величины $X_1, \ldots, X_m$ и $Y_1, \ldots, Y_n$ независимы в совокупности. Тогда для любых (борелевских) функций $f \colon \mathbb{R}^m \to \mathbb{R}^1$ и $g \colon \mathbb{R}^n \to \mathbb{R}^1$ случайные величины $U = f(X_1, \ldots, X_m)$ и $V = g(Y_1, \ldots, Y_n)$ независимы.
\end{Lemma}
\begin{proof}
См., например, Ширяев\,А.\,Н. \cite{Shiryaev_Prob}, гл. II, § 6, стр. 256.
\end{proof}

\begin{Lemma}\label{GARCH Lemma 2}
Пусть независимые случайные величины $X$ и $Y$ имеют конечное математическое ожидание. Тогда
\begin{itemize}
  \item[(i)] математическое ожидание случайной величины ${X}\cdot{Y}$ конечно;
  \item[(ii)] $\mathbb{E}[{X}\cdot{Y}] = \mathbb{E}{X}\cdot\mathbb{E}{Y}$.
\end{itemize}
\end{Lemma}

\begin{proof}
См. Ширяев\,А.\,Н. \cite{Shiryaev_Prob}, гл. II, § 6, стр. 267, теорема 6.
\end{proof}

\begin{Lemma}\label{GARCH Lemma 3}
Пусть случайные величины $X^2$ и $Y^2$ имеют конечное математическое ожидание. Тогда случайная величина $X \cdot Y$ также имеет конечное математическое ожидание.
\end{Lemma}

\begin{proof}
В силу свойства математического ожидания $|\mathbb{E}Z| \leq \mathbb{E}|Z|$ и неравенства $|X \cdot Y| \leq \tfrac{1}{2} \cdot X^2 + \tfrac{1}{2} \cdot Y^2$ получаем:
\[
    |\mathbb{E}[X \cdot Y]| \leq \mathbb{E}|X \cdot Y| \leq \tfrac{1}{2} \cdot \mathbb{E}X^2 + \tfrac{1}{2} \cdot \mathbb{E}Y^2 < \infty \text{.}
\]
\end{proof}

\begin{Lemma}\label{GARCH Lemma 4}
Для любого $t \geq 0$ случайные величины $\sigma_t$ и $\xi_t$ независимы.
\end{Lemma}

\begin{proof}
При $t = 0$ независимость случайных величин $\sigma_0$  и $\xi_0$ содержится непосредственно в определении GARCH(1,1)-процесса.

При $t = 1$ независимость $\sigma_1$ и $\xi_1$ следует из того, что случайные величины $\sigma_0$, $\xi_0$, $\xi_1$ независимы в совокупности, и того, что
$\sigma_1 = \sqrt{\omega + \delta\cdot\sigma_0^2 + \gamma\cdot\sigma_0^2\cdot\xi_0^2}$, т.\,е. $\sigma_1$ является функцией от $\sigma_0$, $\xi_0$.

Независимость $\sigma_t$ и $\xi_t$ при $t \geq 2$ обосновывается аналогично тому, как это сделано при $t = 1$. Действительно, $\sigma_t$ есть функция от $\sigma_0, \xi_0, \xi_1, \ldots, \xi_{t-1}$, при этом величины $\sigma_0, \xi_0, \xi_1, \ldots, \xi_t$ независимы в совокупности.
\end{proof}

\begin{Proposition}\label{GARCH Proposition 1}
Пусть последовательность случайных величин $\{\varepsilon_t\}_{t=0}^{\infty}$ образует GARCH(1,1)-процесс. Тогда для любого $t \geq 0$
\begin{itemize}
\item[(i)] $\mathbb{E}{\varepsilon_t^2}<\infty$;
\item[(ii)] $\mathbb{E}{{\varepsilon }_{t}}=0$;
\item[(iii)] $\mathbb{E}\varepsilon _{t}^{2}=\frac{\omega }{1-\delta -\gamma }$;
\item[(iv)] $\operatorname{cov}\left( {{\varepsilon }_{t}},{{\varepsilon }_{s}} \right)=0$ при $t\ne s$, $s\ge 0$.
\end{itemize}
\end{Proposition}

\begin{proof}
(i) ($t=0$) По условию случайные величины $\sigma_0^2$ и $\xi_0^2$ имеют конечное математическое ожидание. При этом независимость $\sigma_0^2$ и $\xi_0^2$ вытекает из независимости $\sigma_0$ и $\xi_0$. Следовательно, в силу леммы 2 случайная величина $\varepsilon_0^2 = \sigma_0^2\cdot\xi_0^2$ имеет конечное математическое ожидание.

($t=1$) Согласно лемме 4, случайные величины $\sigma_1$ и $\xi_1$ независимы. Значит, $\sigma_1^2$ и $\xi_1^2$ также независимы. Кроме того, по условию, математическое ожидание $\xi_1^2$ конечно, а конечность $\mathbb{E}{\sigma_1^2}$ вытекает из конечности $\mathbb{E}{\sigma_0^2}$, $\mathbb{E}{\varepsilon_0^2}$ и формулы $\sigma_1^2 = \omega + \delta\cdot\sigma_0^2 + \gamma\cdot\varepsilon_0^2$. Следовательно, $\varepsilon_1^2 = \sigma_1^2\cdot\xi_1^2$ имеет конечное математическое ожидание.

($t\geq2$) Доказательство конечности $\mathbb{E}{\varepsilon_t^2}$ при $t\geq2$ проводится аналогично случаю $t = 1$.

(ii) Для $t\geq0$ имеем
\[
    \mathbb{E}\varepsilon_t = \mathbb{E}[\sigma_t\cdot\xi_t] = \mathbb{E}{\sigma_t}\cdot\mathbb{E}{\xi_t} = 0 \text{.}
\]
Здесь мы воспользовались независимостью случайных величин $\sigma_t$ и $\xi_t$, а также $\mathbb{E}{\xi_t} = 0$.

(iii) ($t=0$) При $t=0$ имеем
\[
    \mathbb{E}{\varepsilon_0^2} = \mathbb{E}{\sigma_0^2}\cdot\mathbb{E}{\xi_0^2} = \frac{\omega}{1-\delta -\gamma}\cdot1 = \frac{\omega}{1-\delta -\gamma} \text{.}
\]

($t=1$) Пусть $t=1$. По лемме 4 и доказанному выше, получаем
\[
    \mathbb{E}{\varepsilon_1^2} = \mathbb{E}{\sigma_1^2}\cdot\mathbb{E}{\xi_1^2} = \mathbb{E}{\sigma_1^2} = \omega + \delta\cdot\mathbb{E}{\sigma_0^2} + \gamma\cdot\mathbb{E}{\varepsilon_0^2} =
\]
\[
    =\omega + \delta\cdot{\frac{\omega}{1-\delta -\gamma}} + \gamma\cdot{\frac{\omega}{1-\delta -\gamma}} = \frac{\omega}{1-\delta -\gamma} \text{.}
\]

($t\geq2$) Доказательство утверждения при $t\geq2$ выполняется аналогично рассмотренному случаю $t=1$.

(iv) Пусть $0\leq{s}<t$. Математическое ожидание $\xi_t$ конечно по определению GARCH(1,1)-процесса. Конечность математического ожидания случайной величины $\sigma_t\cdot\varepsilon_s$ следует из конечности $\mathbb{E}{\sigma_t^2}$ и $\mathbb{E}{\varepsilon_s^2}$, а также леммы \ref{GARCH Lemma 3}. Кроме этого, при $0\leq{s}<t$ случайные величины $\xi_t$ и $\sigma_t\cdot\varepsilon_s$ независимы. Поэтому
\[
    \operatorname{cov}(\varepsilon_t,\varepsilon_s) = \mathbb{E}[\varepsilon_t\cdot\varepsilon_s] = \mathbb{E}[\xi_t\cdot(\sigma_t\cdot\varepsilon_s)] = \mathbb{E}{\xi_t}\cdot\mathbb{E}[\sigma_t\cdot\varepsilon_s] = 0 \text{.}
\]
\end{proof}

\begin{Remark}
В ходе доказательства пункта (i) утверждения \ref{GARCH Proposition 1} попутно было установлено, что $\mathbb{E}\sigma_t^2 < \infty$ для всех $t \geq 0$.
\end{Remark}



\begin{problem}
Рассмотрим следующий AR(1)-ARCH(1) процесс: \\
$Y_{t}=1+0.5Y_{t-1}+\varepsilon_{t}$, $\varepsilon_{t}=\nu_{t}\cdot \sigma_{t}$ \\
$\nu_{t}$ независимые $N(0;1)$ величины. \\
$\sigma^{2}_{t}=1+0.8\varepsilon^{2}_{t-1}$\\
Также известно, что $Y_{100}=2$, $Y_{99}=1.7$
\begin{enumerate}
\item Найдите $E_{100}(\varepsilon^{2}_{101})$, $E_{100}(\varepsilon^{2}_{102})$, $E_{100}(\varepsilon^{2}_{103})$, $E(\varepsilon^{2}_{t})$
\item $Var(Y_{t})$, $Var(Y_{t}|\mathcal{F}_{t-1})$
\item Постройте доверительный интервал для $Y_{101}$:
\begin{enumerate}
\item проигнорировав условную гетероскедастичность
\item учтя условную гетерескедастичность
\end{enumerate}
\end{enumerate}
\begin{sol}
\end{sol}
\end{problem}




\begin{problem}
Рассмотрим GARCH(1,2) процесс $\e_t=\sigma_t \nu_t$, $\sigma^2=0.2+0.5\sigma^2_{t-1}+0.2\e_{t-1}^2+0.1\e_{t-2}^2$. Найдите безусловную дисперсию $\Var(y_t)$
\begin{sol}

\end{sol}
\end{problem}


\begin{problem}
Для GARCH(1,1) процесса $\e_t=\sigma_t \nu_t$, $\sigma^2_{t}= w + \alpha \e_{t-1}^2+ \beta \sigma^2_{t-1}$ найдите $\E( \E(\e_t^2 | \cF_{t-1} ) )$
\begin{sol}

\end{sol}
\end{problem}



\begin{problem}
Рассмотрим GARCH(1,1) процесс  $\e_t=\sigma_t \nu_t$, $\sigma^2_{t}=0.1 + 0.7 \sigma^2_{t-1} + 0.2 \e_{t-1}^2$. Известно, $\sigma_T=1$, $\e_T=1$. Найдите $\E(\sigma_{T+2}^2 | \cF_T )$.
\begin{sol}

\end{sol}
\end{problem}


\begin{problem}
Найдите безусловную дисперсию GARCH-процессов
\begin{enumerate}
\item $\e_t=\sigma_t \cdot z_t$, $\sigma^2_t=0.1+0.8\sigma^2_{t-1}+0.1\e^2_{t-1}$
\item $\e_t=\sigma_t \cdot z_t$, $\sigma^2_t=0.4+0.7\sigma^2_{t-1}+0.1\e^2_{t-1}$
\item $\e_t=\sigma_t \cdot z_t$, $\sigma^2_t=0.2+0.8\sigma^2_{t-1}+0.1\e^2_{t-1}$
\end{enumerate}


\begin{sol}
$1$, $2$, $2$
\end{sol}
\end{problem}


\begin{problem}
Являются ли верными следующие утверждения?
\begin{enumerate}
\item GARCH-процесс является процессом белого шума, условная дисперсия которого
изменяется во времени
\item Модель GARCH(1,1) предназначена для прогнозирования меры изменчивости цены
финансового инструмента, а не для прогнозирования самой цены инструмента
\item При помощи GARCH-процесса можно устранять гетероскедастичность
\item Безусловная дисперсия GARCH-процесса изменяется во времени
\item Модель GARCH(1,1) может быть использована для прогнозирования
волатильности финансовых инструментов на несколько торговых недель вперёд
\end{enumerate}


\begin{sol}
\end{sol}
\end{problem}



\begin{problem}
Рассмотрим GARCH-процесс $\e_t=\sigma_t \cdot z_t$, $\sigma^2_t=k+g_1\sigma^2_{t-1}+a_1\e^2_{t-1}$. Найдите
\begin{enumerate}
\item $\E(z_t)$, $\E(z_t^2)$, $\E(\e_t)$, $\E(\e_t^2)$
\item $\Var(z_t)$, $\Var(\e_t)$, $\Var(\e_t \mid \mathcal{F}_{t-1})$
\item $\E(\e_t \mid \mathcal{F}_{t-1})$, $\E(\e_t^2 \mid \mathcal{F}_{t-1})$, $\E(\sigma^2_t \mid \mathcal{F}_{t-1})$
\item $\E(z_tz_{t-1})$, $\E(z_t^2z_{t-1}^2)$, $\Cov(\e_t,\e_{t-1})$, $\Cov(\e_t^2,\e_{t-1}^2)$
\item $\lim_{h\to\infty} \E(\sigma^2_{t+h} \mid \mathcal{F}_t)$
\end{enumerate}


\begin{sol}
\end{sol}
\end{problem}



\begin{problem}
Используя 500 наблюдений дневных логарифмических доходностей $y_t$ ,
была оценена GARCH(1,1)-модель: $\hy_t=-0.000708+\he_t$, $\e_t=\sigma_t \cdot z_t$, $\sigma^2_t=0.000455+0.6424\sigma^2_{t-1}+0.2509\e^2_{t-1}$. Также известно, что $\hs^2_{499}=0.002568$, $\he^2_{499}=0.000014$, $\he^2_{500}=0.002178$.
Найдите
\begin{enumerate}
\item  $\hs^2_{500}$, $\hs^2_{501}$, $\hs^2_{502}$
\item Волатильность в годовом выражении в процентах, соответствующую
наблюдению с номером $t = 500$
\end{enumerate}


\begin{sol}
\end{sol}
\end{problem}


\begin{problem}
Рассмотрим ARCH(1) процесс
\[
\begin{cases}
y_t = 2 + \e_t \\
\e_t = \sigma_t \cdot \nu_t \\
\sigma^2_t = 10 + 0.5\e_t^2
\end{cases}
\]

\begin{enumerate}
\item Найдите $\Var(y_{101})$, постройте 95\%-ый предиктивный интервал для $y_{101}$
\item Известно, что $y_{100}=3$, постройте 95\%-ый предиктивный интервал для $y_{101}$
\item Известно, что $y_{100}=12$, постройте 95\%-ый предиктивный интервал для $y_{101}$
\end{enumerate}


\begin{sol}
\end{sol}
\end{problem}

\begin{problem}
Может ли у GARCH процесса условная дисперсия $\e_t$ быть больше, чем безусловная? А меньше, чем безусловная?
\begin{sol}
Да, может быть и больше, и меньше.
\end{sol}
\end{problem}

\begin{problem}
Как известно, у GARCH процесса условная дисперсия $\e_t$ может быть как больше, так и меньше безусловной.
\begin{enumerate}
\item Имеет ли смысл строить предиктивный интервал для $y_t$, используя условную дисперсию, если она больше безусловной?
\item При построении предиктивного интервала эконометресса Агнесса использует безусловную дисперсию, если она меньше условной, и условную дисперсию, если она меньше безусловной. Корректно ли поступает Агнесса?
\end{enumerate}
\begin{sol}

\end{sol}
\end{problem}



\begin{problem}
Рассмотрим процесс AR(1)-GARCH(1,1):
\[
\begin{cases}
y_t = 2 + 0.6 y_{t-1} + \e_t \\
\e_t = \sigma_t \cdot \nu_t \\
\sigma^2_t = 6 + 0.4 \sigma^2_{t-1} + 0.2\e_t^2
\end{cases}
\]


Найдите $\Var(\e_t | \cF_{t-1})$, $\Var(y_t| \cF_{t-1})$, $\Var(\e_t)$, $\Var(y_t)$


\begin{sol}
\[
\Var(\e_t | \cF_{t-1}) = \Var(y_t| \cF_{t-1}) = 6 + 0.4 \sigma^2_{t-1} + 0.2\e_t^2
\]
\[
\Var(\e_t) = 6/(1-0.4-0.2)=6/0.4=15
\]
\[
\Var(y_t)=15/(1-0.36)
\]
\end{sol}
\end{problem}




% !TEX root = ../ts_pset_main.tex


\Opensolutionfile{solution_file}[solutions/sols_040]
% в квадратных скобках фактическое имя файла

\chapter{Единичный корень}

\begin{problem}
Винни-Пух пытается выявить закономерность в количестве придумываемых им каждый день ворчалок.  Винни-Пух решил разобраться, является ли оно стационарным процессом, для этого он оценил регрессию

\[ \Delta \hat{y}_t = \underset{(0.5)}{4.5} - \underset{(0.1)}{0.4}y_{t-1} +\underset{(0.5)}{0.7} \Delta y_{t-1} \]

Из-за опилок в голове Винни-Пух забыл, какой тест ему нужно провести, то ли Доктора Ватсона, то ли Дикого Фуллера.

\begin{enumerate}
\item Аккуратно сформулируйте основную и альтернативную гипотезы
\item Проведите подходящий тест на уровне значимости 5\%
\item Сделайте вывод о стационарности ряда
\item Почему Сова не советовала Винни-Пуху пользоваться широко применяемым в Лесу $t$-распределением?
\end{enumerate}


\begin{sol}

\begin{enumerate}
\item $H_0$: ряд содержит единичный корень, $\beta=0$; $H_a$: ряд не содержит единичного корня, $\beta<0$
\item $ADF=-0.4/0.1=-4$, $ADF_{crit}=-2.89$, $H_0$ отвергается
\item Ряд стационарен
\item При верной $H_0$ ряд не стационарен, и  $t$-статистика имеет не $t$-распределение, а распределение Дики-Фуллера.
\end{enumerate}
\end{sol}
\end{problem}



\Closesolutionfile{solution_file}



\Opensolutionfile{solution_file}[solutions/sols_050]
% в квадратных скобках фактическое имя файла

\chapter{Векторная авторегрессия}



\begin{problem}
Рассмотрим систему уравнений:
\[
\begin{cases}
x_t = -\frac{1}{6}x_{t-1} + \frac{2}{6}y_{t-1} + \e_{xt} \\
y_t = -\frac{4}{6}x_{t-1} + \frac{1}{6}y_{t-1} + \e_{yt}
\end{cases}
\]
\begin{enumerate}
\item Есть ли у данной системы стационарное решение?
\item Если стационарное решение имеется, то найдите $\E(x_t)$ и $\E(y_t)$
\item Нарисуйте в осях $(x_t, y_t)$ типичную тракторию стационарного решения
\end{enumerate}

\begin{sol}

\end{sol}
\end{problem}


\begin{problem}
Рассмотрим систему уравнений:
\[
\begin{cases}
x_t = -0.2x_{t-1} + 0.6y_{t-1} + \e_{xt} \\
y_t = 1.2x_{t-1} + 0.4y_{t-1} + \e_{yt}
\end{cases}
\]
\begin{enumerate}
\item Есть ли у данной системы коинтегрированное решение?
\item Если коинтегрированное решение имеется, то найдите коинтеграционное соотношение и представьте модель в виде модели коррекции ошибок
\item Нарисуйте в осях $(x_t, y_t)$ типичную тракторию коинтегрированного решения
\end{enumerate}

\begin{sol}

\end{sol}
\end{problem}



\begin{problem}
Белые шумы $\e_t$ и $u_t$ независимы. Пусть $y_t=2-0.5t+u_t$, $x_t=1+0.5t+\e_t$.
\begin{enumerate}
\item Является ли процесс $z_t=x_t+y_t$ стационарным?
\item Являются ли процессы $x_t$ и $y_t$ коинтегрированными?
\end{enumerate}
\begin{sol}

$z_t$ стационарный, $x_t$ и $y_t$ не коинтегрированы
\end{sol}
\end{problem}


\begin{problem}
Два процесса $(x_y)$ и $(y_t)$ называются независимыми, если независимы любые случайные величины $x_s$ и $y_t$.

Докажите каждое утверждение или приведите контр-пример.
\begin{enumerate}
  \item Сумма двух белых шумов является белым шумом.
  \item Сумма двух независимых белых шумов является белым шумом.
  \item Сумма двух стационарных процессов стационарна.
  \item Сумма двух независимых стационарных процессов стационарна.
  \item Сумма двух нестационарных процессов нестационарна.
  \item Сумма двух независимых нестационарных процессов нестационарна.
\end{enumerate}

\begin{sol}

\end{sol}
\end{problem}

\begin{problem}
Какие процессы могут быть коинтегрированы: $x_t \sim I(0)$, $y_t \sim I(1)$, $z_t \sim I(2)$, $w_t \sim I(2)$, $s_t \sim I(1)$?

\begin{sol}
$y_t$ и $s_t$; $z_t$ и $w_t$.
\end{sol}

\end{problem}

\begin{problem}
Белые шумы $(\e_t)$ и $(u_t)$ независимы.

Классифицируйте каждый процесс\footnote{Если у уравнения не заданы начальные условия, то подразумевается стационарное решение, если оно, конечно, есть.} как $ARIMA(p, d, q)$, определите порядок интеграции каждого процесса и определите, какие пары процессов коинтегрированы:
\begin{enumerate}
  \item $a_t = 0.5 a_{t-1} + u_t$
  \item $b_t = b_{t-1} + u_t$, $b_0 = 0$
  \item $c_t = 0.5 b_t + \e_t$
  \item $d_t = 0.3 b_t + a_t$
  \item $e_t = e_{t-1} + \e_t$
  \item $g_t = g_{t-1} + b_t$
  \item $h_t = 0.7 h_{t-1} + b_t$
\end{enumerate}

\begin{sol}
  \begin{enumerate}
    \item $a_t = 0.5 a_{t-1} + u_t$, AR(1)
    \item $b_t = b_{t-1} + u_t$, $b_0 = 0$, ARIMA(0, 1, 0)
    \item $c_t = 0.5 b_t + \e_t$, ARIMA(0, 1, 1)
    \item $d_t = d_{t-1} + a_t$, ARIMA(1, 1, 0)
    \item $e_t = e_{t-1} + \e_t$, ARIMA(0, 1, 0)
    \item $g_t = g_{t-1} + b_t$, ARIMA(0, 2, 0)
    \item $h_t = 0.7 h_{t-1} + b_t$, ARIMA(1, 1, 0)
  \end{enumerate}
  коинтегрированы: $b_t$, $c_t$, $d_t$, $h_t$.
\end{sol}
\end{problem}

\begin{problem}
  Процессы $u_t$ и $\e_t$ — независимые белые шумы с дисперсиями $\sigma^2_u$ и $\sigma^2_{\e}$. Рассмотрим процессы
  \[
    y_t = \begin{cases} 
      y_{t-1} + \e_t, \text{ при } t>0, \\
      0, \text{ при } t = 0;
    \end{cases}
  \]
  \[
    z_t = 
\begin{cases} 
  z_{t-1} + \e_t + 0.5\e_{t-1}, \text{ при } t>0, \\
      0, \text{ при } t = 0;
    \end{cases}
  \]
  
  \[
    w_t = 
\begin{cases} 
  0.5w_{t-1} + y_{t-1} + u_t, \text{ при } t>0, \\
      0, \text{ при } t = 0;
    \end{cases}
  \]

  \[
    r_t = 
\begin{cases} 
      -2y_t + 0.5r_{t-1} + y_{t-1} + u_t, \text{ при } t>0, \\
      r_0, \text{ при } t = 0;
    \end{cases}
  \]

  \begin{enumerate}
    \item Найдите порядок интеграции каждого процесса;
    \item Какие пары процессов являются коинтегрированными? Найдите коинтеграционные соотношения для коинтегрированных пар.
  \end{enumerate}

\begin{sol}
Процессы $y_t$ и $z_t$ коинтегрированы, $z_t - 1.5y_t$ стационарен.
Процессы $y_t$ и $r_t$ коинтегрированы, $r_t + 2y_t$ стационарен.
\end{sol}
\end{problem}

\Closesolutionfile{solution_file}



% !TEX root = ../ts_pset_main.tex


\chapter{Модели состояние-наблюдение}
%%%%%%%%%%%%%%%%  фильтр Калмана

\begin{problem}Представьте процесс AR(1),
$y_{t}=0.9y_{t-1}-0.2y_{t-2}+\varepsilon_{t}$,
$\varepsilon\sim$WN(0;1) в виде модели состояние-наблюдение.
\begin{enumerate}
\item Выбрав в качестве состояний вектор $\left(%
\begin{array}{c}
  y_{t} \\
  y_{t-1} \\
\end{array}%
\right)$
\item Выбрав в качестве состояний вектор $\left(%
\begin{array}{c}
  y_{t} \\
  \hat{y}_{t,1} \\
\end{array}%
\right)$
\end{enumerate}
Найдите дисперсии ошибок состояний
\begin{sol}
\end{sol}
\end{problem}

\begin{problem}Представьте процесс MA(1),
$y_{t}=\varepsilon_{t}+0.5\varepsilon_{t-1}$,
$\varepsilon\sim$WN(0;1) в виде модели состояние-наблюдение.
\begin{enumerate}
\item $\left(%
\begin{array}{c}
  \varepsilon_{t} \\
  \varepsilon_{t-1} \\
\end{array}%
\right)$
\item  $\left(%
\begin{array}{c}
  \varepsilon_{t}+0.5\varepsilon_{t-1} \\
  0.5\varepsilon_{t}
\end{array}%
\right)$
\end{enumerate}
\begin{sol}
\end{sol}
\end{problem}


\begin{problem}Представьте процесс ARMA(1,1),
$y_{t}=0.5y_{t-1}+\varepsilon_{t}+\varepsilon_{t-1}$,
$\varepsilon\sim$WN(0;1) в
виде модели состояние-наблюдение. \\
Вектор состояний имеет вид $x_{t},x_{t-1}$, где
$x_{t}=\frac{1}{1-0.5L}\varepsilon_{t}$
\begin{sol}
\end{sol}
\end{problem}



\begin{problem} Рекурсивные коэффициенты
\begin{enumerate}
\item Oцените модель вида $y_{t}=a+b_{t}x_{t}+\varepsilon_{t}$,
где $b_{t}=b_{t-1}$.
\item Сравните графики filtered state и smoothed state.
\item Сравните финальное состояние $b_{T}$ с коэффициентом в
обычной модели линейной регрессии, $y_{t}=a+bx_{t}+\varepsilon_{t}$.
\end{enumerate}
\begin{sol}
\end{sol}
\end{problem}




\Closesolutionfile{solution_file}


\chapter{Решения и ответы к избранным задачам}
\chaptermark{Избранные решения}


% для гиперссылок на условия
% http://tex.stackexchange.com/questions/45415
\renewenvironment{solution}[1]{%
         % add some glue
         \vskip .5cm plus 2cm minus 0.1cm%
         {\bfseries \hyperlink{problem:#1}{#1.}}%
}%
{%
}%



\protect \hypertarget {soln:1.1}{}
\begin{solution}{{1.1}}
В данном случае статистика $DW$ не применима, так как есть лаг $y_{t-1}$ среди регрессоров.
\end{solution}
\protect \hypertarget {soln:1.2}{}
\begin{solution}{{1.2}}
\begin{enumerate}
\item $\E(\e_t)=0$, $\Var(\e_1)=\sigma^2$, $\Var(\e_t)=2\sigma^2$ при $t\geq 2$.  Гетероскедастичная.
\item $\Cov(e_t,e_{t+1})=\sigma^2$. Автокоррелированная.
\item $\hb$ --- несмещенная, неэффективная
\item Более эффективной будет $\hb_{gls}=(X'V^{-1}X)^{-1}X'V^{-1}y$, где
\[
X=\begin{pmatrix}
x_1 \\
x_2 \\
\vdots \\
x_n
\end{pmatrix}
\]

Матрица $V$ известна с точностью до константы $\sigma^2$, но в формуле для $\hb_{gls}$ неизвестная $\sigma^2$ сократится.

Другой способ построить эффективную оценку --- применить МНК к преобразованным наблюдениям, т.е. $\hb_{gls}=\frac{\sum x'_i y'_i}{\sum x_i^{\prime 2}}$, где $y'_1=y_1$, $x'_1=x_1$, $y'_t=y_t-y_{t-1}$, $x'_t=x_t-x_{t-1}$ при $t\geq 2$.
\end{enumerate}
\end{solution}
\protect \hypertarget {soln:1.3}{}
\begin{solution}{{1.3}}

Для простоты закроем глаза на малое количество наблюдений и как индейцы пираха будем считать, что пять --- это много.

\end{solution}
\protect \hypertarget {soln:1.4}{}
\begin{solution}{{1.4}}
1. Поскольку имеют место соотношения $\varepsilon_1 = \rho \varepsilon_0 + u_1$ и $Y_1 =\mu + \varepsilon_1$, то из условия задачи получаем, что $\varepsilon_1 \sim N(0,\sigma^2 / (1 - \rho^2))$
и $Y_1 \sim N(\mu,\sigma^2 / (1 - \rho^2))$. Поэтому
\[
f_{Y_1}(y_1) = \frac{1}{\sqrt{2\pi\sigma^2/(1-\rho^2)}}\exp{\left(-\frac{(y_1 - \mu)^2}{2\sigma^2/(1 - \rho^2)}\right)}.
\]

Далее, найдем $f_{Y_2|Y_1}(y_2|y_1)$. Учитывая, что $Y_2 = \rho Y_1 + (1- \rho) \mu + u_2$, получаем $Y_2|\{Y_1 = y_1\} \sim N(\rho y_1 + (1- \rho) \mu, \sigma^2)$. Значит,
\[
f_{Y_2|Y_1}(y_2|y_1) = \frac{1}{\sqrt{2\pi\sigma^2}}\exp{\left(-\frac{(y_2 - \rho y_1 - (1- \rho) \mu)^2}{2\sigma^2}\right)}.
\]

Действуя аналогично, получаем, что для всех $t \geq 2$ справедлива формула
\[
f_{Y_{t}|Y_{t-1}}(y_{t}|y_{t-1}) = \frac{1}{\sqrt{2\pi\sigma^2}}\exp{\left(-\frac{(y_{t} - \rho y_{t-1} - (1- \rho) \mu)^2}{2\sigma^2}\right)}.
\]

Таким образом, находим функцию правдоподобия
\[
\mathrm{L}(\mu, \rho, \sigma^2) = f_{Y_T,\ldots,Y_1}(y_T,\dots,y_1) = f_{Y_1}(y_1)\prod_{t=2}^{T}f_{Y_t|Y_{t-1}}(y_t|y_{t-1}) \text{,}
\]
где $f_{Y_1}(y_1)$ и $f_{Y_t|Y_{t-1}}(y_t|y_{t-1})$ получены выше.

2. Для нахождения неизвестных параметров модели запишем логарифмическую условную функцию правдоподобия:
\[
l(\mu, \rho, \sigma^2|Y_1 = y_1) = \sum_{t=2}^{T}\log{f_{Y_t|Y_{t-1}}(y_t|y_{t-1})} =
\]
\[
=-\frac{T-1}{2} \log(2 \pi) - \frac{T-1}{2} \log{\sigma^2} - \frac{1}{2\sigma^2} \sum_{t=2}^{T}(y_t - \rho y_{t-1} - (1 - \rho) \mu)^2 \text{.}
\]

Найдем производные функции $l(\mu, \rho, \sigma^2|Y_1 = y_1)$ по неизвестным параметрам:
\[
\frac{\partial l}{\partial \mu} = -\frac{1}{2\sigma^2} \sum_{t=2}^{T} 2(y_t - \rho y_{t-1} - (1 - \rho) \mu) \cdot (\rho - 1) \text{,}
\]
\[
\frac{\partial l}{\partial \rho} = -\frac{1}{2\sigma^2} \sum_{t=2}^{T} 2(y_t - \rho y_{t-1} - (1 - \rho) \mu) \cdot (\mu - y_{t-1}) \text{,}
\]
\[
\frac{\partial l}{\partial {\sigma^2}} =  - \frac{T-1}{2\sigma^2} + \frac{1}{2\sigma^4} \sum_{t=2}^{T}(y_t - \rho y_{t-1} - (1 - \rho) \mu)^2 \text{.}
\]

Оценки неизвестных параметров модели могут быть получены как решение следующей системы уравнений:
\[
\left\{
  \begin{aligned}
    \frac{\partial l}{\partial \mu} = 0 \text{,} \\
    \frac{\partial l}{\partial \rho} = 0 \text{,} \\
    \frac{\partial l}{\partial {\sigma^2}} = 0 \text{.}
  \end{aligned}
\right.
\]

Из первого уравнения системы получаем, что
\[
\sum_{t=2}^{T}y_{t} - \hat{\rho} \sum_{t=2}^{T}y_{t-1} = (T - 1) (1- \hat{\rho}) \hat{\mu} \text{,}
\]
откуда
\[
\hat{\mu} = \frac{\sum_{t=2}^{T}y_{t} - \hat{\rho} \sum_{t=2}^{T}y_{t-1}}{(T - 1) (1- \hat{\rho})} = \frac{3 - \hat{\rho} \cdot 3}{4\cdot(1-\hat{\rho})} = \frac{3}{4} \text{.}
\]

Далее, если второе уравнение системы переписать в виде
\[
\sum_{t=2}^{T}(y_t - \hat{\mu} - \hat{\rho} (y_{t-1} - \hat{\mu}))(y_{t-1} - \hat{\mu}) = 0 \text{,}
\]
то легко видеть, что
\[
\hat{\rho} = \frac{\sum_{t=2}^{T}(y_t - \hat{\mu})(y_{t-1} - \hat{\mu})}{\sum_{t=2}^{T}(y_{t-1} - \hat{\mu})^2} \text{.}
\]
Следовательно, $\hat{\rho} =-1/11= -0.0909$.

Наконец, из третьего уравнения системы
\[
\hs^2 =\frac{1}{T-1} \sum_{t=2}^{T}(y_t - \hat{\rho} y_{t-1} - (1 - \hat{\rho}) \hat{\mu})^2 \text{.}
\]
Значит, $\hs^2 = 165/242= 0.6818$. Ответы: $\hat{\mu} = 3/4= 0.75$, $\hat{\rho} = -1/11=-0.0909$, $\hs^2 =165/242=0.6818$.
\end{solution}
\protect \hypertarget {soln:1.5}{}
\begin{solution}{{1.5}}
Несмещёнными остаются. Состоятельными не всегда остаются, например, состоятельность исчезает, если все случайные ошибки тождественно равны между собой.

\end{solution}
\protect \hypertarget {soln:1.6}{}
\begin{solution}{{1.6}}
\end{solution}
\protect \hypertarget {soln:1.7}{}
\begin{solution}{{1.7}}
\end{solution}
\protect \hypertarget {soln:1.8}{}
\begin{solution}{{1.8}}
\end{solution}
\protect \hypertarget {soln:1.9}{}
\begin{solution}{{1.9}}
\begin{enumerate}
\item $\E(\e_t)=0$, $\Var(\e_t)=\sigma^2/(1-\rho^2)$
\item $\Cov(\e_t,\e_{t+h})=\rho^h\cdot \sigma^2/(1-\rho^2)$
\item $\Corr(\e_t,\e_{t+h})=\rho^h$
\end{enumerate}
\end{solution}
\protect \hypertarget {soln:1.10}{}
\begin{solution}{{1.10}}
\end{solution}
\protect \hypertarget {soln:1.11}{}
\begin{solution}{{1.11}}
\end{solution}
\protect \hypertarget {soln:1.12}{}
\begin{solution}{{1.12}}
\end{solution}
\protect \hypertarget {soln:1.13}{}
\begin{solution}{{1.13}}
\end{solution}
\protect \hypertarget {soln:1.14}{}
\begin{solution}{{1.14}}
\end{solution}
\protect \hypertarget {soln:1.15}{}
\begin{solution}{{1.15}}
\end{solution}



% в биб-файл
%---------------------------------------------------------------------------------------------------------------------------------------------------------

\chapter{Источники мудрости}
\printbibliography[heading=none]

\printindex

\chapter*{Список обозначений}


\end{document}
