%%%%%%%%%%%%%%%%% стационарность

\problem{
Пусть $Y_{t}$ - стационарный процесс. \\
Верно ли, что стационарны: \\
а) $Z_{t}=2Y_{t}$ \\
б) $Z_{t}=Y_{t}+1$ \\
в) $Z_{t}=\Delta Y_{t}$ \\
г) $Z_{t}=2Y_{t}+3Y_{t-1}$ }
\solution{а, б, в, г - стационарны}




\problem{
Известно, что временной ряд $Y_{t}$ порожден стационарным процессом, задаваемым соотношением $Y_{t}=1+0.5Y_{t-1}+\varepsilon_{t}$. Имеется 1000 наблюдений. \\
Вася построил регрессию $Y_{t}$ на константу и $Y_{t-1}$. Петя построил регрессию на константу и $Y_{t+1}$.\\
Как (примерно) будут соотносится между собой их оценки коэффициентов? }
\solution{Они будут примерно одинаковы. Оценка наклона определяется автоковариационной функцией.}


%%%%%%%%%%%%%%%%%% ARCH-GARCH


\problem{
Рассмотрим следующий AR(1)-ARCH(1) процесс: \\
$Y_{t}=1+0.5Y_{t-1}+\varepsilon_{t}$, $\varepsilon_{t}=\nu_{t}\cdot \sigma_{t}$ \\
$\nu_{t}$ независимые $N(0;1)$ величины. \\
$\sigma^{2}_{t}=1+0.8\varepsilon^{2}_{t-1}$\\
Также известно, что $Y_{100}=2$, $Y_{99}=1.7$ \\
а) Найдите $E_{100}(\varepsilon^{2}_{101})$, $E_{100}(\varepsilon^{2}_{102})$, $E_{100}(\varepsilon^{2}_{103})$, $E(\varepsilon^{2}_{t})$ \\
б) $Var(Y_{t})$, $Var(Y_{t}|\mathcal{F}_{t-1})$ \\
в) Постройте доверительный интервал для $Y_{101}$: \\
- проигнорировав условную гетероскедастичность \\
- учтя условную гетерескедастичность }
\solution{}


\problem{
Рассмотрим GARCH(1,1) процесс \\
...}
\solution{}



%%%%%%%%%%%%%%%%% Оператор лага

\problem{
Пусть $X_{t}$, $t=0,1,2,...$ - случайный процесс и $Y_{t}=(1+L)^{t}X_{t}$.
Выразите $X_{t}$ с помощью $Y_{t}$ и оператора лага $L$.}
\solution{$X_{t}=(1-L)^{t}Y_{t}$}

\problem{Пусть $ F_{n} $ - последовательность чисел Фибоначчи. Упростите величину
\[ F_{1}+C^{1}_{5}F_{2}+C^{2}_{5}F_{3}+C^{3}_{5}F_{4}+C^{4}_{5}F_{5}+C^{5}_{5}F_{6} \]}
\solution{$ F_{n}=L(1+L)F_{n} $, значит $ F_{n}=L^{k}(1+L)^{k}F_{n} $ или $ F_{n+k}=(1+L)^{k}F_{n} $}




\problem{
Пусть $X_{t}$, $t=...-2,-1,0,1,2,...$ - случайный процесс. И $Y_{t}=X_{-t}$. Какое рассуждение верно?

а) $LY_{t}=LX_{-t}=X_{-t-1}$

б) $LY_{t}=Y_{t-1}=X_{-t+1}$ }
\solution{ а - неверно, б - верно. }


%%%%%%%%%%%%%%%% состояние-наблюдение, фильтр Калмана

\problem{Представьте процесс AR(1),
$y_{t}=0.9y_{t-1}-0.2y_{t-2}+\varepsilon_{t}$,
$\varepsilon\sim$WN(0;1) в виде модели состояние-наблюдение. \\
а) Выбрав в качестве состояний вектор $\left(%
\begin{array}{c}
  y_{t} \\
  y_{t-1} \\
\end{array}%
\right)$ \\
б) Выбрав в качестве состояний вектор $\left(%
\begin{array}{c}
  y_{t} \\
  \hat{y}_{t,1} \\
\end{array}%
\right)$ \\
Найдите дисперсии ошибок состояний }
\solution{}

\problem{Представьте процесс MA(1),
$y_{t}=\varepsilon_{t}+0.5\varepsilon_{t-1}$,
$\varepsilon\sim$WN(0;1) в виде модели состояние-наблюдение. \\
a) $\left(%
\begin{array}{c}
  \varepsilon_{t} \\
  \varepsilon_{t-1} \\
\end{array}%
\right)$ \\
b) $\left(%
\begin{array}{c}
  \varepsilon_{t}+0.5\varepsilon_{t-1} \\
  0.5\varepsilon_{t} 
\end{array}%
\right)$ }
\solution{}

\problem{Представьте процесс ARMA(1,1),
$y_{t}=0.5y_{t-1}+\varepsilon_{t}+\varepsilon_{t-1}$,
$\varepsilon\sim$WN(0;1) в
виде модели состояние-наблюдение. \\
Вектор состояний имеет вид $x_{t},x_{t-1}$, где
$x_{t}=\frac{1}{1-0.5L}\varepsilon_{t}$ }
\solution{}


\problem{ Рекурсивные коэффициенты 
\begin{enumerate}
\item Oцените модель вида $y_{t}=a+b_{t}x_{t}+\varepsilon_{t}$,
где $b_{t}=b_{t-1}$. 
\item Сравните графики filtered state и smoothed state. 
\item Сравните финальное состояние $b_{T}$ с коэффициентом в
обычной модели линейной регрессии, $y_{t}=a+bx_{t}+\varepsilon_{t}$. 
\end{enumerate} }
\solution{}



