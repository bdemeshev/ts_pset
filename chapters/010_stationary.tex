% !TEX root = ../ts_pset_main.tex


\chapter{Стационарные процессы}

Сюда относятся задачи на стационарность до явного упоминания ARMA/ARIMA :)

\begin{problem}
Запишите процесс $y_t = 4 + 0.4y_{t-1} + 0.3\e_{t-1} + \e_t$ с помощью оператора лага.
\begin{sol}
\[
(1 - 0.4L)y_t = 4 + (1 + 0.3L)\e_t
\]
\end{sol}
\end{problem}


\begin{problem}
Пусть $x_{t}$, $t=0,1,2, \ldots$ — случайный процесс и $y_{t}=(1+L)^{t}x_{t}$.
Выразите $x_{t}$ с помощью $y_{t}$ и оператора лага $L$.

\begin{sol}
$x_{t}=(1-L)^{t}y_{t}$
\end{sol}
\end{problem}

\begin{problem}
Пусть $ F_{n} $ — последовательность чисел Фибоначчи. Рассмотрим величину
\[
\frac{F_{101}+C^{1}_{5}F_{102}+C^{2}_{5}F_{103}+C^{3}_{5}F_{104}+C^{4}_{5}F_{105}+C^{5}_{5}F_{106}}
{F_{111}}
\]
\begin{enumerate}
\item Запишите величину с помощью оператора лага
\item Упростите величину
\end{enumerate}

\begin{sol}
$ F_{n}=L(1+L)F_{n} $, значит $ F_{n}=L^{k}(1+L)^{k}F_{n} $ или $ F_{n+k}=(1+L)^{k}F_{n} $

Ответ: $1$
\end{sol}
\end{problem}




\begin{problem}
Пусть $x_{t}$, $t=\ldots -2,-1,0,1,2,\ldots $ — случайный процесс. И $y_{t}=x_{-t}$. Какое рассуждение верно?

\begin{enumerate}
\item $Ly_{t}=Lx_{-t}=x_{-t-1}$;
\item $Ly_{t}=y_{t-1}=x_{-t+1}$;
\item $x_t L y_t = x_t y_{t-1}$;
\item $x_t L y_t = x_{t-1} y_t$;
\end{enumerate}
\begin{sol}
а — неверно, б — верно, в — верно, г — нет.
\end{sol}
\end{problem}

%%%%%%%%%%%%%%%%% стационарность

\begin{problem}
Пусть $y_{t}$ — стационарный процесс. Верно ли, что стационарны:
\begin{enumerate}
\item $z_{t}=2y_{t}$
\item $z_{t}=y_{t}+1$
\item $z_{t}=\Delta y_{t}$
\item $z_{t}=2y_{t}+3y_{t-1}$
\end{enumerate}
\begin{sol}
а, б, в, г — стационарны
\end{sol}
\end{problem}





\begin{problem}
Известно, что временной ряд $y_{t}$ порожден стационарным процессом, задаваемым соотношением $y_{t}=1+0.5y_{t-1}+\varepsilon_{t}$. Имеется 1000 наблюдений.


Вася построил регрессию $y_{t}$ на константу и $y_{t-1}$. Петя построил регрессию на константу и $y_{t+1}$.


Как примерно будут соотносится между собой их оценки коэффициентов?
\begin{sol}
Они будут примерно одинаковы. Оценка наклона определяется автоковариационной функцией.
\end{sol}
\end{problem}


\begin{problem}
Правильный кубик подбрасывают три раза, обозначим результаты подбрасываний $X_1$, $X_2$ и $X_3$. Также ввёдем обозначения для сумм $L=X_1+X_2$, $R=X_2+X_3$ и $S=X_1+X_2+X_3$.
\begin{enumerate}
\item Интуитивно, без вычислений, определите знак обычных и частных корреляций $\Corr(L, R)$, $\Corr(L, S)$, $\pCorr(L, R; S)$,
  $\pCorr(L, S; R)$, $\Corr(X_1, R)$, $\pCorr(X_1, R; S)$, $\pCorr(X_1, R; L)$, $\pCorr(L, R; X_2)$, $\pCorr(L, R; X_1)$;
\item Какие из корреляций по модулю равны единице?
\item Найдите все упомянутые обычные и частные корреляции.
\end{enumerate}
\begin{sol}

\end{sol}
\end{problem}


\begin{problem}
Известно, что $\e_t$ — белый шум. У каких разностных уравнений есть слабо стационарные решения?
\begin{enumerate}
\item $y_t=1+\e_t+0.5\e_{t-1}+0.25\e_{t-2}$
\item $y_t=-2y_{t-1}-3y_{t-2}+\e_t+\e_{t-1}$
\item $y_t=-0.5y_{t-1} + \e_t$
\item $y_t=1-1.5 y_{t-1} - 0.5 y_{t-2} + \e_t - 1.5\e_{t-1} - 0.5\e_{t-2}$
\item $y_t=1+0.64y_{t-2}+\e_t+0.64\e_{t-1}$
\item $y_t=1+t+\e_t$
\item $y_t=1+y_{t-1}+\e_t$
\end{enumerate}
\begin{sol}

\begin{enumerate}
\item $y_t=1+\e_t+0.5\e_{t-1}+0.25\e_{t-2}$ — стационарный
\item $y_t=-2y_{t-1}-3y_{t-2}+\e_t+\e_{t-1}$
\item $y_t=-0.5y_{t-1} + \e_t$ — стационарный
\item $y_t=1-1.5 y_{t-1} - 0.5 y_{t-2} + \e_t - 1.5\e_{t-1} - 0.5\e_{t-2}$
\item $y_t=1+0.64y_{t-2}+\e_t+0.64\e_{t-1}$ — стационарный
\item $y_t=1+t+\e_t$ — нестационарный
\item $y_t=1+y_{t-1}+\e_t$ — нестационарный
\end{enumerate}
\end{sol}
\end{problem}







\begin{problem}
Пусть $\e_t$ — белый шум. Рассмотрим процесс $y_t=2+0.5y_{t-1}+\e_t$ с различными начальными условиями, указанными ниже.

\begin{enumerate}
\item Найдите $\E(y_t)$, $\Var(y_t)$ и определите, является ли процесс  стационарным, если:
\begin{enumerate}
\item $y_1=0$
\item $y_1=4$
\item $y_1=4+\e_1$
\item $y_1=4+\frac{2}{\sqrt{3}}\e_1$
\end{enumerate}
\item Как точно следует понимать фразу «процесс $y_t=2+0.5y_{t-1}+\e_t$ является стационарным»?
\end{enumerate}




\begin{sol}
Процесс стационарен только при $y_1=4+\frac{2}{\sqrt{3}}\e_1$. Фразу нужно понимать как «у стохастического разностного уравнения $y_t=2+0.5y_{t-1}+\e_t$ есть стационарное решение».
\end{sol}
\end{problem}




\begin{problem}
Верно ли, что при удалении из стационарного ряда каждого второго наблюдения получается стационарный ряд?


\begin{sol}
Пусть $y_t, y_{t-1}, y_{t-2} \dots $ - стационарный временной ряд. Тогда 
$
\E(y_t) = \text{const}, \Cov(y_t, y_{t-k}) = \gamma_k,  \forall t,k
$. Рассмотрим теперь ряд $y_t$, из которого удалили каждое второго наблюдение: 
\[
y_t, y_{t-1}, y_{t-2}, y_{t-3}, y_{t-4}, \dots \rightarrow y_t, y_{t-2}, y_{t-4}, y_{t-6}, y_{t-8}, \dots \rightarrow \hat{y}_t, \hat{y}_{t-1}, \hat{y}_{t-2}, \hat{y}_{t-3}, \hat{y}_{t-4}, \dots
\]
Проверим условия стационарности для нового ряда $\hat{y}_t$:
\[
\E(\hat{y}_t) = \E(y_t) = \E(y_{t-2}) = \E(\hat{y}_{t-1})) = \E(y_{t-4}) = \E(\hat{y}_{t-2})) = \dots = \E(\hat{y}_k) = \text{const} 
\]
\[
\Cov(\hat{y}_t, \hat{y}_{t-k}) = \Cov(y_{2t}, y_{2t-2k}) = \gamma_{2k} 
\]
Все предпосылки для стационарности выполнены - $\hat{y}_t$ стационарен.
\end{sol}
\end{problem}



\begin{problem}
У эконометрессы Ефросиньи был стационарный ряд. Ей было скучно и она подбрасывала неправильную монетку, выпадающую орлом с вероятностью $0.7$. Если выпадал орёл, она оставляла очередной $y_t$, если решка — то зачёркивала. Получается ли у Ефросиньи стационарный ряд?


\begin{sol}
Пусть $y_t, y_{t-1}, y_{t-2} \dots $ имеющийся у Ефросиньи стационарный временной ряд, а $\hat{y}_t, \hat{y}_{t-1}, \hat{y}_{t-2}, \dots$ это временной ряд, получаемый при зачеркивании. Пусть $\hat{y}_{t-1}$ получилась путем выпадения орла на $y_k-1$. Тогда $\hat{y}_{t}$ равно $y_k$ с вероятностью 0.7 (вероятность того, что выпадет орел),
равно $y_{k+1}$ с вероятностью $0.3\cdot 0.7$ (на $y_k$ выпадет решка, а на $y_{k+1}$ выпадет орел), ..., равно $y_{k+n}$ с вероятностью $0.3^{n-1}\cdot 0.7$ (на $y_k$...$y_{k+n-1}$ выпадет решка, а на $y_{k+n}$ выпадет орел) и т.д. Обозначим событие "выпало $n$ решек и 1 орел" за $I_n$. Тогда:
\begin{align*} 

\E(\hat{y}_t) = \E(y_k \cdot I_{0} + y_{k+1} \cdot I_{1} + \dots y_{k+n} \cdot I_{n} + \dots) 
= \E(y_k \cdot I_{0}) + \E(y_{k+1} \cdot I_{1}) + \dots + \E(y_{k+n} \cdot I_{n}) + \dots = \\
= \E(y_k) \cdot p_{I_{0}} + \E(y_{k+1}) \cdot p_{I_{1}} + \dots + \E(y_{k+n}) \cdot p_{I_{n}} + \dots = \E(y_k) \cdot (1-q)^0\cdot q + \E(y_{k+1}) \cdot (1-q)^1\cdot q + \dots  + \E(y_{k+n}) \cdot (1-q)^n \cdot q + \dots = \E(y_k)\sum_{i=0} (1-q)^i\cdot q =  \E(y_k) \cdot \frac{q}{1 - (1 - q)} = E(y_k) = \text{const}
\end{align*}
\begin{align*} 

\Cov(\hat{y_t}; \hat{y}_{t+j}) = \E(\hat{y_t}\cdot\hat{y}_{t+j}) - \E(\hat{y_t})\E(\hat{y}_{t+j}) = E(\hat{y_t}\cdot\hat{y}_{t+j}) - \E(y_k)^2 =
0\cdot\E(y_k\cdoty_k) + \dots + 0\cdot\E(y_k\cdoty_{k+j-1}) + q^j \E(y_k\cdoty_{k+j}) +
+ \dots + (1-q)^{h-1}q\cdot C_{i-1}^{j-1}q^{j-1}(1-q)^{i-j}q \E(y_{k+h}\cdot y_{k+h+i}) + \dots - \E(y_k)^2= \sum_{h > 0, i > j-1} (1-q)^{h-1 + i -j} C_{i-1}^{j-1}q^{j+1}\E(y_{k+h}\cdot y_{k+h+i}) - \E(y_k)^2
 = \sum_{h > 0, i > j-1} (1-q)^{h-1 + i -j} C_{i-1}^{j-1}q^{j+1}(\gamma_{i} + \E(y_{k})^2) - \E(y_k)^2 = \sum_{i>j-1} (1-q)^{i-j-1}q^{j+1}C_{i-1}^{j-1}\gamma_i \sum_{h>0} (1-q)^h + \E(y_k)^2\sum_{i>j-1} (1-q)^{i-j-1}q^{j+1}C_{i-1}^{j-1} - \E(y_k)^2 
\end{align*}
Использовали:
$
\E(y_{k+h}*y_{k+h+i}) = \Cov(y_{k+h};y_{k+h+i}) + \E(y_{k+h})^2 = \gamma_{i} + \E(y_{k})^2 
$

Заметим, что нам удалось расписать ковариацию как выражение, зависящее только от расстояния между наблюдениями. Итого математическое ожидание константно, а ковариации зависят только от расстояния между наблюдениями - ряд стационарен.
\end{sol}
\end{problem}


\begin{problem}
Имеется временной ряд, $\e_1$, $\e_2$, \ldots, $\e_{101}$. Величины $\e_t$ нормально распределены, $N(0,\sigma^2)$, и независимы. Построим график этого процесса.
\begin{enumerate}
\item Является ли этот процесс белым шумом?
\item Сколько в среднем раз график пересекает ось абсцисс?
\item Оцените вероятность того, что график пересечет ось абсцисс более 60 раз.
\end{enumerate}

\begin{sol}
да, это белый шум. Величина $N$ распределена биномиально, $Bin(n=100,p=1/2)$, $\E(N)=50$.
\end{sol}
\end{problem}


\begin{problem}
Величины $x_t$ независимы и равновероятно принимают значения $0$ и $1$. Величины $y_t$ независимы и нормальны $\cN(0;24)$.
Процессы $(x_t)$ и $(y_t)$ независимы. Для каждого из пунктов ответьте на три вопроса. Верно ли, что величины $z_t$ одинаково распределены? Верно ли, что они независимы? Верно ли, что процесс $(z_t)$ — белый шум?
\begin{enumerate}
  \item $z_t = x_t (1-x_{t-1})y_t$;
\item $z_t = y_{t-1}y_t$;
\end{enumerate}
\begin{sol}

\begin{enumerate}
  \item $z_t = x_t (1-x_{t-1})y_t$;
    Процесс $z_t$ — белый шум, $\E(z_t)=0$, $\Var(z_t)=6$. Величины $z_t$ зависимы. Например, если $z_t \neq 0$, то $z_{t+1}=z_{t-1}=0$. Величины $z_t$ одинаково распределены.
\item $z_t = y_{t-1}y_t$;
Процесс $z_t$ — белый шум. Величины $z_t$ зависимы. Величины $z_t$ одинаково распределены.
\end{enumerate}


\end{sol}
\end{problem}




\begin{problem}
Величина $Z$ равновероятно принимает значения $0$ и $1$. Условное распределение вектора $X=(X_1, X_2)$ при известном $Z$ известно:
\[
  \begin{pmatrix}
    X_1 \\
    X_2 \\
  \end{pmatrix}|Z=0 \sim \cN\left(
    \begin{pmatrix}
      0 \\
      0 \\
    \end{pmatrix};
    \begin{pmatrix}
      1 & 0 \\
      0 & 1 \\
    \end{pmatrix}
    \right)
\]

\[
  \begin{pmatrix}
    X_1 \\
    X_2 \\
  \end{pmatrix}|Z=1 \sim \cN\left(
    \begin{pmatrix}
      1 \\
      1 \\
    \end{pmatrix};
    \begin{pmatrix}
      4 & -1 \\
      -1 & 9 \\
    \end{pmatrix}
    \right)
\]

Найдите
\begin{enumerate}
  \item Частную корреляцию $\pCorr(X_1, X_2; Z)$;
  \item Условную корреляцию $\Corr(X_1, X_2 | Z)$;
\end{enumerate}

  \begin{sol}
Проекции: $\tilde X_1 = X_1 + Z$; $\tilde X_2 = X_2 + Z$; $\E(X_i|Z)=1-Z$; $\Cov(X_i, Z)=-1/4$;

Величина $Z$ имеет распределение Бернулли, поэтому $\E(Z)=1/2$ и $\Var(Z)=1/4$;

\[
  \pCorr(X_1, X_2; Z) = \frac{-1/2}{12.5} = -\frac{1}{\sqrt{50}}
\]
\[
  \Corr(X_1, X_2|Z)=-Z/6
\]
  \end{sol}
\end{problem}

\begin{problem}
Приведите пример процесса каждого из четырёх типов:
\begin{enumerate}
  \item Слабостационарный и одновременно сильностационарный;
  \item Слабостационарный но не сильностационарный;
  \item Сильностационарный но не слабостационарный;
  \item Не сильностационарный и не слабостационарный.
\end{enumerate}
\begin{sol}
\begin{enumerate}
  \item $u_t \sim \cN(0;1)$ и независимы;
  \item $u_t \sim \cN(0;1)$ и независимы при $t>1$, а при $t=1$ величина $u_t$ равновероятно
  принимает значения $-1$ и $1$;
  \item Величины $u_t$ независимы и одинаково распределены с бесконечным математическим ожиданием;
  \item $u_t \sim \cN(t;1)$ и независимы.
\end{enumerate}
\end{sol}
\end{problem}



\begin{problem}
Процесс $(u_t)$ — белый шум. Величины $u_t$ одинаково непрерывно распределены.

Назовём момент времени $t$ — поворотной точкой (turning point), если он является локальным пиком, больше обоих своих соседей 
или локальной ямой, меньше обоих своих соседей.

Рассмотрим процесс $z_t$ — индикаторы того, что точка $t$ является поворотной. 
Процесс $s_t = z_2 + \ldots + z_{t-1}$ — считает количество поворотных точек за период от $1$ до $t$.
Величины $z_1$ и $z_t$ в сумму не входят, так как мы не считаем края наблюдаемого отрезка поворотными точками. 


\begin{enumerate}
  \item Найдите вероятность $\P(z_t = 1)$;
  \item Найдите $\E(s_t)$;
  \item Найдите $\Cov(z_1, z_2)$, $\Cov(z_1, z_3)$, $\Cov(z_1, z_4)$;
  \item Найдите $\Var(s_t)$;
\end{enumerate}
\begin{sol}
  \begin{enumerate}
  \item $\P(z_t = 1) = 2/3$;
  \item $\E(s_t) = (t-2) \cdot 2/3$;
  \item $\Cov(z_1, z_2)$, $\Cov(z_1, z_3)$, $\Cov(z_1, z_4) = 0$;
  \item $\Var(s_t) = (16t - 29)/90$;
  \end{enumerate}

\end{sol}
\end{problem}
  


\begin{problem}
  Процессы $(a_t)$ и $(b_t)$ — стационарны. 
  Кроме того, $\Corr(a_t, b_t) = 0$ для любого момента времени $t$. 
  
  Рассмотрим произведение этих процессов $y_t = a_t b_t$ и сумму $x_t = a_t b_t$.

  Предположим, что все необходимые ожидания и ковариации существуют. 

  \begin{enumerate}
    \item Верно ли, что процесс $(x_t)$ — стационарный? Докажите, или приведите контр-пример.
    \item Верно ли, что процесс $(y_t)$ — стационарный? Докажите, или приведите контр-пример.
    \item Как изменятся ответы на предыдущие пункты, если $\Corr(a_t, b_s) = 0$ для любых моментов времени $t$ и $s$.
  \end{enumerate}
\begin{sol}
\begin{enumerate}
  \item Процесс $(x_t)$ не обязательно стационарен;
\item Процесс $(y_t)$ не обязательно стационарен;
\item Если любые корреляции равны нулю, то процесс-сумма будет стационарным, а процесс-произведение — не обязательно. 
\end{enumerate}
\end{sol}
\end{problem}
