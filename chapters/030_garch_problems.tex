% !TEX root = ../ts_pset_main.tex


\chapter{GARCH}

Книжечка: \cite{francq2019garch}.

\input{chapters/030_garch_theory}

\begin{problem}
Рассмотрим следующий AR(1)-ARCH(1) процесс: \\
$Y_{t}=1+0.5Y_{t-1}+\varepsilon_{t}$, $\varepsilon_{t}=\nu_{t}\cdot \sigma_{t}$ \\
$\nu_{t}$ независимые $\cN(0;1)$ величины. \\
$\sigma^{2}_{t}=1+0.8\varepsilon^{2}_{t-1}$\\
Также известно, что $Y_{100}=2$, $Y_{99}=1.7$
\begin{enumerate}
\item Найдите $\E_{100}(\varepsilon^{2}_{101})$, $\E_{100}(\varepsilon^{2}_{102})$, $\E_{100}(\varepsilon^{2}_{103})$, $\E(\varepsilon^{2}_{t})$.
\item Найдите $\Var(Y_{t})$, $\Var(Y_{t}|\cF_{t-1})$.
\item Постройте доверительный интервал для $Y_{101}$ при известном $Y_{100}$ и $Y_{99}$.
\end{enumerate}
\begin{sol}
\end{sol}
\end{problem}




\begin{problem}
Рассмотрим GARCH(1,2) процесс $\e_t=\sigma_t \nu_t$, $\sigma^2=0.2+0.5\sigma^2_{t-1}+0.2\e_{t-1}^2+0.1\e_{t-2}^2$. 
Найдите безусловную дисперсию $\Var(y_t)$.
\begin{sol}

\end{sol}
\end{problem}


\begin{problem}
Для GARCH(1,1) процесса $\e_t=\sigma_t \nu_t$, $\sigma^2_{t}= w + \alpha \e_{t-1}^2+ \beta \sigma^2_{t-1}$ найдите $\E( \E(\e_t^2 | \cF_{t-1} ) )$
\begin{sol}

\end{sol}
\end{problem}



\begin{problem}
Рассмотрим GARCH(1,1) процесс  $\e_t=\sigma_t \nu_t$, $\sigma^2_{t}=0.1 + 0.7 \sigma^2_{t-1} + 0.2 \e_{t-1}^2$. Известно, $\sigma_T=1$, $\e_T=1$. Найдите $\E(\sigma_{T+2}^2 | \cF_T )$.
\begin{sol}

\end{sol}
\end{problem}


\begin{problem}
Найдите безусловную дисперсию GARCH-процессов
\begin{enumerate}
\item $\e_t=\sigma_t \cdot z_t$, $\sigma^2_t=0.1+0.8\sigma^2_{t-1}+0.1\e^2_{t-1}$
\item $\e_t=\sigma_t \cdot z_t$, $\sigma^2_t=0.4+0.7\sigma^2_{t-1}+0.1\e^2_{t-1}$
\item $\e_t=\sigma_t \cdot z_t$, $\sigma^2_t=0.2+0.8\sigma^2_{t-1}+0.1\e^2_{t-1}$
\end{enumerate}


\begin{sol}
$1$, $2$, $2$
\end{sol}
\end{problem}


\begin{problem}
Являются ли верными следующие утверждения?
\begin{enumerate}
\item GARCH-процесс является процессом белого шума, условная дисперсия которого
изменяется во времени
\item Модель GARCH(1,1) предназначена для прогнозирования меры изменчивости цены
финансового инструмента, а не для прогнозирования самой цены инструмента
\item При помощи GARCH-процесса можно устранять гетероскедастичность
\item Безусловная дисперсия GARCH-процесса изменяется во времени
\item Модель GARCH(1,1) может быть использована для прогнозирования
волатильности финансовых инструментов на несколько торговых недель вперёд
\end{enumerate}


\begin{sol}
\end{sol}
\end{problem}



\begin{problem}
Рассмотрим GARCH-процесс $\e_t=\sigma_t \cdot z_t$, $\sigma^2_t=k+g_1\sigma^2_{t-1}+a_1\e^2_{t-1}$. Найдите
\begin{enumerate}
\item $\E(z_t)$, $\E(z_t^2)$, $\E(\e_t)$, $\E(\e_t^2)$
\item $\Var(z_t)$, $\Var(\e_t)$, $\Var(\e_t \mid \cF_{t-1})$
\item $\E(\e_t \mid \cF_{t-1})$, $\E(\e_t^2 \mid \cF_{t-1})$, $\E(\sigma^2_t \mid \cF_{t-1})$
\item $\E(z_tz_{t-1})$, $\E(z_t^2z_{t-1}^2)$, $\Cov(\e_t,\e_{t-1})$, $\Cov(\e_t^2,\e_{t-1}^2)$
\item $\lim_{h\to\infty} \E(\sigma^2_{t+h} \mid \cF_t)$
\end{enumerate}


\begin{sol}
\end{sol}
\end{problem}



\begin{problem}
Используя 500 наблюдений дневных логарифмических доходностей $y_t$ ,
была оценена GARCH(1,1)-модель: $\hy_t=-0.000708+\he_t$, $\e_t=\sigma_t \cdot z_t$, $\sigma^2_t=0.000455+0.6424\sigma^2_{t-1}+0.2509\e^2_{t-1}$. Также известно, что $\hs^2_{499}=0.002568$, $\he^2_{499}=0.000014$, $\he^2_{500}=0.002178$.
Найдите
\begin{enumerate}
\item  $\hs^2_{500}$, $\hs^2_{501}$, $\hs^2_{502}$
\item Волатильность в годовом выражении в процентах, соответствующую
наблюдению с номером $t = 500$
\end{enumerate}


\begin{sol}
\end{sol}
\end{problem}


\begin{problem}
Рассмотрим ARCH(1) процесс
\[
\begin{cases}
y_t = 2 + \e_t \\
\e_t = \sigma_t \cdot \nu_t \\
\sigma^2_t = 10 + 0.5\e_{t-1}^2
\end{cases}
\]

\begin{enumerate}
\item Найдите $\Var(y_{101})$, постройте 95\%-ый предиктивный интервал для $y_{101}$
\item Известно, что $y_{100}=3$, постройте 95\%-ый предиктивный интервал для $y_{101}$
\item Известно, что $y_{100}=12$, постройте 95\%-ый предиктивный интервал для $y_{101}$
\end{enumerate}


\begin{sol}
\end{sol}
\end{problem}

\begin{problem}
Может ли у GARCH процесса условная дисперсия $\e_t$ быть больше, чем безусловная? А меньше, чем безусловная?
\begin{sol}
Да, может быть и больше, и меньше.
\end{sol}
\end{problem}

\begin{problem}
Как известно, у GARCH процесса условная дисперсия $\e_t$ может быть как больше, так и меньше безусловной.
\begin{enumerate}
\item Имеет ли смысл строить предиктивный интервал для $y_t$, используя условную дисперсию, если она больше безусловной?
\item При построении предиктивного интервала эконометресса Агнесса использует безусловную дисперсию, если она меньше условной, и условную дисперсию, если она меньше безусловной. Корректно ли поступает Агнесса?
\end{enumerate}
\begin{sol}

\end{sol}
\end{problem}



\begin{problem}
Рассмотрим процесс AR(1)-GARCH(1,1):
\[
\begin{cases}
y_t = 2 + 0.6 y_{t-1} + \e_t \\
\e_t = \sigma_t \cdot \nu_t \\
\sigma^2_t = 6 + 0.4 \sigma^2_{t-1} + 0.2\e_t^2
\end{cases}
\]


Найдите $\Var(\e_t | \cF_{t-1})$, $\Var(y_t| \cF_{t-1})$, $\Var(\e_t)$, $\Var(y_t)$


\begin{sol}
\[
\Var(\e_t | \cF_{t-1}) = \Var(y_t| \cF_{t-1}) = 6 + 0.4 \sigma^2_{t-1} + 0.2\e_t^2
\]
\[
\Var(\e_t) = 6/(1-0.4-0.2)=6/0.4=15
\]
\[
\Var(y_t)=15/(1-0.36)
\]
\end{sol}
\end{problem}


