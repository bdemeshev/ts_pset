% !TEX root = ../ts_pset_main.tex


\chapter{ETS}

Почитать про ETS модели в книжке \cite{hyndman2018forecasting}.

\begin{problem}
  Рассмотрим ETS-ANN модель с $\alpha = 1/2$, $y_1=6$, $y_2=9$, $y_3 = 6$, $\sigma^2=9$.


  \begin{enumerate}
    \item Найдите величину $\ell_0$, которая минимизирует $RSS$;
    \item Постройте точечный прогноз $\hat y_{4|2}$, $\hat y_{5|2}$;
     \item Постройте 95\%-ый предиктивный интервал для $y_{4}$ и $y_{5}$.
  \end{enumerate}
\begin{sol}
  \[
    \hat y_{4|3} = \ell_3 
  \]
  \[
    y_4 - \hat y_{4|3} = \ell_3 + \varepsilon_4 - \ell_3 = \varepsilon_4  
  \]
  \[
  \Var(y_4 - \hat y_{4|3} \mid \cF_3) = \Var(\varepsilon_4 \mid \cF_3) = \Var(\varepsilon_4)  
  \]
  \[
  \hat y_{5|3} = \ell_3 
  \]
  \begin{multline}
  y_5 - \hat y_{5|3} = \ell_4  + \varepsilon_5 - \ell_3  = (\ell_3 + \alpha \varepsilon_4)  + 
   \varepsilon_5 - \ell_3 = \\
  = \varepsilon_5 + \alpha  \varepsilon_4 
  \end{multline}
  \[
  \Var(y_5 - \hat y_{5|3} \mid \cF_3) = \Var(\varepsilon_5 + \alpha  \varepsilon_4 )  
  \]
    
\end{sol}
\end{problem}

\begin{problem}
  Рассмотрим ETS-AAN модель с $\alpha = 1/2$, $\beta=3/4$, $\ell_{0}=7$, $b_0=2$, $y_1=6$, $y_2=9$, $y_3=3$, $\sigma^2=9$.
  \begin{enumerate}
    \item Постройте точечный прогноз $\hat y_{4|3}$, $\hat y_{5|3}$;
    \item Постройте 95\%-ый предиктивный интервал для $y_{4}$ и $y_{5}$.
  \end{enumerate}
\begin{sol}
\[
\hat y_{4|3} = \ell_3 + b_3
\]
\[
y_4 - \hat y_{4|3} = \ell_3 + b_3 + \varepsilon_4 - (\ell_3 + b_3) = \varepsilon_4  
\]
\[
\Var(y_4 - \hat y_{4|3} \mid \cF_3) = \Var(\varepsilon_4 \mid \cF_3) = \Var(\varepsilon_4)  
\]
\[
\hat y_{5|3} = \ell_3 + 2b_3
\]
\begin{multline}
y_5 - \hat y_{5|3} = \ell_4 + b_4 + \varepsilon_5 - (\ell_3 + 2b_3) = (\ell_3 + b_3 + \alpha \varepsilon_4)  + 
(b_3 + \beta \varepsilon_4) + \varepsilon_5 - (\ell_3 + 2b_3) = \\
= \varepsilon_5 + (\alpha + \beta) \varepsilon_4 
\end{multline}
\[
\Var(y_5 - \hat y_{5|3} \mid \cF_3) = \Var(\varepsilon_5 + (\alpha + \beta) \varepsilon_4 )  
\]
\end{sol}
\end{problem}

\begin{problem}
  Рассмотрим ETS-AAN модель с $\alpha = 1/2$, $\beta=3/4$, $\ell_{0}=7$, $y_1=6$, $y_2=9$, $\sigma^2=16$.


  \begin{enumerate}
    \item Найдите величину $b_0$, которая минимизирует $RSS$;
    \item Постройте точечный прогноз $\hat y_{3|2}$, $\hat y_{4|2}$;
     \item Постройте 95\%-ый предиктивный интервал для $y_{3}$ и $y_{4}$.
  \end{enumerate}
\begin{sol}
\end{sol}
\end{problem}

\begin{problem}
  Рассмотрим ETS-AAN модель с $\alpha = 1/2$, $\beta=3/4$, $\ell_{0}=7$, $y_1=6$, $y_2=9$, $y_3=3$.

  Выпишите сумму квадратов ошибок прогнозов на один шаг вперёд через $b_0$.
  % минимизация тут длинная
\begin{sol}
\end{sol}
\end{problem}



\begin{problem}
  Рассмотрим ETS-AAN модель с $\alpha = 1/2$, $\beta=3/4$, $\ell_{99}=8$, $b_{99}=1$, $y_{99}=10$, $y_{100}=8$, $\sigma^2=16$.
  \begin{enumerate}
    \item Найдите $\ell_{100}$, $b_{100}$, $\ell_{98}$, $b_{98}$;
    \item Постройте точечный прогноз $\hat y_{101|100}$, $\hat y_{102|100}$;
    \item Постройте 95\%-ый предиктивный интервал для $y_{101}$ и $y_{102}$.
  \end{enumerate}
\begin{sol}
  Для начала запишем уравнения для ETS-AAN модели в общем виде:
  \[
    \begin{cases}
          u_t \sim \cN (0, \sigma^2), \quad iid \\
          b_t = b_{t-1} + \beta u_t \\
          \ell_t = \ell_{t-1} + b_{t-1} + \alpha u_t \\
          y_t = \ell_{t-1} + b_{t-1} + u_t
    \end{cases}
  \]
  
  Теперь подставим известные параметры и начальные значения:
  \[
      \begin{cases}
          u_t \sim \cN (0, 16)   \\
          b_t = b_{t-1} + \frac{3}{4} \cdot u_t,  b_{99}=1 \\
          \ell_t = \ell_{t-1} + b_{t-1} + \frac{1}{2} \cdot u_t, \ell_{99}=8 \\
          y_t = \ell_{t-1} + b_{t-1} + u_t,  y_{99}=10,  y_{100}=8
      \end{cases}
  \]
  
  \begin{enumerate}
    \item Пользуясь этим уравнениям, найдём искомые $\ell_{100}$, $b_{100}$, $\ell_{98}$ и $b_{98}$:
  \begin{align*}
      y_{100} = \ell_{99} + b_{99} + u_{100} \quad &\Rightarrow \quad u_{100} = y_{100} - \ell_{99} - b_{99} = 8 - 8 - 1 = -1 \\
      &\Rightarrow \quad \ell_{100} = \ell_{99} + b_{99} + \frac{1}{2} \cdot u_{100} = 8 + 1 - \frac{1}{2} = 8.5 \\
      &\Rightarrow \quad b_{100} = b_{99} + \frac{3}{4} \cdot u_{100} = 1 - \frac{3}{4} = \frac{1}{4} = 0.25\\
  \end{align*}
  \[
      \begin{cases}
          b_{99} = b_{98} + \frac{3}{4} \cdot u_{99} \\
          \ell_{99} = \ell_{98} + b_{98} + \frac{1}{2} \cdot u_{99} \\
          y_{99} = \ell_{98} + b_{98} + u_{99}
      \end{cases}
  \]
  \begin{align*}
      &\Rightarrow \quad b_{98} = b_{99} - \frac{3}{4} \cdot u_{99} \\
      &\Rightarrow \quad \ell_{99} = \ell_{98} + b_{99} - \frac{3}{4} \cdot u_{99} + \frac{1}{2} \cdot u_{99} =  \ell_{98} + b_{99} - \frac{1}{4} \cdot u_{99} \\
      &\Rightarrow \quad \ell_{98} = \ell_{99} - b_{99} + \frac{1}{4} \cdot u_{99} \\
      &\Rightarrow \quad y_{99} = \ell_{99} - b_{99} + \frac{1}{4} \cdot u_{99} + b_{99} - \frac{3}{4} \cdot u_{99} + u_{99} = \ell_{99} + \frac{1}{2} \cdot u_{99} \\
      &\Rightarrow \quad u_{99} = 2 \cdot (y_{99} - \ell_{99}) = 2 \cdot (10 - 8) = 4 \\
      &\Rightarrow \quad b_{98} = b_{99} - \frac{3}{4} \cdot u_{99} = 1 - \frac{3}{4} \cdot 4 = -2 \\
      &\Rightarrow \quad \ell_{98} = \ell_{99} - b_{99} + \frac{1}{4} \cdot u_{99} = 8 - 1 + \frac{1}{4} \cdot 4 = 8\\
  \end{align*}
  
  Итак, ответ в этом пункте:
  \[
      \ell_{100} = 8.5, \qquad b_{100} = 0.25, \qquad \ell_{98} = 8, \qquad b_{98} = -2
  \]
  
  \item  Точечный прогноз $\hat{y}_{101 \mid 100}$ равен математическому ожиданию $y_{101}$ 
  при условии всей информации $\cF_{100}$, которую мы знаем на шаге 100, а именно:
  \[
      \hat{y}_{101 \mid 100} = \E(y_{101} \mid \cF_{100}) = \E(\ell_{100} + b_{100} + u_{101}\mid \cF_{100}) = \ell_{100} + b_{100} = 8.5 + 0.25 = 8.75
  \]
  
  Аналогично найдём $\hat{y}_{102 \mid 100}$:
  \begin{align*}
      \hat{y}_{102 \mid 100} &= \E(y_{102} \mid \cF_{100}) = \E(\ell_{101} + b_{101} + u_{102}) = \E(\ell_{101}) + \E(b_{101}) = \\ 
      &= \E\left(\ell_{100} + b_{100} + \frac{1}{2} \cdot u_{101}\right) + \E\left(b_{100} + \frac{3}{4} \cdot u_{101}\right) = \\
      &= \ell_{100} + b_{100} + b_{100} = 8.5 + 0.25 + 0.25 = 9
  \end{align*}
  
  \item В общем виде 95\% предиктивные интервалы для $y_{101}$ и $y_{102}$ вычисляются по следующим формулам соответственно:
  \begin{gather*}
      y_{101} \in \left[\E(y_{101} \mid \cF_{100}) - 1.96 \sqrt{\Var(y_{101} \mid \cF_{100})},\ \  \E(y_{101} \mid \cF_{100}) + 1.96 \sqrt{\Var(y_{101} \mid \cF_{100})}\right] \\
      y_{102} \in \left[\E(y_{102} \mid \cF_{100}) - 1.96 \sqrt{\Var(y_{102} \mid \cF_{100})},\ \  \E(y_{102} \mid \cF_{100}) + 1.96 \sqrt{\Var(y_{102} \mid \cF_{100})}\right]
  \end{gather*}
  
  Значит, нам осталось найти только дисперсии $y_{101}$ и $y_{102}$ при условии всё той же информации $\cF_{100}$:
  \begin{align*}
      \Var(y_{101} \mid \cF_{100}) &= \Var(\ell_{100} + b_{100} + u_{101}) = \Var(u_{101}) = 16 \\
      \Var(y_{102} \mid \cF_{100}) &= \Var(\ell_{101} + b_{101} + u_{102}) = \Var(\ell_{100} + b_{100} + \frac{1}{2} \cdot u_{101} + b_{100} + \frac{3}{4} \cdot u_{101} + u_{102}) = \\
      &= \Var\left(\frac{5}{4} \cdot u_{101} + u_{102}\right) = \frac{25}{16} \cdot \Var(u_{101}) + \Var(u_{102}) = \frac{25}{16} \cdot 16 + 16 = 41
  \end{align*}
  
  Значит, 95\% предиктивные интервалы для $y_{101}$ и $y_{102}$ следующие:
  \[
  \begin{aligned}
      y_{101} &\in \big[8.75 - 1.96 \cdot 4; 8.75 + 1.96 \cdot 4\big] \\
      y_{102} &\in \big[9 - 1.96 \cdot \sqrt{41}; 9 + 1.96 \cdot \sqrt{41}\big]
  \end{aligned}
  \quad \Rightarrow \quad
  \begin{aligned}
      y_{101} &\in \big[0.91; 16.59\big] \\
      y_{102} &\in \big[-3.55; 21.55\big]
  \end{aligned}
  \]
\end{enumerate}
\end{sol}
\end{problem}


\begin{problem}
Для каждой из ETS моделей найдите эквивалентную модель класса ARIMA:
	\begin{enumerate}
		\item Простое экспоненциальное сглаживание, ETS-ANN;
		\item Аддитивное сглаживание Хольта, ETS-AAN;
		\item Аддитивное сглаживание Хольта с угасающим трендом, ETS-AAdN;
		\item Аддитивное сглаживание Хольта-Винтерса для месячных данных, ETS-AAA;
		\item Аддитивное сглаживание Хольта-Винтерса с угасающим трендом для месячных данных, ETS-AAdA;
		\item ETS-ANA;
	\end{enumerate}
\begin{sol}
	\begin{enumerate}
		\item Простое экспоненциальное сглаживание, ETS-ANN; ARIMA(0,1,1)
		\item Аддитивное сглаживание Хольта, ETS-AAN; ARIMA(0,2,2)
		\item Аддитивное сглаживание Хольта с угасающим трендом, ETS-AAdN; ARIMA(1,1,2)
		\item Аддитивное сглаживание Хольта-Винтерса для месячных данных, ETS-AAA; ARIMA(0,1,13)-SARIMA(0,1,0)
		\item Аддитивное сглаживание Хольта-Винтерса с угасающим трендом для месячных данных, ETS-AAdA; ARIMA(0,1,13)-SARIMA(0,1,0)
		\item ETS-ANA; ARIMA(0,1,12)-SARIMA(0,1,0)
	\end{enumerate}
\end{sol}
\end{problem}


\begin{problem}
Рассмотрим ETS-AAN модель. По каким параметрам модели оптимальные точки можно получить в явном виде?
\begin{sol}
По $\ell_0$, $b_0$;
\end{sol}
\end{problem}

\begin{problem}
Процесс $y_t$ описывается $ETS(MNM)$ моделью. 
Верно ли, что процесс $z_t = \ln y_t$ точно описывается $ETS(ANA)$ моделью? А примерно?
\begin{sol}
  Только примерно, $\ln (1 + x) \approx x$.
\end{sol}
\end{problem}
  

\begin{problem}
Рассмотрим $ETS(AA_dN)$ модель с $\phi = 0.9$, $\alpha=0.3$, $\beta=0.1$ и $\sigma^2=16$. 
Выразите 95\% предиктивный интервал для $y_{t+1}$ и $y_{t+2}$ через $\ell_t$, $b_t$, $y_t$ и $u_t$. 
\begin{sol}
\end{sol}
\end{problem}

\begin{problem}
Найдите $\E(y_t)$, $\Var(y_t)$, $\Cov(y_t, y_{t+1})$ для $ETS(AAN)$ модели с заданными $\ell_0$, $b_0$, $\alpha$, $\beta$ и $\sigma^2$.
\begin{sol}
  Выпишем модель $ETS(AAN)$ в общем виде:
  \[
      \begin{cases}
          u_t \sim \mathcal{N}(0, \sigma^2), \quad iid \\
          b_t = b_{t-1} + \beta u_t \\
          \ell_t = \ell_{t-1} + b_{t-1} + \alpha u_t \\
          y_t = \ell_{t-1} + b_{t-1} + u_t
      \end{cases}
  \]
  
  Для начала выпишем выражение для $b_t$:
  \[
      b_t = b_{t-1} + \beta u_t = b_0 + \beta(u_1 + \ldots + u_t) = b_0 + \sum_{i = 1}^t \beta u_i
  \]
  
  Из последнего уравнения модели можно видеть, что $y_t$ выражается через сумму $\ell_{t-1} + b_{t-1}$. 
  Значит, чтобы в дальнейшем посчитать требуемые $\E(y_t)$, $\Var(y_t)$, $\Cov(y_t, y_{t-1})$, нужно привести эту сумму к известным нам величинам: $\ell_0$, $b_0$, $\alpha$, $\beta$ и сумме некоторых $u_s$, для которых все эти величины мы можем найти, поскольку знаем их распределение. Докажем по индукции, что для $\ell_{t} + b_{t}$ верно равенство:
  \begin{align*}
      \ell_{t} + b_{t} 
      &= \ell_0 + (t + 1) b_0 + (\alpha + t\beta) u_1 + \ldots + (\alpha + 2\beta) u_{t-1} + (\alpha + \beta) u_t = \\
      &= \ell_0 + (t + 1) b_0 + \sum_{i = 1}^t \big(\alpha + (t - i + 1)\beta\big) u_i
  \end{align*}
  
  Шаг индукции для $t = 1$ доказывается просто:
  \[
      \ell_1 + b_1 = (\ell_0 + b_0 + \alpha u_1) + (b_0 + \beta u_1) = \ell_0 + 2 b_0 + (\alpha + \beta) u_1
  \]
  
  Теперь докажем шаг индукции: предположим, что для $t - 1$ такая формула верна, и выразим через неё аналогичную для $t$.
  \begin{align*}
  \ell_{t-1} + b_{t-1} &= \ell_0 + t b_0 + \sum_{i = 1}^{t-1} \big(\alpha + (t - i)\beta\big) u_i \\
      \ell_t + b_t &= (\ell_{t-1} + b_{t-1} + \alpha u_t) + (b_{t-1} + \beta u_t) = \big(\ell_{t-1} + b_{t-1}\big) + b_{t-1} + (\alpha + \beta) u_t = \\
      &= \Big( \ell_0 + t b_0 + \sum_{i = 1}^{t-1} \big(\alpha + (t - i)\beta\big) u_i \Big) + b_{t-1} + (\alpha + \beta) u_t = \\
      &= \Big( \ell_0 + t b_0 + \sum_{i = 1}^{t-1} \big(\alpha + (t - i)\beta\big) u_i \Big) + b_0 + \sum_{i = 1}^{t-1} \beta u_i + (\alpha + \beta) u_t = \\
      &= \ell_0 + (t + 1) b_0 + \sum_{i = 1}^{t-1} \Big(\big(\alpha + (t - i)\beta\big) u_i + \beta u_i \Big) + (\alpha + \beta) u_t = \\
      &= \ell_0 + (t + 1) b_0 + \sum_{i = 1}^{t-1} \big(\alpha + (t - i + 1)\beta\big) u_i + (\alpha + \beta) u_t = \\
      &= \ell_0 + (t + 1) b_0 + \sum_{i = 1}^{t} \big(\alpha + (t - i + 1)\beta\big) u_i \\
      &&\blacksquare
  \end{align*}
  
  Теперь с помощью этой формулы можем найти все требуемые величины.
  \[
      \E(y_t) = \E(\ell_{t-1} + b_{t-1} + u_t) = \E(\ell_{t-1} + b_{t-1}) = \E(\ell_0 + t b_0 + \sum_{i = 1}^{t-1} \big(\alpha + (t - i)\beta\big) u_i) = \E(\ell_0 + t b_0) = \ell_0 + tb_0 \\
  \]
  \begin{align*}
      \Var(y_t) &= \Var(\ell_{t-1} + b_{t-1} + u_t) + \Var(\ell_0 + t b_0 + \sum_{i = 1}^{t-1} \big(\alpha + (t - i)\beta\big) u_i + u_t) = \\
      &= \sum_{i = 1}^{t-1}\big(\alpha + (t - i)\beta\big)^2 \Var(u_i) + \Var(u_t) = \sum_{i = 1}^{t-1}\big(\alpha + (t - i)\beta\big)^2 \sigma^2 + \sigma^2 = \\
      &= \Big[k = t - i\Big]= \left( 1 + \sum_{k = 1}^{t-1}\big(\alpha + k\beta\big)^2 \right)\sigma^2 \\
  \end{align*}
  
  \begin{align*}
      \Cov(y_t, y_{t+1}) &= \Cov(\ell_{t-1} + b_{t-1} + u_t, l_{t} + b_{t} + u_{t+1}) = \\
      &= \Cov(\ell_0 + t b_0 + \sum_{i = 1}^{t-1} \big(\alpha + (t - i)\beta\big) u_i + u_t, l_0 + (t+1) b_0 + \sum_{i = 1}^{t} \big(\alpha + (t - i + 1)\beta\big) u_i + u_{t+1}) = \\
      &= \Cov(\sum_{i = 1}^{t-1} \big(\alpha + (t - i)\beta\big) u_i + u_t, \sum_{i = 1}^{t} \big(\alpha + (t - i + 1)\beta\big) u_i) = \\
      &= \sum_{i = 1}^{t-1} \big(\alpha + (t - i)\beta\big)\big(\alpha + (t - i + 1)\beta\big) \sigma^2 + (\alpha + \beta) \sigma^2 = \\
      &= \left( (\alpha + \beta) + \sum_{i = 1}^{t-1} \big(\alpha + (t - i)\beta\big)\big(\alpha + (t - i + 1)\beta\big) \right)\sigma^2 = \\
      &= \Big[k = t - i\Big]= \left( (\alpha + \beta) + \sum_{k = 1}^{t-1} \big(\alpha + k\beta\big)\big(\alpha + (k + 1)\beta\big) \right)\sigma^2
  \end{align*}
\end{sol}
\end{problem}


\begin{problem}
Полугодовой $y_t$ моделируется с помощью $ETS(AAA)$ процесса:
    
\[
\begin{cases}
    u_t \sim \cN(0; 4) \\
    s_t = s_{t-2} + 0.1 u_t \\
    b_t = b_{t-1} + 0.2 u_t \\
    \ell_t = \ell_{t-1} + b_{t-1} + 0.3 u_t \\
    y_t = \ell_{t-1} + b_{t-1} + s_{t-2} + u_t \\
\end{cases}    
\]

\begin{enumerate}
    \item Известно, что $s_{100} = 2$, $s_{99} = -1.9$, $b_{100} = 0.5$, $\ell_{100} = 4$. Найдите 95\% предиктивный интервал для $y_{102}$. 
    \item В этой задаче все параметры известны. Сколько параметров оценивается в реальной задаче прогнозирования с помощью $ETS(AAA)$ модели?
\end{enumerate}
\begin{sol}

  Запишем $y_{102}$:
  \[
  y_{102} = \ell_{101} + b_{101} + s_{100} + u_{102} = \ell_{100} + b_{100} + 0.3 u_{101} + b_{100} + 0.2 u_{101} + s_{100} + u_{102}
  \]

  Найдём условное математическое ожидание $y_{102}$ при известной информации $\cF_{100}$:
  \[
  \E(y_{102}\mid \cF_{100}) = \E(\ell_{100} + b_{100} + 0.3 u_{101} + b_{100} + 0.2 u_{101} + s_{100} + u_{102}\mid \cF_{100}) = \ell_{100} + 2 b_{100} + s_{100} = 4 + 2 \cdot 0.5 + 2 = 7
  \]

  Аналогично, найдём условную дисперсию $y_{102}$ при известной информации $\cF_{100}$:
  \begin{align*}  
  \Var(y_{102}\mid \cF_{100}) = \Var(\ell_{100} + b_{100} + 0.3 u_{101} + b_{100} + 0.2 u_{101} + s_{100} + u_{102}\mid \cF_{100}) = &\\
  = \Var((0.3+0.2)u_{101} + u_{102}\mid \cF_{100}) = 0.25 \Var(u_{101}) + \Var(u_{102}) = 1 + 4 = 5&
\end{align*}
  
  В результате, $(y_{102} \mid \cF_{100}) \sim \cN(7, 5)$, а значит 95\% доверительный интервал имеет вид:
  \[
  \left[7 - 1.96 \cdot \sqrt{5}, 7 + 1.96 \cdot \sqrt{5} \right]
  \]
  
  Ответ: 7 свободных параметров для ETS(AAA) с полугодовой сезонностью:
  \[
  s_0, \quad b_0,  \quad \ell_0, \quad \alpha, \quad \beta, \quad \gamma, \quad \sigma^2
  \]
  
  Примечание: 
  \[
  s_0 + s_{-1} = 0 \quad \quad \quad  s_{-1} = - s_{0}
  \]

\end{sol}
\end{problem}

\begin{problem}
Вспомним $ETS(AAN)$ модель, кстати, вот и уравнения:

\[
\begin{cases}
y_t = \ell_{t-1} + b_{t-1} + u_t \\
\ell_t = \ell_{t-1} + b_{t-1} + \alpha u_t \\
b_t = b_{t-1} + \beta u_t \\
u_t \sim \cN(0;\sigma^2) \\
% s_t = s_{t-12} + \gamma \varepsilon_t \\
\end{cases}
\]

\begin{enumerate}
	\item 
	Докажите, что ни при каких $\ell_0$ и $b_0$ этот процесс не будет стационарным. 
	Или опровергните и приведите пример, при каких будет. 
	
	Константы $\alpha$, $\beta$ лежат в интервале $(0;1)$.
	
	\item При $\ell_{100} = 20$, $b_{100} = 2$, $\alpha=0.2$, $\beta=0.3$, $\sigma^2 = 16$ постройте
	интервальный прогноз на один и два шага вперёд. 
\end{enumerate}
\begin{sol}
\end{sol}
\end{problem}





\chapter{TBATS}

Оригинальная статья, \cite{de2011forecasting}.

Относим к ETS как модель с одной ошибкой в разных уравнениях. 

\begin{problem}
  Найдите предел
  \[
    \lim_{w \to 0} \frac{y^w - 1}{w}
  \]
\begin{sol}
  $\ln y$
\end{sol}
\end{problem}


