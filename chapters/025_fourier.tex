% !TEX root = ../ts_pset_main.tex


\chapter{Вступайте в ряды Фурье!}


Суть преобразования Фурье. 
Вместо исходного временного ряда $x_0$, $x_1$, \ldots, $x_{N-1}$ мы получаем ряд комплексных чисел $X_0$, $X_1$, \ldots, $X_{N-1}$.
Эти комплексные числа $X_k$ показывают, насколько сильно проявляется каждая частота в исходном ряду.

Чтобы получить одно комплексное число $X_k$:

\begin{enumerate}
  \item Разрежем круг на $N$ равных частей. Каждая часть образует угол $2\pi/N$.
  \item Разместим исходные числа $x_0$, $x_1$, \ldots, $x_{N-1}$ на разрезах по часовой стрелке с шагом $k$.
    При этом число $x_0$ приходится на угол $0$; число $x_1$ — на угол $2\pi/N \cdot k$;
число $x_2$ — на угол $2\pi/N \cdot 2k$, и так далее.
  \item Трактуем $x_i$ как силу ветра в направлении разреза.
  \item $X_k$ — усреднённая сила ветра.
\end{enumerate}


Прямое преобразование Фурье задаётся формулой\footnote{Иногда множитель $1/N$ относят к обратному преобразованию Фурье, иногда поровну разносят как $1/\sqrt{N}$.}:
\[
  X_k = \frac{1}{N} \sum_{n=0}^{N-1} x_n w^{kn},
\]
где комплексное число $w$ кодирует поворот на $1/N$ часть круга по часовой стрелке, $w = \exp\left(\frac{-2i\pi}{N} \right)$.



Обратное преобразование Фурье
\[
  x_n = \sum_{k=0}^{N-1} X_k (w^{*})^{nk},
\]
где комплексное число $w^{*}$ является сопряжённым к числу $w$.


\begin{problem}
  Немножко теории:
  \begin{enumerate}
    \item Посмотрите видео от 3blue1brown, \url{https://www.youtube.com/watch?v=cV7L95IkVdE}.
    \item Прочтите про дискретное преобразование Фурье на brilliant, \url{https://brilliant.org/wiki/discrete-fourier-transform/}.
  \end{enumerate}
\begin{sol}
\end{sol}
\end{problem}



\begin{problem}
  Про Фурье :)
  \begin{enumerate}
    \item Зачем Фурье собирал огарки свечей в бенедиктинской артиллерийской школе?
    \item Первый раз Фурье был арестован за недостаточную поддержку якобинцев. За что Фурье был арестован во второй раз?
    \item После потерей французами Каира Фурье вёл переговоры о перимирии. Что было у него в руке в момент переговоров? 
    Что произошло с этим предметом?
  \end{enumerate}
  \begin{sol}
    \begin{enumerate}
      \item Чтобы заниматься математикой по ночам.
      \item За поддержку якобинцев.
      \item Кофейник. Был разбит пулей.
    \end{enumerate}
\end{sol}
\end{problem}


\begin{problem}
  Вспомним комплексные числа :)
  \begin{enumerate}
    \item Найдите сумму $7 + 7 \exp(2i\pi/3) + 7 \exp(4i\pi/3)$;
    \item Найдите сумму $6 + 4\exp(i\pi)$;
  \end{enumerate}

  \begin{sol}
  \end{sol}
\end{problem}


\begin{problem}
Найдите прямое преобразование Фурье последовательностей
\begin{enumerate}
  \item $1$, $4$, $1$, $4$, $1$, $4$;
  \item $1$, $9$;
  \item $8$;
  \item $1$, $0$, $0$, $0$;
\end{enumerate}
  \begin{sol}
  \end{sol}
\end{problem}

\begin{problem}
  Прямое преобразование Фурье можно записать в матричном виде $X = \frac{1}{N}Fx$.
  \begin{enumerate}
    \item Как устроена матрица $F$?
    \item Найдите $F\cdot F^{*}$, где $F^{*}$ — транспонированная и сопряжённая матрица к $F$;
    \item Как устроена матрица $F^{-1}$?
    \item Как записывается обратное преобразование Фурье в матричном виде?
  \end{enumerate}

  \begin{sol}
  \end{sol}
\end{problem}


\begin{problem}

Обратное преобразование Фурье задаётся формулой
\[
  x_n = \sum_{k=0}^{N-1} X_k (w^{*})^{nk},
\]
где комплексное число $w^{*}$ является сопряжённым к числу $w = \exp\left(\frac{-2i\pi}{N} \right)$.


  Докажите, что обратное преобразование Фурье, действительно, от комплексных чисел $(X_k)$ переходит к исходныму ряду $(x_n)$.
  \begin{sol}
  \end{sol}
\end{problem}


\begin{problem}
  В типичной задаче исходный ряд $x_0$, $x_1$, \ldots, $x_{N-1}$ является действительными числами.
  Докажите, что при дискретном преобразовании Фурье числа $X_k$ и $X_{N-k}$ являются комплексно-сопряжёнными.

  \begin{sol}
  \end{sol}
\end{problem}


\begin{problem}
  Рассмотрим ряд месячной периодичности. Число наблюдений делится на 12. Исследователь Василий рассматривает в качестве регрессоров следующие переменные: столбец из единиц,
  $\sin\left(\frac{2\pi}{12} t\right)$,
 $\cos\left(\frac{2\pi}{12} t\right)$,
 $\sin\left(\frac{2\pi}{12} 2t\right)$,
 $\cos\left(\frac{2\pi}{12} 2t\right)$,
 $\sin\left(\frac{2\pi}{12} 3t\right)$,
 $\cos\left(\frac{2\pi}{12} 3t\right)$,
 $\sin\left(\frac{2\pi}{12} 4t\right)$,
 $\cos\left(\frac{2\pi}{12} 4t\right)$,
 $\sin\left(\frac{2\pi}{12} 5t\right)$,
 $\cos\left(\frac{2\pi}{12} 5t\right)$,
 $\cos\left(\frac{2\pi}{12} 6t\right)$.

 \begin{enumerate}
   \item Являются ли эти регрессоры ортогональными?
   \item Василий рассматривает два варианта действий.
     Вариант А: построить 12 регрессий исходного ряда на каждый регрессор в отдельности. Вариант Б: построить одну регрессию.
     Будут ли отличаться коэффициенты при регрессорах?
    \item Можно ли добавить в качестве регрессора $\sin\left(\frac{2\pi}{12} 6t\right)$ или  $\cos\left(\frac{2\pi}{12} 7t\right)$?
  \end{enumerate}
 \begin{sol}
   Да, ряды являются ортогональными. Можно строить регрессии на эти регрессоры в любых комбинациях, оценки бет выходят одни и те же.
   Другие ряды добавить нельзя — будет строгая мультиколлинеарность.
 \end{sol}
 \end{problem}

 \begin{problem}
   Исследовательница Агриппина взяла ряд длиной 6 наблюдений и построила его регрессию на тригонометрические ряды Фурье:
   \[
     \hat x_t = 3.5 - 1.73 \sin(2\pi t/6) + 1.00 \cos(2\pi t/6) - 0.58\sin(4\pi t/6) + 1.00 \cos(4\pi t/6) +0.30 \cos(6\pi t/6)
   \]

   Найдите прямое преобразование Фурье исходного ряда.
   \begin{sol}
     На всякий случай, это был ряд $1$, $2$, $3$, $4$, $5$, $6$.
   \end{sol}
 \end{problem}


 \begin{problem}
   Исследовательница Агриппина взяла ряд длиной 6 наблюдений и нашла его преобразование Фурье:
   \[
     1.5, \; -\frac{1}{6}+\frac{1}{\sqrt{12}}i, \; 0, \; -\frac{1}{6}, \; 0, -\frac{1}{6} - \frac{1}{\sqrt{12}}i.
   \]
   \begin{enumerate}
     \item Найдите регрессию этого ряда на тригонометрические ряды Фурье;
     \item Восстановите исходный ряд;
   \end{enumerate}

   \begin{sol}
   $1$, $1$, $1$, $2$, $2$, $2$
   \end{sol}
 \end{problem}


