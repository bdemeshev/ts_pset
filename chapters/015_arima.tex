% !TEX root = ../ts_pset_main.tex

\chapter{ARMA}

Многие источники неверно рассказывают критерий стационарности ARIMA процесса. 
Проверено, мин нет: \cite{van2010time}, \cite{tsay2005analysis}.


\begin{problem}
Рассмотрим модель $y_t=\mu + \e_t$, где $\e_t$ — стационарный AR(1) процесс $\e_t=\rho \e_{t-1} + u_t$ с $u_t \sim N(0,\sigma^2)$. Найдите условную логарифмическую функцию правдоподобия $\ell(\mu, \rho, \sigma^2 | y_1)$.
\begin{sol}

\end{sol}
\end{problem}

\begin{problem}
Известно, что $\e_t$ — белый шум. Классифицируйте в рамках классификации ARIMA процесс $y_t=1+\e_t + 0.5\e_{t-1} + 0.4\e_{t-2} + 0.3\e_{t-3} + 0.2y_{t-1} + 0.1y_{t-2}$.
\begin{sol}

ARMA(2,3), ARIMA(2,0,3)
\end{sol}
\end{problem}


\begin{problem}
На графике представлены данные по уровню озера Гур\'{о}н в футах в 1875-1972 годах:

\begin{minted}[mathescape,
               linenos,
               numbersep=5pt,
               frame=lines,
               framesep=2mm]{r}
level <- LakeHuron
df <- data.frame(level, obs = 1875:1972)
n <- nrow(df) # used later for answers
v.acf <- acf(level, plot = FALSE)$acf
v.pacf <- pacf(level, plot = FALSE)$acf
acfs.df <- data.frame(lag = c(1:15, 1:15),
    acf = c(v.acf[2:16], v.pacf[1:15]),
    acf.type = rep(c("ACF", "PACF"), each = 15))
model <- arima(level, order = c(1, 0, 1))
resids <- model$residuals
resid.acf <- acf(resids, plot = FALSE)$acf
\end{minted}



\begin{minted}[mathescape,
               linenos,
               numbersep=5pt,
               frame=lines,
               framesep=2mm]{r}
tikz("../R_plots/huron_ts.tikz", standAlone = FALSE, bareBones = TRUE)
ggplot(df, aes(x = obs, y = level)) + geom_line() +
    labs(x = "Год", y = "Уровень озера (футы)")
dev.off()
\end{minted}


%\begin{minipage}{\textwidth}
%\begin{tikzpicture}[scale = 0.025]
%\input{R_plots/huron_ts.tikz}
%\end{tikzpicture}
%\end{minipage}




График автокорреляционной и частной автокорреляционной функций:

\begin{minted}[mathescape,
               linenos,
               numbersep=5pt,
               frame=lines,
               framesep=2mm]{r}
ggplot(acfs.df, aes(x = lag, y = acf, fill = acf.type))+
    geom_histogram(position = "dodge", stat = "identity")+
  xlab("Лаг") + ylab("Корреляция") +
  guides(fill = guide_legend(title = NULL))+
  geom_hline(yintercept = 1.96 / sqrt(nrow(df)))+
  geom_hline(yintercept = -1.96 / sqrt(nrow(df)))
\end{minted}


\begin{enumerate}
\item Судя по графикам, какие модели класса ARMA или ARIMA имеет смысл оценить?
\item По результатам оценки некоей модели ARMA c двумя параметрами, исследователь посчитал оценки автокорреляционной функции для остатков модели. Известно, что для остатков модели первые три выборочные автокорреляции равны соответственно $0.00467$, $-0.0129$ и $-0.063$. С помощью подходящей статистики проверьте гипотезу о том, что первые три корреляции ошибок модели равны нулю.
\end{enumerate}


\begin{sol}
\begin{enumerate}
\item Процесс $AR(2)$, т.к. две первые частные корреляции значимо отличаются от нуля, а гипотезы о том, что каждая последующая равна нулю не отвергаются.
\item Можно использовать одну из двух статистик
\[
\text{Ljung-Box}=n(n+2)\sum_{k=1}^3\frac{\hat{\rho}_k^2}{n-k}=
0.42886
\]
\[
\text{Box-Pierce}=n\sum_{k=1}^3\hat{\rho}_k^2=
0.4076
\]
Критическое значение хи-квадрат распределения с 3-мя степенями свободы для $\alpha=0.05$ равно $\chi^2_{3,crit}=7.81$.
Вывод: гипотеза $H_0$ об отсутствии корреляции ошибок модели не отвергается.
\end{enumerate}
\end{sol}
\end{problem}




\begin{problem}
Процесс $x_t$ — это процесс $y_t$, наблюдаемый с ошибкой, т.е. $x_t=y_t+\nu_t$. Ошибки $\nu_t$ являются белым шумом и не коррелированы с $y_t$.
\begin{enumerate}
\item Является ли процесс $x_t$ MA(1) процессом, если $y_t$ —  MA(1) процесс? Если да, то как связаны их автокорреляционные функциии?
\item Является ли процесс $x_t$ стационарным AR(1) процессом, если $y_t$ —  стационарный AR(1) процесс? Если да, то как связаны их автокорреляционные функциии?
\end{enumerate}


\begin{sol}

\end{sol}
\end{problem}


\begin{problem}
Рассмотрим стационарный AR(1) процесс $y_t=\rho y_{t-1} + \e_t$, где $\e_t \sim N(0,1)$. Имеется ряд $y_1$, $y_2$, \ldots, $y_{101}$. Построен график этого процесса. Как от $\rho$ зависит математическое ожидание количества пересечений графика с осью абсцисс?


\begin{sol}
Среднее количество пересечений равно 50 помножить на вероятность того, что два соседних $y_t$ разного знака. Найдём вдвое меньшую вероятность, $\P(y_1>0, y_2 <0)$.
\end{sol}
\end{problem}



\begin{problem}
Рассмотрим процессы:

\begin{enumerate}
\item[A] Процесс скользящего среднего:
\[
y_t=\e_t+2\e_{t-1}+3
\]

\item[B]
\[
a_t=\e_t+\e_1 + 3
\]

\item[C]
\[
b_t=t\e_t + 3
\]

\item[D]
\[
c_t=\cos\left(\frac{\pi t}{2}\right)\e_1 + \sin\left(\frac{\pi t}{2}\right)\e_2 + 2
\]

\item[E] Процесс случайного блуждания со смещением:
\[
\begin{cases}
z_t=\e_t+z_{t-1}+3 \\
z_0=0
\end{cases}
\]

\item[F] Процесс с трендом:
\[
w_t=2+3t+\e_t
\]

\item[G] Еще один процесс:
\[
r_t=\begin{cases}
1, \; \text{при четных t} \\
-1, \; \text{при нечетных t}
\end{cases}
\]

\item[H] Приращение случайного блуждания
\[
s_t=\Delta z_t
\]

\item[I] Приращение процесса с трендом
\[
d_t=\Delta w_t
\]
\end{enumerate}

Для каждого процесса:

\begin{enumerate}
\item Найдите $\E(y_t)$, $\Var(y_t)$
\item Найдите $\gamma_k = \Cov(y_t, y_{t-k})$
\item Найдите $\rho_k = \Corr(y_t,y_{t-k})$. Если ни одна корреляция $\rho_k$ не зависит от времени $t$, то постройте график зависимости $\rho_k$ от $k$.
\item Является ли процесс стационарным?
\item Сгенерируйте одну реализацию процесса. Постройте её график и график оценки автокорреляционной функции.
\end{enumerate}


\begin{sol}

\[
\E(b_t) = 3
\]

\[
\Var(b_t) = t^2 \sigma^2_{\e}
\]

\[
\Cov(b_t, b_{t-k}) = 0, k \geq 1
\]

\[
\Corr(b_t, b_{t-k}) = 0, k \geq 1
\]

$b_t$ — нестационарный из-за дисперсии


\[
\E(c_t) = 2
\]

\[
\Var(c_t) = \sigma^2_{\e}
\]

\[
\Cov(c_t, c_{t-k}) = \cos( \pi k /2)\sigma^2_{\e}, k \geq 1
\]

\[
\Corr(c_t, c_{t-k}) = \cos( \pi k /2), k \geq 1
\]

$c_t$ — стационарный
\end{sol}
\end{problem}





\begin{problem}
Эконометресса Антуанетта построила график автоковариационной функции временного ряда и распечатала его:

здесь график

Потом она с ужасом обнаружила, что до презентации исследования остается совсем мало времени, а распечатать надо было график автокорреляционной функции. Что надо исправить Антуанетте на графике, чтобы успеть еще сделать причёску и макияж (это очень важно для презентации)?



\begin{sol}
зачеркнуть одну цифру
\end{sol}
\end{problem}


\begin{problem}
Рассмотрите стационарные процессы
\begin{enumerate}
\item[A.] AR(1): $y_t = 5 + 0.3y_{t-1} + \e_t$
\item[B.] AR(2): $y_t = 5 + 0.3y_{t-1} + 0.1 y_{t-2} + \e_t$
\item[C.] MA(1): $y_t = 5 + 0.3\e_{t-1} + \e_t$
\item[D.] MA(2): $y_t = 5 + 0.3\e_{t-1} + 0.9\e_{t-2} + \e_t$
\item[E.] ARMA(1, 1): $y_t = 5 + 0.3y_{t-1} + 0.4\e_{t-1} + \e_t$
\end{enumerate}

Если возможно, то представьте каждый процесс в виде:
\begin{enumerate}
\item $MA(\infty)$.
\item $AR(\infty)$.
\item $y_t = c + \gamma_1 y_{t-1} + u_t$, где $u_t$ некоррелирован с $y_{t-1}$. Будет ли $u_t$ белым шумом?
\item $y_t = c + \gamma_1 y_{t+1} + u_t$, где $u_t$ некоррелирован с $y_{t+1}$. Будет ли $u_t$ белым шумом?
\item $y_t = c + \gamma_1 y_{t-1} + \gamma_2 y_{t-2} + u_t$, где $u_t$ некоррелирован с $y_{t-1}$ и $y_{t-2}$. Будет ли $u_t$ белым шумом?
\item $y_t = c + \gamma_1 y_{t+1} + \gamma_2 y_{t+2} + u_t$, где $u_t$ некоррелирован с $y_{t+1}$ и $y_{t+2}$. Будет ли $u_t$ белым шумом?
\end{enumerate}
\begin{sol}
\end{sol}
\end{problem}


\begin{problem}
Рассмотрите стационарные процессы
\begin{enumerate}
\item[A.] AR(1): $y_t = 5 + 0.3y_{t-1} + \e_t$
\item[B.] AR(2): $y_t = 5 + 0.3y_{t-1} + 0.1 y_{t-2} + \e_t$
\item[C.] MA(1): $y_t = 5 + 0.3\e_{t-1} + \e_t$
\item[D.] MA(2): $y_t = 5 + 0.3\e_{t-1} + 0.9\e_{t-2} + \e_t$
\item[E.] ARMA(1, 1): $y_t = 5 + 0.3y_{t-1} + 0.4\e_{t-1} + \e_t$
\end{enumerate}

Для каждого из процессов:
\begin{enumerate}
\item Найдите математическое ожидание $\E(y_t)$.
\item Найдите первые три значения автокорреляционной функции $\rho_1$, $\rho_2$, $\rho_3$.
\item Найдите первые три значения частной автокорреляционной функции $\phi_{11}$, $\phi_{22}$, $\phi_{33}$.
\end{enumerate}
\begin{sol}
\end{sol}
\end{problem}

\begin{problem}
  Известна автокорреляционная функция стационарного процесса $(y_t)$: $\rho_1 = 0.7$, $\rho_2 = 0.3$, и $\rho_k = 0$ при $k\geq 3$. Кроме того, $\E(y_t)=4$. Выпишите возможные уравнения процесса.
\begin{sol}
По нулевым корреляциям догадываемся, что это процесс $MA(2)$.
\[
y_y = 4 + u_t + \alpha_1 u_{t-1} + \alpha_2 u_{t-2}
\]
\[
\begin{cases}
  \frac{\alpha_1 \alpha_2 + \alpha_1}{\alpha_2} = 7/3 \\
  \frac{\alpha_1^2 + \alpha_2^2 + 1}{\alpha_2} = 10/3 \\
\end{cases}
\]


\end{sol}
\end{problem}

\begin{problem}
  Известна частная автокорреляционная функция стационарного процесса $(y_t)$: $\phi_{11} = 0.7$, $\phi_{22} = 0.3$, и $\phi_{kk} = 0$ при $k\geq 3$. Кроме того, $\E(y_t)=4$. Выпишите возможные уравнения процесса.
\begin{sol}
\end{sol}
\end{problem}

\begin{problem}
Если возможно, то найдите процесс с данной автокорреляционной или частной автокорреляционной функцией.

\begin{enumerate}
  \item $ACF = (0.9, -0.9, 0, 0, 0, \ldots)$;
  \item $PACF = (0.9, -0.9, 0, 0, 0, \ldots)$;
  \item $PACF = (0.9, 0, 0, 0, 0, \ldots)$;
  \item $PACF = (0, 0.9, 0, 0, 0, 0, \ldots)$;
  \item $ACF = (0.9, 0, 0, 0, 0, \ldots)$;
  \item $ACF = (0, 0.9, 0, 0, 0, 0, \ldots)$;
\end{enumerate}

\begin{sol}
  \begin{enumerate}
    \item $ACF = (0.9, -0.9, 0, 0, 0, \ldots)$ не бывает, так как определитель корреляционной матрицы 3 на 3 отрицательный;
    \item $PACF = (0.9, -0.9, 0, 0, 0, \ldots)$ — AR(2);
    \item $PACF = (0.9, 0, 0, 0, 0, \ldots)$ — $y_t = 0.9y_{t-1} + u_t$;
    \item $PACF = (0, 0.9, 0, 0, 0, 0, \ldots)$ — $y_t = 0.9y_{t-2} + u_t$;
    \item $ACF = (0.9, 0, 0, 0, 0, \ldots)$ — не бывает, подозрение падает на MA(1), но решения только с комплексными коэффициентами, геометрически: два угла с косинусом 0.9, то есть примерно по 30 градусов, и они даже в сумме не могут дать перпендикуляр;
    \item $ACF = (0, 0.9, 0, 0, 0, 0, \ldots)$ — не бывает, если проредить процесс через один, то должна получится невозможная ACF;
  \end{enumerate}
   В целом PACF может быть любая,
   \url{http://projecteuclid.org/euclid.aos/1176342881}.
\end{sol}
\end{problem}


\begin{problem}
Рассмотрим стационарный процесс $y_t = 4 + 0.7y_{t-1} - 0.12y_{t-2} + \e_t$, где $\e_t$ — белый шум, причём $\Cov(\e_t, y_{t-k})=0$ при $k \geq 1$.

\begin{enumerate}
  \item Найдите автокорреляционную функцию: $\rho_1$, $\rho_2$ и общую формулу для $\rho_k$.
  \item Найдите $\lim_{k \to \infty} \rho_k$.
  \item Найдите частную автокорреляционную функцию: $\phi_{11}$, $\phi_{22}$, \ldots.
\end{enumerate}
\begin{sol}
  $\phi_{kk}=0$ при $k \geq 3$.
\end{sol}
\end{problem}


\begin{problem}
Рассмотрим стационарный процесс с уравнением
\[
y_t = 10 + 0.69 y_{t-1} + \e_t - 0.71 \e_{t-1}.
\]

Выпишите гораздо более простой процесс со свойствами близкими к свойствам данного процесса.
\begin{sol}
Заметим, что $0.69\approx 0.71$, сокращаем множитель $1-0.7L$, получаем $y_t = 100/3 + \e_t$.
\end{sol}
\end{problem}


\begin{problem}
Процесс $\e_t$ — белый шум. Рассмотрим уравнение
\[
y_t = 0.5 y_{t-1} + \e_t.
\]

Какие из указанных процессов $(y_t)$ являются его решением? Стационарным решением?
\begin{enumerate}
  \item $y_t = 0.5^t$;
  \item $y_t = \sum_{i=0}^{\infty} 0.5^i \e_{t-i}$;
  \item $y_t = 0.5^t + \sum_{i=0}^{\infty} 0.5^i \e_{t-i}$;
  \item $y_t = 0.5^t\e_{100} + \sum_{i=0}^{\infty} 0.5^i \e_{t-i}$;
  \item $y_t = 0.5^t + \sum_{i=0}^{t} 0.5^i \e_{t-i}$;
  \item $y_t = \sum_{i=0}^{t} 0.5^i \e_{t-i}$;
\end{enumerate}


\begin{sol}
Стационарным решением является $y_t = \sum_{i=0}^{\infty} 0.5^i \e_{t-i}$. Решениями также являются: $y_t = 0.5^t + \sum_{i=0}^{\infty} 0.5^i \e_{t-i}$, $y_t = 0.5^t\e_{100} + \sum_{i=0}^{\infty} 0.5^i \e_{t-i}$, $y_t = 0.5^t + \sum_{i=0}^{t} 0.5^i \e_{t-i}$, $y_t = \sum_{i=0}^{t} 0.5^i \e_{t-i}$.
\end{sol}
\end{problem}



\begin{problem}

Рассмотрим стационарный процесс $y_t$, задаваемый уравнением
\[
y_t = 2 + 0.6 \cdot y_{t-1} - 0.08 y_{t-2} + \e_t,
\]
где $\e_t \sim \cN(0; 4)$.

\begin{enumerate}
\item  Найдите $\E_t(y_{t+1})$, $\Var_t(y_{t+1})$
\item Найдите $\E_t(y_{t+2})$, $\Var_t(y_{t+2})$
\item Постройте 95\%-ый предиктивный интервал для $y_{102}$, если $y_{99}=5$, $y_{100}=5.1$
\item Найдите $\E(y_t)$, $\Var(y_t)$
\item Найдите $\lim_{h\to\infty}\E_t(y_{t+h})$, $\lim_{h\to\infty}\Var_t(y_{t+h})$
\end{enumerate}


\begin{sol}

$\E_t(y_{t+1})=2+0.6y_{t-1}-0.08y_{t-2}$, $\Var_t(y_{t+1})=4$

$\E_t(y_{t+2})=3.2 + 0.28 y_t- 0.048y_{t-1}$, $\Var_t(y_{t+2})=1.36 \cdot 4$

$\E_{100}(y_{102})= 4.388$, $\Var_{100}(y_{102})=5.44$.

Предиктивный интервал $[4.388 - 1.96 \sqrt{5.44};4.388 + 1.96 \sqrt{5.44}]$

$\E(y_t)=\frac{2}{0.48}\approx 4.17$

\end{sol}
\end{problem}



\begin{problem}
Задан процесс $y_t = 7 + u_t + 0.2 u_{t-1}$, где $u_t$ независимы и нормальны $u_t \sim \cN(0;4)$. Известно, что $y_{100}=7.2$, $u_{100}=1.3$, $y_{100}+(-0.2)y_{99}+(-0.2)^2y_{98}+\ldots+(-0.2)^{99}y_1=5.6$.

Пусть $\cF_t=\sigma(y_t, y_{t-1}, \ldots, y_1, u_t, u_{t-1}, \ldots, u_1)$ и $\cH_t = \sigma(y_t, y_{t-1}, \ldots, y_1)$.
\begin{enumerate}
  \item Найдите $\E(y_{101}|\cF_{100})$, $\Var(y_{101}|\cF_{100})$.
  \item С помощью $AR(\infty)$ представления примерно найдите $\E(y_{101}|\cH_{100})$, $\Var(y_{101}|\cH_{100})$. Постройте 95\%-ый предиктивный интервал для $y_{101}$.
  \item Найдите $\E(y_{101}|y_{100})$, $\Var(y_{101}|y_{100})$.
  \item Найдите $\E(y_{101}|y_{100}, y_{99})$, $\Var(y_{101}|y_{100}, y_{99})$.
\end{enumerate}

\begin{sol}
Заметим, что $\Var(u_t|\cF_t)=0$. Более того, для обратимого процесса $\Var(u_t|y_t, y_{t-1}, \ldots, y_1) \approx \Var(u_t|y_t, y_{t-1}, \ldots) = 0$.
\[
\E(y_{101}|y_{100})=7 + 0 + 0.2\E(u_{100}|y_{100})
\]
\[
\E(u_{100}|y_{100}) = \beta_1 + \beta_2 y_{100}
\]
\[
\beta_2 = \frac{\Cov(y_{100}, u_{100})}{\Var(y_{100})}=4/4.16, \beta_1 = \E(u_{100}) - \beta_2 \E(y_{100})=-4\cdot 7/4.16
\]
\[
\frac{y_t}{1+0.2L} = \frac{7}{1+0.2L} + u_t
\]
Заметим, что $\frac{7}{1+0.2L}=7/1.2$, так как $L\cdot 7 = 7$ (вчера семь равнялось семи).

По условию $\frac{y_{100}}{1+0.2L} \approx 5.6$. Знак «примерно равно» возникает из-за замены бесконечной суммы на конечную.

\end{sol}
\end{problem}



\begin{problem}
  У исследовательницы Аграфены три наблюдения, $y_1 = 0.1$, $y_2 = -0.2$, $y_3 = 0.2$. Аграфена предполагает, что данные подчиняются стационарному AR(1) процессу $y_t = \beta y_{t-1} + u_t$, где $u_t \sim \cN(0;\sigma^2_u)$.

  \begin{enumerate}
    \item Найдите $\E(y_1)$, $\E(y_2|y_1)$, $\E(y_3|y_2)$;
    \item Найдите $\Var(y_1)$, $\Var(y_2|y_1)$, $\Var(y_3|y_2)$;
    \item Найдите функции плотности $f(y_1)$, $f(y_2|y_1)$, $f(y_3|y_2)$;
    \item Выпишете полную логарифмическую функцию правдоподобия $\ell(y|\beta, \sigma^2_u)$.
    \item Если возможно, явно решите задачу максимизации полного правдоподобия.
    \item Выпишите условную логарифмическую функцию правдоподобия $\ell(y_2, y_3|\beta, \sigma^2_u, y_1)$.
    \item Если возможно, явно решите задачу максимизации условного правдоподобия при фиксированном $y_1$.
  \end{enumerate}
  \begin{sol}
    $\E(y_1)=0$, $\Var(y_1)=\sigma^2_u/(1-\beta^2)$, $\E(y_t|y_{t-1})=\beta y_{t-1}$, $\Var(y_t|y_{t-1})=\sigma^2_u$.

    При максимизации условного правдоподобия получаем:
    \[
         \hb = \frac{y_1 y_2 + y_2 y_3}{y_1^2 + y_2^2}
    \]
  \end{sol}
\end{problem}

\begin{problem}
  Белые шумы $u_t$ и $v_t$ независимы, $\Var(u_t) = 1$, $\Var(v_t)=1$. Рассмотрим процесс $y_t = 5u_{t-1} - 4 v_{t-1} + u_t + v_t$.

  \begin{enumerate}
    \item Выпишите классическое представление процесса $y_t$ как ARMA-процесса.
    \item Выразите белый шум из полученного классического представления $y_t$ через белые шумы $(u_t)$ и $(v_t)$.
  \end{enumerate}
  \todo[inline]{можно подобрать цифры, чтобы коэффициент был хороший :)}
  \begin{sol}
  \end{sol}
\end{problem}



\begin{problem}
  Рассмотрим модель случайного блуждания, 
  \[
  \begin{cases}
    y_0 = c, \\
    y_t = y_{t-1} + u_t, \\
    u_t \sim \cN(0, \sigma^2_u) \text{ и независимы}\\
  \end{cases}
  \]
  \begin{enumerate}
    \item Найдите $\E(y_{10})$, $\Var(y_{10})$, закон распределения $y_{10}$;
    \item Найдите $\E(y_{10}|y_7)$, $\Var(y_{10}|y_7)$, условный закон распределения $y_{10}$ при известном $y_7$;
    \item Найдите условный закон распределения $y_{101}$ при известном $y_{100}$, 
    условный закон распределения $y_{102}$ при известном $y_{100}$.
    \item Постройте 95\%-й предиктивный интервал для $y_{101}$, 95\%-й предиктивный интервал для $y_{102}$, 
    если известно, что $c=4$, $\sigma^2_u = 9$, $y_{100}=20$.
    \item Оцените параметры $c$ и $\sigma^2_u$ методом максимального правдоподобия, если $y_1 = 4$, $y_2 = 7$, $y_3 = 6$.
    \item Оцените параметры $c$ и $\sigma^2_u$ методом максимального правдоподобия в общем случае.
  \end{enumerate}

  \begin{sol}
  \end{sol}
\end{problem}


\begin{problem}

  Процессы $y_t$ и $u_t$ стационарны и заданы системой уравнений
  \[
  \begin{cases}
    y_t = \beta y_{t-1} + u_t \\
    u_t = \alpha u_{t-1} + \nu_t, \\
  \end{cases}  
  \]
  где $(\nu_t)$ — белый шум. Коэффициенты $\beta$ и $\alpha$ по модулю меньше единицы.

  Исследовательница Ада оценивает обычную регрессию $\hat y_t = \hat \beta_1 + \hat \beta_2 y_{t-1}$ с помощью МНК.

  Какие оценки она получит при большом размере выборки?
  
  \begin{sol}
    \[
    \plim \hat \beta_2 = \frac{\beta + \alpha}{1 + \beta \alpha}  
    \]
  \end{sol}

\end{problem}


\begin{problem}
Процесс $(u_t)$ — белый шум с дисперсией $\sigma^2_u$. 
Процесс $(y_t)$ задан уравнением  $y_t = 5 + u_t + 2u_{t-1}$.

\begin{enumerate}
  \item Найдите $\E(y_t)$, $\Var(y_t)$, $\Cov(y_t, y_s)$.
\end{enumerate}

Про процесс $(z_t)$ известно, что он представим в виде 
$z_t = c + w_t + \alpha w_{t-1}$, где $(w_t)$ — белый шум с дисперсий $\sigma^2_w$.

Ожидание, дисперсия и автоковариационная функция процесса $(z_t)$ в точности такая же, 
как и у процесса $(y_t)$. А именно, $\E(z_t) = \E(y_t)$, $\Var(z_t) = \Var(y_t)$,
$\Cov(z_t, z_s) = \Cov(y_t, y_s)$. Однако, $\alpha \neq 2$.

\begin{enumerate}[resume]
  \item Найдите константы $c$, $\alpha$ и отношение $\sigma^2_w/\sigma^2_u$.
\end{enumerate}
  
  \begin{sol}
Если обозначить отношение дисперсий буквой $R = \sigma^2_w/\sigma^2_u$,
то равенство дисперсии и ковариации даёт систему уравнений: 
\[
  \begin{cases}
    \alpha R = 2 \\
    (1+\alpha^2)R = 5 \\
  \end{cases}
\]
Решений у неё два, старый процесс $(\alpha=2, R=1)$, и новый $(\alpha=0.5, R=4)$.
Из равенства ожиданий следует, что $c=5$.
  \end{sol}
\end{problem}


\begin{problem}
Приведите три различных последовательности чисел $(a_t)_{t=-\infty}^{+\infty}$ таких, 
что $(1+0.5L)a_t = 0$.

  \begin{sol}
    Берем любое $a_0$, а дальше в обе стороны заполняем числа по принципу $a_t = -0.5 a_{t-1}$.
  \end{sol}
\end{problem}


\begin{problem}
  Процесс $(u_t)$ — белый шум.

  Рассмотрим процесс $w_t = (1+2L)(1-0.5L + 0.5^2 L^2 - 0.5^3 L^3 + \ldots )u_t$.

  \begin{enumerate}
    \item Верно ли, что $w_t$ белый шум?
    \item Придумайте ещё парочку белых шумов, линейно выражающихся через шум $u_t$.
  \end{enumerate}

  \begin{sol}
    Да, $w_t$ — белый шум!
  \end{sol}
\end{problem}





\begin{problem}
  Рассмотрим $MA(1)$ процесс $(y_t)$. 
  \begin{enumerate}
    \item В каких пределах может лежать корреляция $\Corr(y_t, y_{t+1})$?
    \item В каких пределах может лежать частная корреляция $\pCorr(y_t, y_{t+2} ; y_{t+1})$?
  \end{enumerate}
  
  \begin{sol}
    $\Corr(y_t, y_{t+1}) \in [-0.5; 0.5]$, $\pCorr(y_t, y_{t+2} ; y_{t+1}) \in [-1; 1/3]$
  \end{sol}
\end{problem}





\begin{problem}
Процессы $(a_t)$ и $(b_t)$ — обычное и сезонное случайные блуждания. 
Стартовые значения равны нулю, $a_0=0$, $b_{-11} = b_{-10} = \ldots = b_{-1} = 0$.
И далее $a_t = a_{t-1} + u_t$, $b_t = b_{t-12} + \nu_t$. 
Случайные процессы $(u_t)$ и $(\nu_t)$ — независимые белые шумы. 

\begin{enumerate}
  \item Получится ли взять несколько раз обычную разность $\Delta = 1 - L$ так, чтобы процесс $\Delta^d a_t$ был стационарным? 
  \item Получится ли взять несколько раз обычную разность $\Delta = 1 - L$ так, чтобы процесс $\Delta^d b_t$ был стационарным? 
  \item Как изменятся ответы на предыдущие вопросы, если брать сезонную разность $\Delta_{12} = 1 - L^{12}$?
\end{enumerate}

  \begin{sol}
    Процессы $\Delta b_t$, $\Delta_{12} a_t$, $\Delta_{12} b_t$ — стационарные. 
    Превратить сезонное случайное блуждание в стационарный процесс взятием обычной разности не получится. 
  \end{sol}
\end{problem}


\begin{problem}


  \begin{sol}
  \end{sol}
\end{problem}


\begin{problem}


  \begin{sol}
  \end{sol}
\end{problem}
