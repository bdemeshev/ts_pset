\protect \hypertarget {soln:1.1}{}
\begin{solution}{{1.1}}
В данном случае статистика $DW$ не применима, так как есть лаг $y_{t-1}$ среди регрессоров.
\end{solution}
\protect \hypertarget {soln:1.2}{}
\begin{solution}{{1.2}}
\begin{enumerate}
\item $\E(\e_t)=0$, $\Var(\e_1)=\sigma^2$, $\Var(\e_t)=2\sigma^2$ при $t\geq 2$.  Гетероскедастичная.
\item $\Cov(e_t,e_{t+1})=\sigma^2$. Автокоррелированная.
\item $\hb$ --- несмещенная, неэффективная
\item Более эффективной будет $\hb_{gls}=(X'V^{-1}X)^{-1}X'V^{-1}y$, где
\[
X=\begin{pmatrix}
x_1 \\
x_2 \\
\vdots \\
x_n
\end{pmatrix}
\]

Матрица $V$ известна с точностью до константы $\sigma^2$, но в формуле для $\hb_{gls}$ неизвестная $\sigma^2$ сократится.

Другой способ построить эффективную оценку --- применить МНК к преобразованным наблюдениям, т.е. $\hb_{gls}=\frac{\sum x'_i y'_i}{\sum x_i^{\prime 2}}$, где $y'_1=y_1$, $x'_1=x_1$, $y'_t=y_t-y_{t-1}$, $x'_t=x_t-x_{t-1}$ при $t\geq 2$.
\end{enumerate}
\end{solution}
\protect \hypertarget {soln:1.3}{}
\begin{solution}{{1.3}}

Для простоты закроем глаза на малое количество наблюдений и как индейцы пираха будем считать, что пять --- это много.

\end{solution}
\protect \hypertarget {soln:1.4}{}
\begin{solution}{{1.4}}
1. Поскольку имеют место соотношения $\varepsilon_1 = \rho \varepsilon_0 + u_1$ и $Y_1 =\mu + \varepsilon_1$, то из условия задачи получаем, что $\varepsilon_1 \sim N(0,\sigma^2 / (1 - \rho^2))$
и $Y_1 \sim N(\mu,\sigma^2 / (1 - \rho^2))$. Поэтому
\[
f_{Y_1}(y_1) = \frac{1}{\sqrt{2\pi\sigma^2/(1-\rho^2)}}\exp{\left(-\frac{(y_1 - \mu)^2}{2\sigma^2/(1 - \rho^2)}\right)}.
\]

Далее, найдем $f_{Y_2|Y_1}(y_2|y_1)$. Учитывая, что $Y_2 = \rho Y_1 + (1- \rho) \mu + u_2$, получаем $Y_2|\{Y_1 = y_1\} \sim N(\rho y_1 + (1- \rho) \mu, \sigma^2)$. Значит,
\[
f_{Y_2|Y_1}(y_2|y_1) = \frac{1}{\sqrt{2\pi\sigma^2}}\exp{\left(-\frac{(y_2 - \rho y_1 - (1- \rho) \mu)^2}{2\sigma^2}\right)}.
\]

Действуя аналогично, получаем, что для всех $t \geq 2$ справедлива формула
\[
f_{Y_{t}|Y_{t-1}}(y_{t}|y_{t-1}) = \frac{1}{\sqrt{2\pi\sigma^2}}\exp{\left(-\frac{(y_{t} - \rho y_{t-1} - (1- \rho) \mu)^2}{2\sigma^2}\right)}.
\]

Таким образом, находим функцию правдоподобия
\[
\mathrm{L}(\mu, \rho, \sigma^2) = f_{Y_T,\ldots,Y_1}(y_T,\dots,y_1) = f_{Y_1}(y_1)\prod_{t=2}^{T}f_{Y_t|Y_{t-1}}(y_t|y_{t-1}) \text{,}
\]
где $f_{Y_1}(y_1)$ и $f_{Y_t|Y_{t-1}}(y_t|y_{t-1})$ получены выше.

2. Для нахождения неизвестных параметров модели запишем логарифмическую условную функцию правдоподобия:
\[
l(\mu, \rho, \sigma^2|Y_1 = y_1) = \sum_{t=2}^{T}\log{f_{Y_t|Y_{t-1}}(y_t|y_{t-1})} =
\]
\[
=-\frac{T-1}{2} \log(2 \pi) - \frac{T-1}{2} \log{\sigma^2} - \frac{1}{2\sigma^2} \sum_{t=2}^{T}(y_t - \rho y_{t-1} - (1 - \rho) \mu)^2 \text{.}
\]

Найдем производные функции $l(\mu, \rho, \sigma^2|Y_1 = y_1)$ по неизвестным параметрам:
\[
\frac{\partial l}{\partial \mu} = -\frac{1}{2\sigma^2} \sum_{t=2}^{T} 2(y_t - \rho y_{t-1} - (1 - \rho) \mu) \cdot (\rho - 1) \text{,}
\]
\[
\frac{\partial l}{\partial \rho} = -\frac{1}{2\sigma^2} \sum_{t=2}^{T} 2(y_t - \rho y_{t-1} - (1 - \rho) \mu) \cdot (\mu - y_{t-1}) \text{,}
\]
\[
\frac{\partial l}{\partial {\sigma^2}} =  - \frac{T-1}{2\sigma^2} + \frac{1}{2\sigma^4} \sum_{t=2}^{T}(y_t - \rho y_{t-1} - (1 - \rho) \mu)^2 \text{.}
\]

Оценки неизвестных параметров модели могут быть получены как решение следующей системы уравнений:
\[
\left\{
  \begin{aligned}
    \frac{\partial l}{\partial \mu} = 0 \text{,} \\
    \frac{\partial l}{\partial \rho} = 0 \text{,} \\
    \frac{\partial l}{\partial {\sigma^2}} = 0 \text{.}
  \end{aligned}
\right.
\]

Из первого уравнения системы получаем, что
\[
\sum_{t=2}^{T}y_{t} - \hat{\rho} \sum_{t=2}^{T}y_{t-1} = (T - 1) (1- \hat{\rho}) \hat{\mu} \text{,}
\]
откуда
\[
\hat{\mu} = \frac{\sum_{t=2}^{T}y_{t} - \hat{\rho} \sum_{t=2}^{T}y_{t-1}}{(T - 1) (1- \hat{\rho})} = \frac{3 - \hat{\rho} \cdot 3}{4\cdot(1-\hat{\rho})} = \frac{3}{4} \text{.}
\]

Далее, если второе уравнение системы переписать в виде
\[
\sum_{t=2}^{T}(y_t - \hat{\mu} - \hat{\rho} (y_{t-1} - \hat{\mu}))(y_{t-1} - \hat{\mu}) = 0 \text{,}
\]
то легко видеть, что
\[
\hat{\rho} = \frac{\sum_{t=2}^{T}(y_t - \hat{\mu})(y_{t-1} - \hat{\mu})}{\sum_{t=2}^{T}(y_{t-1} - \hat{\mu})^2} \text{.}
\]
Следовательно, $\hat{\rho} =-1/11= -0.0909$.

Наконец, из третьего уравнения системы
\[
\hs^2 =\frac{1}{T-1} \sum_{t=2}^{T}(y_t - \hat{\rho} y_{t-1} - (1 - \hat{\rho}) \hat{\mu})^2 \text{.}
\]
Значит, $\hs^2 = 165/242= 0.6818$. Ответы: $\hat{\mu} = 3/4= 0.75$, $\hat{\rho} = -1/11=-0.0909$, $\hs^2 =165/242=0.6818$.
\end{solution}
\protect \hypertarget {soln:1.5}{}
\begin{solution}{{1.5}}
Несмещёнными остаются. Состоятельными не всегда остаются, например, состоятельность исчезает, если все случайные ошибки тождественно равны между собой.

\end{solution}
\protect \hypertarget {soln:1.6}{}
\begin{solution}{{1.6}}
\end{solution}
\protect \hypertarget {soln:1.7}{}
\begin{solution}{{1.7}}
\end{solution}
\protect \hypertarget {soln:1.8}{}
\begin{solution}{{1.8}}
\end{solution}
\protect \hypertarget {soln:1.9}{}
\begin{solution}{{1.9}}
\begin{enumerate}
\item $\E(\e_t)=0$, $\Var(\e_t)=\sigma^2/(1-\rho^2)$
\item $\Cov(\e_t,\e_{t+h})=\rho^h\cdot \sigma^2/(1-\rho^2)$
\item $\Corr(\e_t,\e_{t+h})=\rho^h$
\end{enumerate}
\end{solution}
\protect \hypertarget {soln:1.10}{}
\begin{solution}{{1.10}}
\end{solution}
\protect \hypertarget {soln:1.11}{}
\begin{solution}{{1.11}}
\end{solution}
\protect \hypertarget {soln:1.12}{}
\begin{solution}{{1.12}}
\end{solution}
\protect \hypertarget {soln:1.13}{}
\begin{solution}{{1.13}}
\end{solution}
\protect \hypertarget {soln:1.14}{}
\begin{solution}{{1.14}}
\end{solution}
\protect \hypertarget {soln:1.15}{}
\begin{solution}{{1.15}}
\end{solution}
\protect \hypertarget {soln:2.1}{}
\begin{solution}{{2.1}}
\[
(1 - 0.4L)y_t = 4 + (1 + 0.3L)\e_t
\]
\end{solution}
\protect \hypertarget {soln:2.2}{}
\begin{solution}{{2.2}}
$x_{t}=(1-L)^{t}y_{t}$
\end{solution}
\protect \hypertarget {soln:2.3}{}
\begin{solution}{{2.3}}
$ F_{n}=L(1+L)F_{n} $, значит $ F_{n}=L^{k}(1+L)^{k}F_{n} $ или $ F_{n+k}=(1+L)^{k}F_{n} $

Ответ: $1$
\end{solution}
\protect \hypertarget {soln:2.4}{}
\begin{solution}{{2.4}}
а — неверно, б — верно, в — верно, г — нет.
\end{solution}
\protect \hypertarget {soln:2.5}{}
\begin{solution}{{2.5}}
а, б, в, г — стационарны
\end{solution}
\protect \hypertarget {soln:2.6}{}
\begin{solution}{{2.6}}
Они будут примерно одинаковы. Оценка наклона определяется автоковариационной функцией.
\end{solution}
\protect \hypertarget {soln:2.7}{}
\begin{solution}{{2.7}}

\end{solution}
\protect \hypertarget {soln:2.8}{}
\begin{solution}{{2.8}}

\begin{enumerate}
\item $y_t=1+\e_t+0.5\e_{t-1}+0.25\e_{t-2}$ — стационарный
\item $y_t=-2y_{t-1}-3y_{t-2}+\e_t+\e_{t-1}$
\item $y_t=-0.5y_{t-1} + \e_t$ — стационарный
\item $y_t=1-1.5 y_{t-1} - 0.5 y_{t-2} + \e_t - 1.5\e_{t-1} - 0.5\e_{t-2}$
\item $y_t=1+0.64y_{t-2}+\e_t+0.64\e_{t-1}$ — стационарный
\item $y_t=1+t+\e_t$ — нестационарный
\item $y_t=1+y_{t-1}+\e_t$ — нестационарный
\end{enumerate}
\end{solution}
\protect \hypertarget {soln:2.9}{}
\begin{solution}{{2.9}}
Процесс стационарен только при $y_1=4+\frac{2}{\sqrt{3}}\e_1$. Фразу нужно понимать как «у стохастического разностного уравнения $y_t=2+0.5y_{t-1}+\e_t$ есть стационарное решение».
\end{solution}
\protect \hypertarget {soln:2.10}{}
\begin{solution}{{2.10}}
да, стационарный
\end{solution}
\protect \hypertarget {soln:2.11}{}
\begin{solution}{{2.11}}
да, получается
\end{solution}
\protect \hypertarget {soln:2.12}{}
\begin{solution}{{2.12}}
да, это белый шум. Величина $N$ распределена биномиально, $Bin(n=100,p=1/2)$, $\E(N)=50$.
\end{solution}
\protect \hypertarget {soln:2.13}{}
\begin{solution}{{2.13}}

\begin{enumerate}
  \item $z_t = x_t (1-x_{t-1})y_t$;
    Процесс $z_t$ — белый шум, $\E(z_t)=0$, $\Var(z_t)=6$. Величины $z_t$ зависимы. Например, если $z_t \neq 0$, то $z_{t+1}=z_{t-1}=0$. Величины $z_t$ одинаково распределены.
\item $z_t = y_{t-1}y_t$;
Процесс $z_t$ — белый шум. Величины $z_t$ зависимы. Величины $z_t$ одинаково распределены.
\end{enumerate}


\end{solution}
\protect \hypertarget {soln:2.14}{}
\begin{solution}{{2.14}}
Проекции: $\tilde X_1 = X_1 + Z$; $\tilde X_2 = X_2 + Z$; $\E(X_i|Z)=1-Z$; $\Cov(X_i, Z)=-1/4$;

Величина $Z$ имеет распределение Бернулли, поэтому $\E(Z)=1/2$ и $\Var(Z)=1/4$;

\[
  \pCorr(X_1, X_2; Z) = \frac{-1/2}{12.5} = -\frac{1}{\sqrt{50}}
\]
\[
  \Corr(X_1, X_2|Z)=-Z/6
\]
  
\end{solution}
\protect \hypertarget {soln:2.15}{}
\begin{solution}{{2.15}}
\begin{enumerate}
  \item $u_t \sim \cN(0;1)$ и независимы;
  \item $u_t \sim \cN(0;1)$ и независимы при $t>1$, а при $t=1$ величина $u_t$ равновероятно
  принимает значения $-1$ и $1$;
  \item Величины $u_t$ независимы и одинаково распределены с бесконечным математическим ожиданием;
  \item $u_t \sim \cN(t;1)$ и независимы.
\end{enumerate}
\end{solution}
\protect \hypertarget {soln:2.16}{}
\begin{solution}{{2.16}}
  \begin{enumerate}
  \item $\P(z_t = 1) = 2/3$;
  \item $\E(s_t) = (t-2) \cdot 2/3$;
  \item $\Cov(z_1, z_2)$, $\Cov(z_1, z_3)$, $\Cov(z_1, z_4) = 0$;
  \item $\Var(s_t) = (16t - 29)/90$;
  \end{enumerate}

\end{solution}
\protect \hypertarget {soln:3.1}{}
\begin{solution}{{3.1}}

\end{solution}
\protect \hypertarget {soln:3.2}{}
\begin{solution}{{3.2}}

ARMA(2,3), ARIMA(2,0,3)
\end{solution}
\protect \hypertarget {soln:3.3}{}
\begin{solution}{{3.3}}
\begin{enumerate}
\item Процесс $AR(2)$, т.к. две первые частные корреляции значимо отличаются от нуля, а гипотезы о том, что каждая последующая равна нулю не отвергаются.
\item Можно использовать одну из двух статистик
\[
\text{Ljung-Box}=n(n+2)\sum_{k=1}^3\frac{\hat{\rho}_k^2}{n-k}=
0.42886
\]
\[
\text{Box-Pierce}=n\sum_{k=1}^3\hat{\rho}_k^2=
0.4076
\]
Критическое значение хи-квадрат распределения с 3-мя степенями свободы для $\alpha=0.05$ равно $\chi^2_{3,crit}=7.81$.
Вывод: гипотеза $H_0$ об отсутствии корреляции ошибок модели не отвергается.
\end{enumerate}
\end{solution}
\protect \hypertarget {soln:3.4}{}
\begin{solution}{{3.4}}

\end{solution}
\protect \hypertarget {soln:3.5}{}
\begin{solution}{{3.5}}
Среднее количество пересечений равно 50 помножить на вероятность того, что два соседних $y_t$ разного знака. Найдём вдвое меньшую вероятность, $\P(y_1>0, y_2 <0)$.
\end{solution}
\protect \hypertarget {soln:3.6}{}
\begin{solution}{{3.6}}

\[
\E(b_t) = 3
\]

\[
\Var(b_t) = t^2 \sigma^2_{\e}
\]

\[
\Cov(b_t, b_{t-k}) = 0, k \geq 1
\]

\[
\Corr(b_t, b_{t-k}) = 0, k \geq 1
\]

$b_t$ — нестационарный из-за дисперсии


\[
\E(c_t) = 2
\]

\[
\Var(c_t) = \sigma^2_{\e}
\]

\[
\Cov(c_t, c_{t-k}) = \cos( \pi k /2)\sigma^2_{\e}, k \geq 1
\]

\[
\Corr(c_t, c_{t-k}) = \cos( \pi k /2), k \geq 1
\]

$c_t$ — стационарный
\end{solution}
\protect \hypertarget {soln:3.7}{}
\begin{solution}{{3.7}}
зачеркнуть одну цифру
\end{solution}
\protect \hypertarget {soln:3.8}{}
\begin{solution}{{3.8}}
\end{solution}
\protect \hypertarget {soln:3.9}{}
\begin{solution}{{3.9}}
\end{solution}
\protect \hypertarget {soln:3.10}{}
\begin{solution}{{3.10}}
По нулевым корреляциям догадываемся, что это процесс $MA(2)$.
\[
y_y = 4 + u_t + \alpha_1 u_{t-1} + \alpha_2 u_{t-2}
\]
\[
\begin{cases}
  \frac{\alpha_1 \alpha_2 + \alpha_1}{\alpha_2} = 7/3 \\
  \frac{\alpha_1^2 + \alpha_2^2 + 1}{\alpha_2} = 10/3 \\
\end{cases}
\]


\end{solution}
\protect \hypertarget {soln:3.11}{}
\begin{solution}{{3.11}}
\end{solution}
\protect \hypertarget {soln:3.12}{}
\begin{solution}{{3.12}}
  \begin{enumerate}
    \item $ACF = (0.9, -0.9, 0, 0, 0, \ldots)$ не бывает, так как определитель корреляционной матрицы 3 на 3 отрицательный;
    \item $PACF = (0.9, -0.9, 0, 0, 0, \ldots)$ — AR(2);
    \item $PACF = (0.9, 0, 0, 0, 0, \ldots)$ — $y_t = 0.9y_{t-1} + u_t$;
    \item $PACF = (0, 0.9, 0, 0, 0, 0, \ldots)$ — $y_t = 0.9y_{t-2} + u_t$;
    \item $ACF = (0.9, 0, 0, 0, 0, \ldots)$ — не бывает, подозрение падает на MA(1), но решения только с комплексными коэффициентами, геометрически: два угла с косинусом 0.9, то есть примерно по 30 градусов, и они даже в сумме не могут дать перпендикуляр;
    \item $ACF = (0, 0.9, 0, 0, 0, 0, \ldots)$ — не бывает, если проредить процесс через один, то должна получится невозможная ACF;
  \end{enumerate}
   В целом PACF может быть любая,
   \url{http://projecteuclid.org/euclid.aos/1176342881}.
\end{solution}
\protect \hypertarget {soln:3.13}{}
\begin{solution}{{3.13}}
  $\phi_{kk}=0$ при $k \geq 3$.
\end{solution}
\protect \hypertarget {soln:3.14}{}
\begin{solution}{{3.14}}
Заметим, что $0.69\approx 0.71$, сокращаем множитель $1-0.7L$, получаем $y_t = 100/3 + \e_t$.
\end{solution}
\protect \hypertarget {soln:3.15}{}
\begin{solution}{{3.15}}
Стационарным решением является $y_t = \sum_{i=0}^{\infty} 0.5^i \e_{t-i}$. Решениями также являются: $y_t = 0.5^t + \sum_{i=0}^{\infty} 0.5^i \e_{t-i}$, $y_t = 0.5^t\e_{100} + \sum_{i=0}^{\infty} 0.5^i \e_{t-i}$, $y_t = 0.5^t + \sum_{i=0}^{t} 0.5^i \e_{t-i}$, $y_t = \sum_{i=0}^{t} 0.5^i \e_{t-i}$.
\end{solution}
\protect \hypertarget {soln:3.16}{}
\begin{solution}{{3.16}}

$\E_t(y_{t+1})=2+0.6y_{t-1}-0.08y_{t-2}$, $\Var_t(y_{t+1})=4$

$\E_t(y_{t+2})=3.2 + 0.28 y_t- 0.048y_{t-1}$, $\Var_t(y_{t+2})=1.36 \cdot 4$

$\E_{100}(y_{102})= 4.388$, $\Var_{100}(y_{102})=5.44$.

Предиктивный интервал $[4.388 - 1.96 \sqrt{5.44};4.388 + 1.96 \sqrt{5.44}]$

$\E(y_t)=\frac{2}{0.48}\approx 4.17$

\end{solution}
\protect \hypertarget {soln:3.17}{}
\begin{solution}{{3.17}}
Заметим, что $\Var(u_t|\cF_t)=0$. Более того, для обратимого процесса $\Var(u_t|y_t, y_{t-1}, \ldots, y_1) \approx \Var(u_t|y_t, y_{t-1}, \ldots) = 0$.
\[
\E(y_{101}|y_{100})=7 + 0 + 0.2\E(u_{100}|y_{100})
\]
\[
\E(u_{100}|y_{100}) = \beta_1 + \beta_2 y_{100}
\]
\[
\beta_2 = \frac{\Cov(y_{100}, u_{100})}{\Var(y_{100})}=4/4.16, \beta_1 = \E(u_{100}) - \beta_2 \E(y_{100})=-4\cdot 7/4.16
\]
\[
\frac{y_t}{1+0.2L} = \frac{7}{1+0.2L} + u_t
\]
Заметим, что $\frac{7}{1+0.2L}=7/1.2$, так как $L\cdot 7 = 7$ (вчера семь равнялось семи).

По условию $\frac{y_{100}}{1+0.2L} \approx 5.6$. Знак «примерно равно» возникает из-за замены бесконечной суммы на конечную.

\end{solution}
\protect \hypertarget {soln:3.18}{}
\begin{solution}{{3.18}}
    $\E(y_1)=0$, $\Var(y_1)=\sigma^2_u/(1-\beta^2)$, $\E(y_t|y_{t-1})=\beta y_{t-1}$, $\Var(y_t|y_{t-1})=\sigma^2_u$.

    При максимизации условного правдоподобия получаем:
    \[
         \hb = \frac{y_1 y_2 + y_2 y_3}{y_1^2 + y_2^2}
    \]
  
\end{solution}
\protect \hypertarget {soln:3.19}{}
\begin{solution}{{3.19}}
  
\end{solution}
\protect \hypertarget {soln:3.20}{}
\begin{solution}{{3.20}}
  
\end{solution}
\protect \hypertarget {soln:3.21}{}
\begin{solution}{{3.21}}
    \[
    \plim \hat \beta_2 = \frac{\beta + \alpha}{1 + \beta \alpha}
    \]
  
\end{solution}
\protect \hypertarget {soln:3.22}{}
\begin{solution}{{3.22}}
Если обозначить отношение дисперсий буквой $R = \sigma^2_w/\sigma^2_u$,
то равенство дисперсии и ковариации даёт систему уравнений:
\[
  \begin{cases}
    \alpha R = 2 \\
    (1+\alpha^2)R = 5 \\
  \end{cases}
\]
Решений у неё два, старый процесс $(\alpha=2, R=1)$, и новый $(\alpha=0.5, R=4)$.
Из равенства ожиданий следует, что $c=5$.
  
\end{solution}
\protect \hypertarget {soln:3.23}{}
\begin{solution}{{3.23}}
    Берем любое $a_0$, а дальше в обе стороны заполняем числа по принципу $a_t = -0.5 a_{t-1}$.
  
\end{solution}
\protect \hypertarget {soln:3.24}{}
\begin{solution}{{3.24}}
    Да, $w_t$ — белый шум!
  
\end{solution}
\protect \hypertarget {soln:3.25}{}
\begin{solution}{{3.25}}
    $\Corr(y_t, y_{t+1}) \in [-0.5; 0.5]$
  
\end{solution}
\protect \hypertarget {soln:4.1}{}
\begin{solution}{{4.1}}
  \[
    \hat y_{4|3} = l_3
  \]
  \[
    y_4 - \hat y_{4|3} = l_3 + \varepsilon_4 - l_3 = \varepsilon_4
  \]
  \[
  \Var(y_4 - \hat y_{4|3} \mid \mathcal{F}_3) = \Var(\varepsilon_4 \mid \mathcal{F}_3) = \Var(\varepsilon_4)
  \]
  \[
  \hat y_{5|3} = l_3
  \]
  \begin{multline}
  y_5 - \hat y_{5|3} = l_4  + \varepsilon_5 - l_3  = (l_3 + \alpha \varepsilon_4)  +
   \varepsilon_5 - l_3 = \\
  = \varepsilon_5 + \alpha  \varepsilon_4
  \end{multline}
  \[
  \Var(y_5 - \hat y_{5|3} \mid \mathcal{F}_3) = \Var(\varepsilon_5 + \alpha  \varepsilon_4 )
  \]

\end{solution}
\protect \hypertarget {soln:4.2}{}
\begin{solution}{{4.2}}
\[
\hat y_{4|3} = l_3 + b_3
\]
\[
y_4 - \hat y_{4|3} = l_3 + b_3 + \varepsilon_4 - (l_3 + b_3) = \varepsilon_4
\]
\[
\Var(y_4 - \hat y_{4|3} \mid \mathcal{F}_3) = \Var(\varepsilon_4 \mid \mathcal{F}_3) = \Var(\varepsilon_4)
\]
\[
\hat y_{5|3} = l_3 + 2b_3
\]
\begin{multline}
y_5 - \hat y_{5|3} = l_4 + b_4 + \varepsilon_5 - (l_3 + 2b_3) = (l_3 + b_3 + \alpha \varepsilon_4)  +
(b_3 + \beta \varepsilon_4) + \varepsilon_5 - (l_3 + 2b_3) = \\
= \varepsilon_5 + (\alpha + \beta) \varepsilon_4
\end{multline}
\[
\Var(y_5 - \hat y_{5|3} \mid \mathcal{F}_3) = \Var(\varepsilon_5 + (\alpha + \beta) \varepsilon_4 )
\]
\end{solution}
\protect \hypertarget {soln:4.3}{}
\begin{solution}{{4.3}}
\end{solution}
\protect \hypertarget {soln:4.4}{}
\begin{solution}{{4.4}}
\end{solution}
\protect \hypertarget {soln:4.5}{}
\begin{solution}{{4.5}}
\end{solution}
\protect \hypertarget {soln:4.6}{}
\begin{solution}{{4.6}}
^^I\begin{enumerate}
^^I^^I\item Простое экспоненциальное сглаживание, ETS-ANN; ARIMA(0,1,1)
^^I^^I\item Аддитивное сглаживание Хольта, ETS-AAN; ARIMA(0,2,2)
^^I^^I\item Аддитивное сглаживание Хольта с угасающим трендом, ETS-AAdN; ARIMA(1,1,2)
^^I^^I\item Аддитивное сглаживание Хольта-Винтерса для месячных данных, ETS-AAA; ARIMA(0,1,13)-SARIMA(0,1,0)
^^I^^I\item Аддитивное сглаживание Хольта-Винтерса с угасающим трендом для месячных данных, ETS-AAdA; ARIMA(0,1,13)-SARIMA(0,1,0)
^^I^^I\item ETS-ANA; ARIMA(0,1,12)-SARIMA(0,1,0)
^^I\end{enumerate}
\end{solution}
\protect \hypertarget {soln:4.7}{}
\begin{solution}{{4.7}}
По $l_0$, $b_0$;
\end{solution}
\protect \hypertarget {soln:4.8}{}
\begin{solution}{{4.8}}
  Только примерно, $\ln (1 + x) \approx x$.
\end{solution}
\protect \hypertarget {soln:4.9}{}
\begin{solution}{{4.9}}
\end{solution}
\protect \hypertarget {soln:4.10}{}
\begin{solution}{{4.10}}
\end{solution}
\protect \hypertarget {soln:4.11}{}
\begin{solution}{{4.11}}
\end{solution}
\protect \hypertarget {soln:4.12}{}
\begin{solution}{{4.12}}
\end{solution}
\protect \hypertarget {soln:5.1}{}
\begin{solution}{{5.1}}
  $\ln y$
\end{solution}
\protect \hypertarget {soln:6.1}{}
\begin{solution}{{6.1}}
\end{solution}
\protect \hypertarget {soln:6.2}{}
\begin{solution}{{6.2}}
    \begin{enumerate}
      \item Чтобы заниматься математикой по ночам.
      \item За поддержку якобинцев.
      \item Кофейник. Был разбит пулей.
    \end{enumerate}
\end{solution}
\protect \hypertarget {soln:6.3}{}
\begin{solution}{{6.3}}
  
\end{solution}
\protect \hypertarget {soln:6.4}{}
\begin{solution}{{6.4}}
  
\end{solution}
\protect \hypertarget {soln:6.5}{}
\begin{solution}{{6.5}}
  
\end{solution}
\protect \hypertarget {soln:6.6}{}
\begin{solution}{{6.6}}
  
\end{solution}
\protect \hypertarget {soln:6.7}{}
\begin{solution}{{6.7}}
  
\end{solution}
\protect \hypertarget {soln:6.8}{}
\begin{solution}{{6.8}}
   Да, ряды являются ортогональными. Можно строить регрессии на эти регрессоры в любых комбинациях, оценки бет выходят одни и те же.
   Другие ряды добавить нельзя — будет строгая мультиколлинеарность.
 
\end{solution}
\protect \hypertarget {soln:6.9}{}
\begin{solution}{{6.9}}
     На всякий случай, это был ряд $1$, $2$, $3$, $4$, $5$, $6$.
   
\end{solution}
\protect \hypertarget {soln:6.10}{}
\begin{solution}{{6.10}}
   $1$, $1$, $1$, $2$, $2$, $2$
   
\end{solution}
\protect \hypertarget {soln:7.1}{}
\begin{solution}{{7.1}}
\end{solution}
\protect \hypertarget {soln:7.2}{}
\begin{solution}{{7.2}}

\end{solution}
\protect \hypertarget {soln:7.3}{}
\begin{solution}{{7.3}}

\end{solution}
\protect \hypertarget {soln:7.4}{}
\begin{solution}{{7.4}}

\end{solution}
\protect \hypertarget {soln:7.5}{}
\begin{solution}{{7.5}}
$1$, $2$, $2$
\end{solution}
\protect \hypertarget {soln:7.6}{}
\begin{solution}{{7.6}}
\end{solution}
\protect \hypertarget {soln:7.7}{}
\begin{solution}{{7.7}}
\end{solution}
\protect \hypertarget {soln:7.8}{}
\begin{solution}{{7.8}}
\end{solution}
\protect \hypertarget {soln:7.9}{}
\begin{solution}{{7.9}}
\end{solution}
\protect \hypertarget {soln:7.10}{}
\begin{solution}{{7.10}}
Да, может быть и больше, и меньше.
\end{solution}
\protect \hypertarget {soln:7.11}{}
\begin{solution}{{7.11}}

\end{solution}
\protect \hypertarget {soln:7.12}{}
\begin{solution}{{7.12}}
\[
\Var(\e_t | \cF_{t-1}) = \Var(y_t| \cF_{t-1}) = 6 + 0.4 \sigma^2_{t-1} + 0.2\e_t^2
\]
\[
\Var(\e_t) = 6/(1-0.4-0.2)=6/0.4=15
\]
\[
\Var(y_t)=15/(1-0.36)
\]
\end{solution}
\protect \hypertarget {soln:8.1}{}
\begin{solution}{{8.1}}

\begin{enumerate}
\item $H_0$: ряд содержит единичный корень, $\beta=0$; $H_a$: ряд не содержит единичного корня, $\beta<0$
\item $ADF=-0.4/0.1=-4$, $ADF_{crit}=-2.89$, $H_0$ отвергается
\item Ряд стационарен
\item При верной $H_0$ ряд не стационарен, и  $t$-статистика имеет не $t$-распределение, а распределение Дики-Фуллера.
\end{enumerate}
\end{solution}
\protect \hypertarget {soln:9.1}{}
\begin{solution}{{9.1}}

\end{solution}
\protect \hypertarget {soln:9.2}{}
\begin{solution}{{9.2}}

\end{solution}
\protect \hypertarget {soln:9.3}{}
\begin{solution}{{9.3}}

$z_t$ стационарный, $x_t$ и $y_t$ не коинтегрированы
\end{solution}
\protect \hypertarget {soln:9.4}{}
\begin{solution}{{9.4}}

\end{solution}
\protect \hypertarget {soln:9.5}{}
\begin{solution}{{9.5}}
$y_t$ и $s_t$; $z_t$ и $w_t$.
\end{solution}
\protect \hypertarget {soln:9.6}{}
\begin{solution}{{9.6}}
  \begin{enumerate}
    \item $a_t = 0.5 a_{t-1} + u_t$, AR(1)
    \item $b_t = b_{t-1} + u_t$, $b_0 = 0$, ARIMA(0, 1, 0)
    \item $c_t = 0.5 b_t + \e_t$, ARIMA(0, 1, 1)
    \item $d_t = d_{t-1} + a_t$, ARIMA(1, 1, 0)
    \item $e_t = e_{t-1} + \e_t$, ARIMA(0, 1, 0)
    \item $g_t = g_{t-1} + b_t$, ARIMA(0, 2, 0)
    \item $h_t = 0.7 h_{t-1} + b_t$, ARIMA(1, 1, 0)
  \end{enumerate}
  коинтегрированы: $b_t$, $c_t$, $d_t$, $h_t$.
\end{solution}
\protect \hypertarget {soln:9.7}{}
\begin{solution}{{9.7}}
Процессы $y_t$ и $z_t$ коинтегрированы, $z_t - 1.5y_t$ стационарен.
Процессы $y_t$ и $r_t$ коинтегрированы, $r_t + 2y_t$ стационарен.
\end{solution}
\protect \hypertarget {soln:10.1}{}
\begin{solution}{{10.1}}
\end{solution}
\protect \hypertarget {soln:10.2}{}
\begin{solution}{{10.2}}
\end{solution}
\protect \hypertarget {soln:10.3}{}
\begin{solution}{{10.3}}
\end{solution}
\protect \hypertarget {soln:10.4}{}
\begin{solution}{{10.4}}
\end{solution}
